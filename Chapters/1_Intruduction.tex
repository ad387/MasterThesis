
%% Introduction page
%% =============
%%
Volcanic activities on Earth have  always shaped the earth surface and influenced atmospheric processes. Volcanoes are often particularly recognized by their dramatic consequences of a major volcanic eruption. But volcanoes influence our lives in more than this way. Volcanic gases can effect the weather (timescales of days to weeks) or the climate (timescales of months to years) \cite{schmidt2015volcanismarticle}.
Examples are the lake eruption in Iceland (1783-1784) followed by a very hot summer and a cold winter in central Europa \cite{thordarson2003atmospheric} and the Tambora eruption, indonesia in 1815 which caused the "year without summer" in 1816.\\
%
\newline
%
Considering the plate tectonics of earth  most volcanoes are caused by diverging or converging of the continental plates and therefore located at the margins of the continental plates.
Another possibility for occurrence of volcanoes is the the interior of continental or oceanic shelves. \cite{schmincke2000vulkanismus}\\
The most abundant volatile species released during a volcanic eruption are water vapour (H$_2$O; relative amount of the plume: 50\%-90\%) and carbon dioxide (CO$_2$; relative amount of the plume: 1\%-40\%) \cite{platt2015quantification}. But the short effects of those two gases are rather low since there effect on atmospheric composition is negligibly due to the high abundance of atmospheric H$_2$O and CO$_2$. But on timescales of the age of the earth the volcanic emission of H$_2$O and CO$_2$ are the source of our current atmosphere. \cite{schmidt2015volcanism}\\ 
A typically volcanic plume consists of many different gases alongside H$_2$O and CO$_2$  sulfur dioxide (SO$_2$) contributes with 1\%-25\% to the plume, hydrogen sulfide (H$_2$S) with 1\%-10\% and hydrogen chloride with (HCl) 1\%-10\%. Furthermore there are trace gases for example carbon disulfide (CS$_2$), carbon sulfide (COS) carbon monoxide (CO) hydrogen fluoride (HF) and hydrogen bromide (HBr) \cite{platt2015quantification}\\
%
A decrease of stratospheric ozone (O$_3$) has been observed after the eruption of  El Chickon in 1982 and the eruption of mount Pinatubo 1991. A depletion stratospheric O$_3$ results in ozone holes. The depletion comes from volcanic aerosols which serve anthropogenic chlorine/bromine into more reactive forms \cite{solomon1998ozone}. 
%
Volcanic gases can alter the radiative balance of the earth in timescales relevant for climate change due to scatter and absorption of solar radiation \cite{schmidt2015volcanism}.\\
%
The gas composition of the volcano plume change with activity and could be a indication for the processes inside the earth.\\ 
%
In this work we are particularly interested in the ratio of BrO and SO$_2$. The halogen sulfur ratio is a proxy for volcanic processes. Therefore we make the assumption
that the ratio of BrO and SO2 contains informations about its degassing source depth. A change in BrO/SO2 prior to eruption was observed at Etna and Nevado del Ruiz.\\
%
\newline
%
To gain further knowledge about the volcanoes the Network for Observation of Volcanic and Atmospheric Change (NOVAC) was installed. NOVAC is a Network of DOAS Instruments located next to about 30 volcanoes in America, Africa and Europe. At every Volcano there are two to four DOAS Instruments installed, recording record back-scattered solar radiation spectra at different viewing angles.\\
NOVAC is a network which produces a large amount of data and we have the chance to evaluate long time periods which is a unique opportunity to study correlations of the trace gases.\\
Since the conditions at volcanoes are rough, the instruments need to be rather simple to keep the maintenance cheap and to assure a longer lifetime of the instruments. So we need to waive on temperature stabilization even at the expense of the quality of the data.\\
%
\newline
%
One possibility to measure the volcanic trace gases is to use Differential Optical Absorption Spectroscopy \cite{platt2008differential}. DOAS exploit the wavelength dependency of the absorption of light. Here the gas emissions can be retrieved
from the quotient of the absorption signal of the volcanic plume and a
reference region. This will be explained in a further chapter.\\
%
\newline
%
The reference region, is usually treated as free of
volcanic trace gases. If the reference region is for any reason
contaminated by volcanic trace gases, the reference spectrum has to be
replaced by a volcanic-gas-free reference. Alternative spectra could be for example a
theoretical solar atlas spectrum or a volcanic-gas-free reference
spectrum recorded in the temporal proximity(eg. a day before) by the same instrument. 
The first option comes with the drawback of reduced precision, as the
instrumental effects have to be modeled and added to the retrieval. The
reduction in precision is acceptable for the SO2 retrieval, but not suitable
for a BrO retrieval because then most data would be below the detection
limit. For the second option, the alternative reference spectrum should
have been recorded at similar conditions with respect to meteorology and
radiation as well as in the temporal proximity due to instrumental changes
with time and ambient conditions. We combined both options in order to
achieve both, enhanced accuracy but still maximum possible precision of
the SO2 and BrO retrievals. We present an algorithm which finds the
optimal reference spectrum automatically. As first step, a possible SO2
contamination of the standard reference is checked by a comparison with
the theoretical solar atlas. If a contamination is detected, as second step,
the algorithm picks a volcanic-gas-free reference (beforehand
automatically checked for contamination) from another scan.\\
%
\newline
%
In this work we are mainly dealing with data from Tungurahua in Ecuador in the timespan of 01.08.2008 to 30.07.2009. Later on, we will also show the results of Nevado del Ruiz a volcano located in Colombia.\\