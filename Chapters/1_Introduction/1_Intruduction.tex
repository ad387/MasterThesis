
%% Introduction page
%% =============
%%
Volcanic activities on Earth have always shaped the Earth  surface and influenced atmospheric processes. Volcanoes are often particularly recognized by their dramatic consequences of a major volcanic eruption. But volcanoes influence our lives in more than this way. Volcanic gases can effect the weather by emitting aerosols (timescales of days to weeks) or the climate (timescales of months to years) \citet{schmidt2015volcanismarticle}.
Examples are the Laki eruption in Iceland (1783-1784) followed by a very hot summer and a cold winter in central Europa \citet{thordarson2003atmospheric} and the Tambora eruption, Indonesia in 1815 which caused the "year without summer" in 1816.\\
%
\newline
%
Volcanism is a geological phenomena which is related to the raise of magma  from the Earth's  interior to the Earth's  surface. Volcanic activity is linked to tectonic active regions, thus hotspots mainly are at the margins of the continental plates.
significantly less volcanic activities occurs at the the interior of continental or oceanic shelves \citet{schmincke2000vulkanismus}.

The most abundant volatile species released during a volcanic eruption are water vapour (H$_2$O; relative amount of the plume: 50\%-90\%) and carbon dioxide (CO$_2$; relative amount of the plume: 1\%-40\%) \citet{platt2015quantification}. 

But the short time effects of those two gases are rather small since their effect on atmospheric composition is negligibly due to the high abundance of atmospheric H$_2$O and CO$_2$. 
On higher timescales (100000 years) the volcanic emissions are relevant for preservation of carbon dioxide. 

But on timescales of the age of the Earth  the volcanic emission of H$_2$O and CO$_2$ largely contributed on the formation of our Atmosphere \citet{schmidt2015volcanism}.\\ 

A typically volcanic plume consists of many different gases alongside H$_2$O and CO$_2$  sulfur dioxide (SO$_2$) contributes with 1\%-25\% to the plume, hydrogen sulfide (H$_2$S) with 1\%-10\% and hydrogen chloride with (HCl) 1\%-10\%. Furthermore there are trace gases for example carbon disulfide (CS$_2$), carbon sulfide (COS) carbon monoxide (CO) hydrogen fluoride (HF), hydrogen bromide (HBr) and many other species \citet{platt2015quantification}.\\
%
A decrease of stratospheric ozone (O$_3$) has been observed after the eruption of  El Chickon in 1982 and the eruption of Mount Pinatubo 1991. The depletion comes from volcanic aerosols which serve to transform anthropogenic chlorine/bromine into more reactive forms \citet{solomon1998ozone}. 
%
Volcanic gases can alter the radiative balance of the Earth  due to scatter and absorption of solar radiation \citet{schmidt2015volcanism}.\\
%
The gas composition of the volcano plume change with activity and can be a indicator for the processes inside the Earth .\\ 
%
In this work \textcolor{yellow}{we} are particularly interested in the ratio of BrO and SO$_2$.\textcolor{red}{das muss bessere begründet werden. Warum diese beiden gase? warum nicht Co2? etc}
 
The BrO to \ce{SO2} ratio changes due to pressure differences with depth, thus the BrO/\ce{SO2} ratio changes with its origin source depth. It is possible to conclude from the BrO/\ce{SO2} ratio to the degassing source depth and therefore the ratio is a proxy for volcanic processes. A change in BrO/\ce{SO2} prior to eruption was observed at Etna and Nevado del Ruiz.\\
%
\newline
%
The Network for Observation of Volcanic and Atmospheric Change (NOVAC) is a network to record volcanic emissions. The aim of NOVAC is mainly to provide new parameters for risk assessment and volcanological research, both locally and on a regional and global scale.
NOVAC is a Network of spectrographic instruments located next to about 30 volcanoes in Asia, America, Africa and Europe. At each of these volcanoes are two to four spectrograph's installed, recording back-scattered solar radiation spectra at different viewing angles.\\
NOVAC is a network which produces a large amount of data and thus provides the chance to evaluate long time periods which is a unique opportunity to study volcanic trace gases.\\
The instruments and the maintenance need to be cheap, thus the instruments need to have a rather simple construction. Therefore it was decided not to implement temperature stabilization even at the expense of the quality of the data.\\This thesis utlises NOVAC-data from Tungurahua and Nevado del Ruiz.\\
%
\newline
%
The data recorded by NOVAC are evaluated using Differential Optical Absorption Spectroscopy (DOAS) \citet{platt2008differential}. DOAS utilises the wavelength dependency of the absorption of light and is based on the Beer-Lambert law. The gas concentration is retrieved from the characteristic difference in absorption structures between the plume and a reference region. Thus it is fundamental to have a reference which is free of the gases of interest, to assure that the retrieved concentration is correct.
%
\newline
%
The reference region, is usually treated as free of
volcanic trace gases. If the reference region is for any reason
contaminated by volcanic trace gases, the reference spectrum has to be
replaced by a volcanic-gas-free reference. Alternative spectra could be for example a
theoretical solar atlas spectrum or a volcanic-gas-free reference
spectrum recorded in the temporal proximity (eg. a day before) by the same instrument.
The first option comes with the drawback of reduced precision, as the
instrumental effects have to be modelled and added to the retrieval. The reduction in precision is acceptable for the \ce{SO2} retrieval, but not suitable for a BrO retrieval because then most data would be below the detection limit. The advantage of the first option is, that the absolute column density is calculated, since, the theoretical solar atlas spectrum is absolutely gas free. For the second option, the alternative reference spectrum should
have been recorded at similar conditions with respect to meteorology,
radiation,intensity and in temporal proximity due to instrumental changes
with time and ambient conditions. We combined both options in order to
achieve both, enhanced accuracy but still maximum possible precision of
the \ce{SO2} and BrO retrievals. We present an algorithm which finds the
optimal reference spectrum automatically. As first step, a possible \ce{SO2}
contamination of the standard reference is checked by a comparison with
the theoretical solar atlas. If a contamination is detected, as second step,
the algorithm picks a volcanic-gas-free reference (beforehand
automatically checked for contamination) from another scan.\\
%
\newline
%
In this work mainly uses data from Tungurahua in Ecuador and Nevado del Ruiz a volcano located in Colombia.