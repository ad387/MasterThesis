%% Latex markup und Zitate funktionieren auch hier


Die Messung der absoluten Menge und des Verhältnisses von Vulkanischen gas Emissionen geben Einsicht in magmatische Prozesse. Das Network for Observation of Volcanic and Atmospheric Change (NOVAC) verfügt über ein System an automatisierten UV-Spektrometern, welche die Gas Emissionen der Vulkane aufzeichnen. Der Ausstoß von BrO und \ce{SO2} kann mithilfe von Differenzieller optischer Absorptionsspektroskopie (DOAS) aus den aufgenommen Spekren bestimmt werden wobei die optische Absorption in der Fahne mit einem Hintergrundspektrum verglichen wird. Dies setzt voraus, dass das Hintergrund Spektrum nicht durch Vulkanische Gase beeinträchtigt ist. Typischerweise wird das Hintergrund Spektrum für einen Scan ein Höhenwinkel gewählt welcher als außerhalb der Fahne liegend identifiziert wird. Es hat sich jedoch gezeigt, dass auch diese Spektren noch durch Vulkanische Emissionen verunreinigt sein können. Als alternative Referenzspektren könnten 1) ein theoretisches Solar Atlas Spektrum oder 2) ein nicht verunreinigtes referenz Spektrum des selben Messgeräts dienen. Option 1) hat den Nachteil einer verringerten Messgenauigkeit, da Instrumenteneffekte hier modelliert werden müssen. Option 2) setzt vorraus, dass das Referenzspektrum unter ähnlichen Wetter- und Strahlungsbedingungen aufgenommen wurde. Wir verwenden die erste Methode um Kontaminierung zu identifizieren und greifen für die Bestimmung der Gas Konzentration auf die zweite Methode zurück um eine hohe fit Qualität sicher zu stellen. Stellen unsere Methode für NOVAC Daten von den Vulkanen Tungurahua und Nevado Del Ruiz vor.