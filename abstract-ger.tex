%% Latex markup und Zitate funktionieren auch hier


Die Messung der absoluten Menge und des von Konzentrationsverhältnissen von Vulkanischen Gas Emissionen geben Einsicht in magmatische Prozesse. Das Network for Observation of Volcanic and Atmospheric Change (NOVAC) besteht aus einem System von automatisierten UV-Spektrometern, welche die Gas Emissionen der Vulkane aufzeichnen. Der Ausstoß von BrO und \ce{SO2} kann mithilfe von Differenzieller optischer Absorptionsspektroskopie (DOAS) aus den aufgenommen Spekren bestimmt werden wobei die optische Absorption in der Fahne mit einem Hintergrundspektrum verglichen wird. Dies setzt voraus, dass das Hintergrundspektrum frei von Vulkanische Gasen ist. Typischerweise wird das Hintergrundspektrum für einen Scan bei einem Elevationswinkel aufgenommen welcher welcher so gewählt wird dass das Instrument nicht in die Fahne schaut. Es hat sich jedoch gezeigt, dass auch diese Spektren noch durch Vulkanische Emissionen verunreinigt sein können. Als alternative Referenzspektren könnten 1) ein theoretisches Solar Atlas Spektrum oder 2) ein nicht verunreinigtes Referenz Spektrum des selben Messgeräts dienen. Option 1) hat den Nachteil einer verringerten Messgenauigkeit, da Instrumenteneffekte hier modelliert werden müssen und ist daher nur für das typischerweise in hoher Konzentration vorkommende \ce{SO2} anwendbar. Option 2) setzt voraus, dass das Referenzspektrum unter ähnlichen Wetter- und Strahlungsbedingungen aufgenommen wurde. Wir verwenden die erste Methode um (\ce{SO2}) Kontaminierung zu identifizieren und greifen für die Bestimmung der Gas Konzentration auf die zweite Methode zurück um eine hohe Qualität der Messung sicher zu stellen. Im Folgenden stellen wir unsere Methode für NOVAC Daten von den Vulkanen Tungurahua und Nevado Del Ruiz vor.