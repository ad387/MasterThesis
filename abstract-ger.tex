%% Latex markup und Zitate funktionieren auch hier


%Die Messung der absoluten Menge und insbesondere die Konzentrationsverhältnisse vulkanischer Gas-Emissionen geben Einsicht in magmatische Prozesse.
%In dieser Arbeit wird das Konzentrationsverhältnis von BrO zu So2 untersucht. Das "Network for Observation of Volcanic and Atmospheric Change" (NOVAC) besteht aus einem System von automatisierten UV-Spektrometern, welche die Gas Emissionen von über 40 Vulkane aufzeichnen. NOVAC stellt also einen großen Datenschatz zur Verfügung.
%Die Emission von BrO und \ce{SO2} kann mithilfe von Differenzieller optischer Absorptionsspektroskopie (DOAS) aus den aufgenommen Spektren bestimmt werden. Insbesondere bei schwachen Absorbern wie BrO wird ein Referenzspektrum, welches frei von vulkanische Gasen ist benötigt, um die Frauenhofer Strukturen zu eliminieren. Typischerweise wird das Referenzspektrum für einen Scan bei einem Elevationswinkel aufgenommen welcher so gewählt wird dass das Instrument nicht in die Fahne schaut. Es ist jedoch möglich dass auch diese Spektren durch vulkanische Emissionen verunreinigt sind. Als alternative Referenzspektren könnten ein theoretisches Solar Atlas Spektrum oder ein nicht verunreinigtes Referenzspektrum des selben Messgeräts dienen. Ein Solar Atlas Spektrum hat den Nachteil einer verringerten Messgenauigkeit, da Instrumenteneffekte hier modelliert werden müssen und ist daher nur für starke Absorber wie \ce{SO2} anwendbar. Die zweite Option setzt voraus, dass das Referenzspektrum unter ähnlichen Wetter- und Strahlungsbedingungen aufgenommen wurde. Wir verwenden die erste Methode um (\ce{SO2}) Kontaminierung zu identifizieren und greifen für die Bestimmung der Gas Konzentration auf die zweite Methode zurück um eine hohe Qualität der Messung sicher zu stellen. Die zentrale Aufgabe dieser Arbeit ist es ein optimales, nicht kontaminiertes Referenzspektrum zu finden und den resultierenden BrO Fehler zu quantifizieren. Die Haupt-Vorteile der neuen Methode sind, dass die Systematische Unterschätzung der Gasmengen umgangen wird, dass aufgrund eines \ce{SO2} Thresholdes der dazu dient die Belastbarkeit der Daten zu erhöhen, ein drittel mehr Daten für die Auswertung zur Verfügung stehen. Desweiteren konnten wir nachweisen, dass auch eine BrO contamination der Referenzen auftritt. 


%Referenzspektrum bei schwachen absorbern um frauenhofer spektrum loszuwerden

%kontaminiert
%
%-> das auch in die Introduction


%Die Messung der absoluten Menge und insbesondere von Konzentrationsverhältnissen vulkanischer Gas-Emissionen geben Einsicht in magmatische Prozesse.
%In dieser Arbeit wird das Konzentrationsverhältnis von BrO zu So2 untersucht. Das Network for Observation of Volcanic and Atmospheric Change (NOVAC) besteht aus einem System von automatisierten UV-Spektrometern, welche die Gas Emissionen der Vulkane aufzeichnen.
%Die Emission von BrO und SO2 kann mithilfe von Differenzieller optischer Absorptionsspektroskopie (DOAS) aus den aufgenommen Spektren bestimmt werden. Insbesondere bei schwachen Absorbern wie BrO wird ein Referenzspektrum, welches frei von Vulkanische Gasen ist benötigt, um die Frauenhofer Strukturen zu eliminieren.
%Typischerweise wird das Referenzspektrum für einen Scan bei einem Elevationswinkel aufgenommen welcher so gewählt wird dass das Instrument nicht in die Fahne schaut.
%Es ist jedoch möglich dass auch diese Spektren durch Vulkanische Emissionen verunreinigt sind. Als alternative Referenzspektren könnten ein theoretisches Solar Atlas Spektrum oder ein nicht verunreinigtes Referenzspektrum des selben Messgeräts dienen.
%Ein Solar Atlas Spektrum hat den Nachteil einer verringerten Messgenauigkeit, da Instrumenteneffekte hier modelliert werden müssen und ist daher nur für starke Absorber wie SO2 anwendbar. Die zweite Option setzt voraus, dass das Referenzspektrum unter ähnlichen Wetter- und Strahlungsbedingungen aufgenommen wurde.
%Wir verwenden die erste Methode um (SO2) Kontaminierung zu identifizieren und greifen für die Bestimmung der Gas Konzentration auf die zweite Methode zurück um eine hohe Qualität der Messung sicher zu stellen.
%Die zentrale Aufgabe dieser Arbeit ist es ein optimales, nicht kontaminiertes Referenzspektrum zu finden und den resultierenden BrO Fehler zu quantifizieren.
%Die Haupt-Vorteile der neuen Methode sind, dass die Systematische Unterschätzung der Gasmengen umgangen wird, dass aufgrund eines SO2 Thresholdes der dazu dient die Belastbarkeit der Daten zu erhöhen, ein drittel mehr Daten für die Auswertung zur Verfügung stehen. Desweiteren konnten wir nachweisen, dass auch eine BrO contamination der Referenzen auftritt.


Die Messung der absoluten Menge vulkanischer Gas-Emissionen und insbesondere deren Konzentrationsverhältnissen geben Einsicht in magmatische Prozesse.
In dieser Arbeit wird das Konzentrationsverhältnis von BrO zu \ce{SO2} untersucht.
Für die Konzentrationsmessungen werden scannende UV-Spektrometer verwendet, welche vom "Network for Observation of Volcanic and Atmospheric Change (NOVAC)" bereit gestellt werden.
Die Konzentrationen werden mithilfe von Differenzieller Optischer Absorptionsspektroskopie(DOAS) aus den aufgenommenen Spektren bestimmt. Hierzu wird insbesondere bei schwachen Absorbern wie BrO ein Referenzspektrum, welches frei von vulkanischen Gasen ist, benötigt, um die Fraunhofer Strukturen zu eliminieren. Mit der derzeitig gängigen Auswertungsmethode ist es ist jedoch möglich, dass auch diese zur selben Zeit aufgenommenen Spektren durch vulkanische Emissionen verunreinigt sind. Als alternative Referenzspektren könnten (1) ein theoretisches Solar Atlas Spektrum oder (2) ein zeitlich versetzt aufgenommenes, nicht verunreinigtes Referenzspektrum des selben Messgeräts dienen. (1) hat den Nachteil einer verringerten Messgenauigkeit, da Instrumenteneffekte hier modelliert werden müssen, während (2) voraussetzt, dass das Referenzspektrum unter ähnlichen äußeren Bedingungen aufgenommen wurde. Im Rahmen dieser Arbeit wird eine neue Vorgehensweise vorgestellt, welche Methode (1) zur Identifizierung von (\ce{SO2}) Kontaminierung nutzt und (2) für die Bestimmung der Gas Konzentration verwendet. Die neue Methode eliminiert  die systematische Unterschätzung der Gasmengen, die Menge an verlässlichen Daten steigt um etwa ein Drittel und wir können nachweisen, dass auch eine BrO Kontaminierung der Referenzen auftritt.


