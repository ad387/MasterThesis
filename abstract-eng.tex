%% Latex markup and citations may be used here


Measurements of magnitude and composition of volcanic gas emissions allow insights in magmatic processes. Within the Network for Observation of Volcanic and Atmospheric Change(NOVAC) automatically scanning UV-spectrometers are monitoring gas emission at  volcanoes. The emissions of BrO and \ce{SO2} can be retrieved from the recorded spectra by applying Differential Optical Absorption Spectroscopy(DOAS) and comparing the optical absorption of the volcanic plume to the background. Therefore, the background spectrum must not be affected by volcanic influence. Classically, the background spectrum is taken from the same scan but from an elevation angle which has been identified to be outside of the volcanic plume. However, experience shows those background spectra can still be contaminated by volcanic gases.  Alternatively reference spectra can be derived from 1) a theoretical solar atlas spectrum or 2) a volcanic-gas-free reference spectrum recorded by the same instrument. 1) comes with a drawback of reduced precision, as the instrumental effects have to be modeled and added to the retrieval. For 2), the alternative reference spectrum should be recorded at similar conditions with respect to meteorology and radiation. We use the first option to check for contamination and the second to evaluate the spectra to maintain a god fit quality. We present our approach and its results when applied on NOVAC data from Tungurahua and Nevado Del Ruiz.
