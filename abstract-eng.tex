%% Latex markup and citations may be used here

%Measurements of magnitude and composition of volcanic gas emissions allow insights in magmatic processes. Within the Network for Observation of Volcanic and Atmospheric Change(NOVAC) automatically scanning UV-spectrometers are monitoring gas emission at volcanoes. The emissions of BrO and \ce{SO2} can be retrieved from the recorded spectra by applying Differential Optical Absorption Spectroscopy(DOAS) and comparing the optical absorption of the volcanic plume to the background. Therefore, the background spectrum must not be affected by volcanic influence. Conventionally, the background spectrum is taken from the same scan but from an elevation angle which has been identified to be outside of the volcanic plume. However, experience shows those background spectra can still be contaminated by volcanic gases.  Alternatively, background spectra can be derived from 1) a theoretical solar atlas spectrum or 2) a volcanic-gas-free background spectrum recorded by the same instrument at another time. 1) comes with a drawback of reduced precision, as the instrumental effects have to be modeled and added to the retrieval. For 2), the alternative background spectrum should be recorded at similar conditions with respect to meteorology and radiation. We use the first option to check for contamination and the second to evaluate the spectra to maintain a good fit quality. We present our approach and its results when applied on NOVAC data from Tungurahua and Nevado del Ruiz.

Measurements of magnitude and composition of volcanic gas emissions allow insights into magmatic processes. In this thesis, the concentration ratio of BrO and \ce{SO2} is analyzed. The measurements are performed with scanning UV-spectrometers provided by the "Network for Observation of Volcanic and Atmospheric Change (NOVAC)".
The concentrations are then retrieved by applying Differential Optical Absorption Spectroscopy(DOAS).
For this purpose, especially weak absorbers like BrO require a gas-free reference spectrum to eliminate the Fraunhofer structures.
However, with the conventional evaluation approach, it is still possible that the chosen same-time-reference spectra are contaminated. Alternative reference spectra could be (1) a theoretical solar atlas spectrum or (2) a temporal shifted uncontaminated reference spectrum which is recorded by the same instrument. (1) comes with the drawback of a decreased measurement precision, as instrumental effects must be modeled, while (2) only works with a reference that is recorded under similar conditions as the measurement spectrum. In this work, a new approach is presented which uses (1) for the identification of (\ce{SO2}) contamination and (2) for the actual measurement of the gas concentration. The novel approach sidesteps the systematic underestimation of the concentration and increases the amount of reliable data by approximately 30\%. Moreover, we are able to prove the occurrence of BrO contamination.