%% Template for Master thesis
%% ===========================
%%
%% You need at least KomaScript v3.0.0,
%% e.g. available in Texlive 2009
\documentclass  [
  paper    = a4,
  BCOR     = 10mm,
  twoside,
  fontsize = 12pt,
  fleqn,
  toc      = bibnumbered,
  toc      = listofnumbered,
  numbers  = noendperiod,
  headings = normal,
  listof   = leveldown,
  version  = 3.03
]                                       {scrreprt}

% used pagages
%\usepackage[margin=10pt,font=small,labelfont=bf,labelsep=endash]{caption}
\usepackage     [utf8]          {inputenc}
\usepackage     [T1]            {fontenc}
\usepackage                     {color}
\usepackage                     {amsmath}
\usepackage                     {graphicx}
\usepackage     [english]       {babel}
\usepackage                     {natbib}
\usepackage                     {hyperref}
\usepackage						{cleveref}
\usepackage						{subfigure}
\usepackage						{multirow}
\usepackage						{booktabs} % Allows the use of \toprule, \midrule and \bottomrule in tables
\usepackage						{multicol} % Required for creating multiple columns in slides

% links
\definecolor{darkblue}{rgb}{0.0,0.0,0.4}
\definecolor{darkgreen}{rgb}{0.0,0.4,0.0}
\hypersetup{
    colorlinks,
    linkcolor=black,
    citecolor=darkgreen,
    urlcolor=darkblue
}
%\renewcommand{\partname}{}
%\renewcommand{\thepart}{}

\begin{document}
  %% title pages similar to providet template instead of maketitle
  %% this will generate title pages similar to the template provided
%% by the Department of Physics and Astronomy Heidelberg
%%
%% More information:
%% http://www.physik.uni-heidelberg.de/aktuelles/studium/
%% (PDF link: ...studium/download/145/Vorlage_Diplomarbeit_Formular.pdf)

%% Titleintro
\thispagestyle{empty}
\begin{center}
  \renewcommand{\baselinestretch}{2.00}
  \Large\sffamily
  Department of Physics and Astronomy\\
  \large University of Heidelberg
  \par\vfill\normalfont
  Master thesis\\
  in Physics\\
  submitted by\\
  Elsa  Wilken\\
  born in Hamburg\\
  2018
\end{center}
\newpage

%% Titlepage
\thispagestyle{empty}
\begin{center}
  \renewcommand{\baselinestretch}{2.00}
  \Large\bfseries\sffamily
    Retrieval Advances of BrO/\ce{SO2} Molar Ratios from NOVAC\\
  \par
  \vfill
  \large\normalfont
  This Master thesis has been carried out by Elsa Wilken\\
  at the\\
  Institute for Environmental Physics, University of Heidelberg, Germany\\
  under the supervision of\\
  Prof. Ulrich Platt
  %% additionally insert second supervisor here if carrying out an
  %% external diploma thesis. Reduce vspace in L. 44 accordingly.
\end{center}\par
\vspace{5\baselineskip}

% reset baselinestretch
\renewcommand{\baselinestretch}{1.00}\normalsize % or english title page
  %% Abstract page
%% =============
%%
%% Content of abstract pages has been put into seperate pages to simplify
%% word counting. Use e.g. the unix command
%%   wc abstract-ger.tex
%% or
%%   wc abstract-eng.tex
%% to get the number of words contained in these files.
\thispagestyle{empty}
\begin{center}
\begin{minipage}[c][0.49\textheight][t]{0.95\textwidth}
	\small
	%\textbf{Optimierte Bestimmung von molaren BrO/\ce{SO2} Verhältnissen aus NOVAC Daten
	%}\par
	\textbf{Zusammenfassung}\par
	\vspace{\baselineskip}
	%% Latex markup und Zitate funktionieren auch hier


Die Messung der absoluten Menge und von Konzentrationsverhältnissen vulkanischer Gas Emissionen geben Einsicht in magmatische Prozesse. Das Network for Observation of Volcanic and Atmospheric Change (NOVAC) besteht aus einem System von automatisierten UV-Spektrometern, welche die Gas Emissionen der Vulkane aufzeichnen. Die Emission von BrO und \ce{SO2} kann mithilfe von Differenzieller optischer Absorptionsspektroskopie (DOAS) aus den aufgenommen Spekren bestimmt werden wobei die optische Absorption in der Fahne mit einem Reference Spectrum verglichen wird. Dies setzt voraus, dass das Reference Spectrum frei von Vulkanische Gasen ist. Typischerweise wird das Reference Spectrum für einen Scan bei einem Elevationswinkel aufgenommen welcher welcher so gewählt wird dass das Instrument nicht in die Fahne schaut. Es hat sich jedoch gezeigt, dass auch diese Spektren noch durch Vulkanische Emissionen verunreinigt sein können. Als alternative Referenzspektren könnten 1) ein theoretisches Solar Atlas Spektrum oder 2) ein nicht verunreinigtes Referenz Spektrum des selben Messgeräts dienen. Option 1) hat den Nachteil einer verringerten Messgenauigkeit, da Instrumenteneffekte hier modelliert werden müssen und ist daher nur für das typischerweise in hoher Konzentration vorkommende \ce{SO2} anwendbar. Option 2) setzt voraus, dass das Referenzspektrum unter ähnlichen Wetter- und Strahlungsbedingungen aufgenommen wurde. Wir verwenden die erste Methode um (\ce{SO2}) Kontaminierung zu identifizieren und greifen für die Bestimmung der Gas Konzentration auf die zweite Methode zurück um eine hohe Qualität der Messung sicher zu stellen. Im Folgenden stellen wir unsere Methode für NOVAC Daten von den Vulkanen Tungurahua und Nevado Del Ruiz vor.
\end{minipage}
\vfill
  \begin{minipage}[c][0.49\textheight][b]{0.95\textwidth}
	\small
%	\textbf{Retrieval advances of BrO/\ce{SO2} molar ratios from NOVAC
%	}\par
	\textbf{Abstract}\par
	\vspace{\baselineskip}
	%% Latex markup and citations may be used here


Measurements of magnitude and composition of volcanic gas emissions allow insights in magmatic processes. Within the Network for Observation of Volcanic and Atmospheric Change(NOVAC) automatically scanning UV-spectrometers are monitoring gas emission at  volcanoes. The emissions of BrO and \ce{SO2} can be retrieved from the recorded spectra by applying Differential Optical Absorption Spectroscopy(DOAS) and comparing the optical absorption of the volcanic plume to the background. Therefore, the background spectrum must not be affected by volcanic influence. Classically, the background spectrum is taken from the same scan but from an elevation angle which has been identified to be outside of the volcanic plume. However, experience shows those background spectra can still be contaminated by volcanic gases.  Alternatively reference spectra can be derived from 1) a theoretical solar atlas spectrum or 2) a volcanic-gas-free reference spectrum recorded by the same instrument. 1) comes with a drawback of reduced precision, as the instrumental effects have to be modeled and added to the retrieval. For 2), the alternative reference spectrum should be recorded at similar conditions with respect to meteorology and radiation. We use the first option to check for contamination and the second to evaluate the spectra to maintain a god fit quality. We present our approach and its results when applied on NOVAC data from Tungurahua and Nevado Del Ruiz.

\end{minipage}\par
%\vfill
\end{center}


  \tableofcontents
  %% Put your contents here
	\chapter{Introduction}	
	
%% Introduction page
%% =============
%%
Volcanic activities on Earth have  always shaped the earth surface and influenced atmospheric processes. Volcanoes are often particularly recognized by their dramatic consequences of a major volcanic eruption. But volcanoes influence our lives in more than this way. Volcanic gases can effect the weather (timescales of days to weeks) or the climate (timescales of months to years) \cite{schmidt2015volcanismarticle}.
Examples are the lake eruption in Iceland (1783-1784) followed by a very hot summer and a cold winter in central Europa \cite{thordarson2003atmospheric} and the Tambora eruption, indonesia in 1815 which caused the "year without summer" in 1816.\\
%
\newline
%
Considering the plate tectonics of earth  most volcanoes are caused by diverging or converging of the continental plates and therefore located at the margins of the continental plates.
Another possibility for occurrence of volcanoes is the the interior of continental or oceanic shelves. \cite{schmincke2000vulkanismus}\\
The most abundant volatile species released during a volcanic eruption are water vapour (H$_2$O; relative amount of the plume: 50\%-90\%) and carbon dioxide (CO$_2$; relative amount of the plume: 1\%-40\%) \cite{platt2015quantification}. But the short effects of those two gases are rather low since there effect on atmospheric composition is negligibly due to the high abundance of atmospheric H$_2$O and CO$_2$. But on timescales of the age of the earth the volcanic emission of H$_2$O and CO$_2$ are the source of our current atmosphere. \cite{schmidt2015volcanism}\\ 
A typically volcanic plume consists of many different gases alongside H$_2$O and CO$_2$  sulfur dioxide (SO$_2$) contributes with 1\%-25\% to the plume, hydrogen sulfide (H$_2$S) with 1\%-10\% and hydrogen chloride with (HCl) 1\%-10\%. Furthermore there are trace gases for example carbon disulfide (CS$_2$), carbon sulfide (COS) carbon monoxide (CO) hydrogen fluoride (HF) and hydrogen bromide (HBr) \cite{platt2015quantification}\\
%
A decrease of stratospheric ozone (O$_3$) has been observed after the eruption of  El Chickon in 1982 and the eruption of mount Pinatubo 1991. A depletion stratospheric O$_3$ results in ozone holes. The depletion comes from volcanic aerosols which serve anthropogenic chlorine/bromine into more reactive forms \cite{solomon1998ozone}. 
%
Volcanic gases can alter the radiative balance of the earth in timescales relevant for climate change due to scatter and absorption of solar radiation \cite{schmidt2015volcanism}.\\
%
The gas composition of the volcano plume change with activity and could be a indication for the processes inside the earth.\\ 
%
In this work we are particularly interested in the ratio of BrO and SO$_2$. The halogen sulfur ratio is a proxy for volcanic processes. Therefore we make the assumption
that the ratio of BrO and SO2 contains informations about its degassing source depth. A change in BrO/SO2 prior to eruption was observed at Etna and Nevado del Ruiz.\\
%
\newline
%
To gain further knowledge about the volcanoes the Network for Observation of Volcanic and Atmospheric Change (NOVAC) was installed. NOVAC is a Network of DOAS Instruments located next to about 30 volcanoes in America, Africa and Europe. At every Volcano there are two to four DOAS Instruments installed, recording record back-scattered solar radiation spectra at different viewing angles.\\
NOVAC is a network which produces a large amount of data and we have the chance to evaluate long time periods which is a unique opportunity to study correlations of the trace gases.\\
Since the conditions at volcanoes are rough, the instruments need to be rather simple to keep the maintenance cheap and to assure a longer lifetime of the instruments. So we need to waive on temperature stabilization even at the expense of the quality of the data.\\
%
\newline
%
One possibility to measure the volcanic trace gases is to use Differential Optical Absorption Spectroscopy \cite{platt2008differential}. DOAS exploit the wavelength dependency of the absorption of light. Here the gas emissions can be retrieved
from the quotient of the absorption signal of the volcanic plume and a
reference region. This will be explained in a further chapter.\\
%
\newline
%
The reference region, is usually treated as free of
volcanic trace gases. If the reference region is for any reason
contaminated by volcanic trace gases, the reference spectrum has to be
replaced by a volcanic-gas-free reference. Alternative spectra could be for example a
theoretical solar atlas spectrum or a volcanic-gas-free reference
spectrum recorded in the temporal proximity(eg. a day before) by the same instrument. 
The first option comes with the drawback of reduced precision, as the
instrumental effects have to be modeled and added to the retrieval. The
reduction in precision is acceptable for the SO2 retrieval, but not suitable
for a BrO retrieval because then most data would be below the detection
limit. For the second option, the alternative reference spectrum should
have been recorded at similar conditions with respect to meteorology and
radiation as well as in the temporal proximity due to instrumental changes
with time and ambient conditions. We combined both options in order to
achieve both, enhanced accuracy but still maximum possible precision of
the SO2 and BrO retrievals. We present an algorithm which finds the
optimal reference spectrum automatically. As first step, a possible SO2
contamination of the standard reference is checked by a comparison with
the theoretical solar atlas. If a contamination is detected, as second step,
the algorithm picks a volcanic-gas-free reference (beforehand
automatically checked for contamination) from another scan.\\
%
\newline
%
In this work we are mainly dealing with data from Tungurahua in Ecuador in the timespan of 01.08.2008 to 30.07.2009. Later on, we will also show the results of Nevado del Ruiz a volcano located in Colombia.\\
	\part{Theoretical Background}
	\section{Volcanism and volcanic chemistry}

	volcano, n.: 1. Physical Geogr. A hill, mountain, or other feature, typically
	conical in form, that is built up of solidified lava and rock debris and has a crater
	or vent through which, in periods of activity, molten rock (lava), rock fragments,
	steam, and gas are emitted from within the planet’s crust
	Oxford English Dictionary (2013)
	This chapter starts with a brief introduction to the reasons for volcanism and
	explains the different types of volcanism. Its purpose is to provide an overview
	of the complex processes happening during the long journey from the mantle
	and crust inside the Earth to the point at which the gases can be measured in
	the atmosphere.
	From melting rock and the magma degassing to the chemistry happening in
	the atmosphere, there are numerous processes influencing the amounts of gas and
	composition that are ultimately measured. The basics of volcanism described
	in Section 2.1 mainly follow the textbooks by Schmincke (2003), Francis and
	Oppenheimer (2004) and Frisch and Meschede (2013). The basic concepts of
	volatile degassing from the magma and which processes influence the chemical
	composition of the gases will be explained in Section 2.2. The chemistry of
	sulphur and bromine after their release into the atmosphere as well as the
	implications of volcanic degassing on the atmosphere and the Earth’s climate will
	be discussed in Section 2.3. The introductory chapter will conclude in Section 2.4
	with examples of the usefulness of volcanic gas emission measurements. An
	outlook will be given as to how measurements of the SO2 emission rate and the
	chemical composition can be used to gain knowledge about volcanic systems
	and especially how they can help to improve the accuracy of volcanic eruption
	forecasts.
	7
	\subsection{Volcanism}

	Volcanoes are the manifestation of large amounts of thermal energy inside Earth.
	A simplified depiction of the structure of Earth can be seen in Figure 2.1. Heat
	from the earth’s core is to a large extent caused by decay of long-lived radioactive
	isotopes but also remaining from potential energy released by an increasing
	gravitational pressure during the formation of Earth. A small part might be
	caused by the impact of cosmic matter that collided with the Earth over the
	4.5 billion years since the Earth’s formation. A fraction the kinetic energy from
	these impacts is still stored in the core to this day (Francis and Oppenheimer,
	2004).
	The transfer of heat from the Earth’s core towards the surface can occur
	in different ways: conduction (which is not of importance in the context of
	volcanology), convection or mantle plume volcanism. Convection in the lower
	and upper mantle leads to movement of the oceanic and continental plates (also
	called lithosphere). At the margins of the lithospheric plates, increased volcanic
	activity occurs. In the case of constructive margins (i.e., new crust is formed) this
	phenomenon is called mid-ocean ridge volcanism for oceanic plates or continental
	rift zone volcanism for continental plates. In the case of destructive plate margins
	(i.e., the plate or slab sinks back into the mantle region) it is called subduction
	zone volcanism. A third, but less frequent, path of heat transfer results from
	basaltic magmas that rise through a plume to the surface. This process leads
	to volcanism further away from plate margins, so called hot-spot volcanoes (e.g.
	the Hawaiian islands or Iceland).
	The geological setting influences the magma composition and gas content and
	therefore the shape and eruption style of a volcano. The evolution of magma
	from the point at which the rock melts to the time of an eruption influences the
	composition of the magma and - the most important aspect for this work - also
	the composition of gases released from volcanoes.
	The processes involved in the formation of mid-ocean ridge volcanoes, subduction
	zone volcanoes and hot-spot volcanoes are shown in Figure 2.1 and briefly
	summarised below:
	• Mid-ocean ridge volcanism occurs when two oceanic plates are pulled apart,
	e.g. by the sides of a plate that sinks under another plate (see below,
	subduction zone volcanoes). The spreading of the plates leads to a thinned
	oceanic lithosphere. As a result of the thinner lithosphere, material from
	the upper mantle that is solid at depths lower than 100 km ascends to
	shallower depths of approximately 50 km. At this depth, the pressure is low
	enough for the mantle material to start melting. This type of volcanism
	leads to basaltic magma (low SiO2 content, relatively few volatiles and
	low viscosity) that rises further and replenishes the gap between the two
	plates.
	• Continental rift zone volcanism is in a way similar to mid-ocean ridge
	volcanism. The process leading to volcanism in this case is a continental
	plate that splits into two diverging plates. One important example of
	continental rift zone volcanism is the Nyriagongo volcano in the Democratic
	Republic of Congo.
	• Subduction zone volcanoes are caused by an oceanic plate converging under
	another oceanic or continental plate. This type of volcanism is, for example
	responsible for the Andes. Two of the volcanoes that were examined during
	this thesis, Popocatépetl in Mexico and Nevado del Ruiz in Colombia, are
	subduction zone volcanoes. The descending plate is dense enough to sink
	back into the lower mantle and contains many water-rich minerals and
	sediments. During the descent the temperature of the plate increases due
	to an uptake of heat from the surrounding warmer mantle material and
	frictional heating. At a depth of several hundred kilometres, the plate
	will reach a temperature at which it starts melting and consolidates with
	the mantle again. Beginning at an earlier stage of approximately 80 – 150
	km, water from the water-rich descending plate evaporates and rises into
	above the descending slab causes it to melt. The result of the melting
	process is water-rich magma with more volatiles than magma from midocean
	ridge volcanoes. Most of the magma at subduction zone volcanoes
	consists mainly of andesite, but since it can consist of melt from various
	parts of the earth (subducting plate, the upper mantle and continental
	lithosphere) basaltic, dacite and rhyolite magmas can be found as well.
	The low viscosity magma at subduction zone volcanoes is one of the causes
	for the violent eruptions.
	• Hot-spot volcanoes can occur in the middle of oceanic or continental plates.
	They are a rare manifestation (only 5% of the world’s volcanoes) but
	include such important volcanoes as the Hawaiian islands, Yellowstone
	or Iceland. This type of volcano emerges when a hot spot at the coremantle
	boundary inside Earth leads to a plume in the mantle, a pathway in
	which solid material rises. When the solid material reaches approximately
	100 – 150 km depth it partially starts to melt and creates basaltic magma.
	During the further ascent the magma can interact with, e.g. continental
	crust, which in turn leads to rhyolitic, more-viscous magma. In some cases
	the solid material from the lower mantle consists of recycled lithospheric
	plates, which leads to a more complex magma.
	After magma is formed from molten rock, it can ascend further depending on
	its buoyancy. At some depth the concentration of dissolved volatiles exceed the
	solubility in the magma and a gas phase is formed. This might lead to degassing
	from the volcano.
	10
	\subsection{Volcanic degassing}
	Volcanoes emit large quantities of various gases during eruptions but also during
	phases characterized by quiescent degassing. Emissions of volcanic gases into the
	atmosphere can influence the climate and serve as a precursor of volcanic activity.
	Even more, in most cases the exsolution of volatiles from the magma is the
	driving force behind the volcanic eruption. This demonstrates the importance of
	studying volcanic gas emissions.
	When measuring volcanic gas emissions it is crucial to understand the origin of
	the gas in order to find meaningful interpretations of the results. It is necessary to
	understand how gases are exsolved from the magma, how they are transported to
	the surface and which interactions during gas ascent can alter the gas composition.
	A simple model based on Henry’s law can be used to explain degassing of volatiles
	from magma and the variability in the gas composition. Henry’s law describes
	which equilibrium concentration c of a gas can be dissolved in a liquid with a
	Henry’s constant k at a gas pressure p:
	c = k · p (2.1)
	In this model the melt starts at depth in a single liquid phase. As the magma
	rises the pressures decreases while c stays constant. When a volatile reaches
	saturation (i.e. c > k · p) a second gas phase forms in addition to the liquid melt.
	This process is referred to as first boiling. The Henry’s constant and thus the
	solubility of a volatile in the melt depend on the system’s temperature as well
	as on the chemical composition of the liquid and the volatile. Another process
	called second boiling is caused by crystallization in magma that cools down. As
	most crystals contain almost no volatiles, the crystallization process leads to
	an increased concentration of volatiles in the melt, and can lead to degassing
	without changes of the system’s pressure.
	This model can help explain the importance of measuring the ratio between
	different trace gases. Figure 2.2 shows an example of degassing due to Henry’s
	law. Figure 2.2 (a) shows the vesicularity1 of a melt as a function of depth. The
	vesicularity is shown for different starting concentrations of CO2 and for three
	values of the CO2/H2O ratio. Higher CO2 concentrations lead to exsolution of
	gas at greater depth, as does a higher CO2/H2O ratio. A magma with a start
	concentration of 1000 mg CO2 per kg melt and a CO2/H2O ratio of 5 has a
	vesicularity of 50% at a depth of approximately 300 m. At this depth practically
	all CO2 and 70% of the H2O are exsolved from the magma but only 20% of
	the SO2 and almost none of the HCl are exsolved. When ascending further the
	degassing of HCl will start at a depth of 100m while at this depth most of the
	H2O has already exsolved. The composition of gases can therefore evolve, and
	in this example the SO2/HCl ratio would decrease during magma ascent. Using
	this model gas compositions can give insights into processes at depth and the
	equilibrium temperatures at which the gas phase was formed as long as the gas
	phase remains unchanged during ascent (Edmonds, 2008).
	In reality, however, exsolved gas bubbles need to rise to the surface to be
	observed. The rise of gas to the surface can alter the chemical composition
	and lead to patterns in the observed gas emission rates. While the mechanisms
	and how they can lead to degassing patterns will be described below, it is
	first necessary to address how the rise of gas can lead to variations in the
	chemical composition: Temperature changes of the ascending gas can lead to a
	re-equilibration (i.e., SO2 reacts to H2S at decreasing temperatures) and gases
	can also interact with crustal rock or groundwater during ascending. According
	to Giggenbach (1996) the composition of gases due to processes at depth does
	not vary during “conventional observational periods”. The author claims that
	most variations observed are not caused by variations of the magma at depth
	but due to processes during the ascent of the gases. However, these processes
	and therefore also the gas composition might still vary with volcanic activity.
	After gas bubbles are formed they need to ascend for degassing at the surface
	to occur. Two different models are frequently used to describe the behaviour
	of gas bubbles after their formation: the rise speed dependent (RSD) model
	and the collapsing foam (CF) model (Parfitt, 2004). In the RSD model bubbles
	form during magma ascent. As gas bubbles always have a lower density than the
	magma they are buoyant. Depending on the magma rise speed and the viscosity
	of the magma, the RSD model can explain two mechanisms for degassing.
	In a low-viscosity magma that ascends with a low velocity, the gas phase can
	form bubbles that can rise to the surface separately from the magma. This is
	called open-system degassing. During their ascent gas bubbles can also coalesce
	and form larger bubbles that rise faster. If the gas phase can leave the magma,
	the viscosity of the magma decreases, which in turn slows down the ascent.
	In magmas with higher viscosity and a relatively high rise speed the bubbles
	might not be able to coalesce but grow by diffusion and decompression. Gas
	bubbles cannot separate from the magma, but ascend together with the magma
	until it is erupted. This process is called closed-system degassing. The formation
	of a gas phase that rises together with the magma decreases the magmas density
	and can thus be a trigger for a faster magma ascent.
	The RSD model can be used to explain periodicities in volcanic activity. A
	simple model with a one-dimensional volcanic conduit containing rising magma
	is explained in Sparks (2003a,b). In this system the magma can obtain the two
	different steady states described above, which can be summarized as:
	• Slow-ascending magma, in which bubbles can coalesce and leave the magma
	and the viscosity increases.
	• Fast-ascending magma, in which bubbles form but cannot coalesce and
	rise with together the magma. This leads to lower-density magma which
	accelerates even more.
Figure 2.3 shows the magma flow rate as a function of magma chamber pressure.
Between A and B we have the steady state where gas can leave the magma
(open-system degassing). If the pressure in the conduit increases further, caused
by material originating from the magma chamber the flow rate increases. At
point B, the magma rises too fast for bubbles to coalesce and the system jumps
directly to state C (there is no steady solution between B and D). In this regime
the magma rises too fast for bubbles to escape. The gas phase leads to a lower
density and this further accelerates the magma, which leads to an eruption of
the system. In the eruptive state between point C and D material is transported
out of the conduit faster than it is replenished from the magma chamber and
the systems pressure decreases. The decreasing pressure and flow rate lead to
point D. At point D the magma chamber pressure is too low to keep up the
explosive regime, and with a further decreasing magma chamber pressure the
system drops back to open-system degassing at point A.
In the CF model magma is stored in a magma chamber or a dyke system at
some depth (Parfitt, 2004). Bubbles form and rise to the top of the reservoir
where they form a closely packed gas foam layer. Once the foam reaches a critical
thickness, it collapses and forms a large gas bubble that rises up to the surface
driving the volcanic eruption (Vergniolle, 1996). Depending on the viscosity of
the magma either an annular flow (low viscosity) or a periodic series of gas slugs
resulting from partial collapse of the foam (high viscosity) can be observed. The
CF model can also explain periodicities in degassing patterns by the time needed
to rebuild the foam layer to a critical thickness after its collapse.
Magma does not necessarily need to erupt for degassing to occur. Degassed magma is more dense than volatile rich magma, and convection can lead to
degassing, where the degassed magma sinks back into the magma chamber. This
process can lead to substantial amounts of degassing without any erupted magma
(Kazahaya et al., 1994; Stevenson and Blake, 1998). It is also often the case that
more gas is released during an eruption than can be dissolved in the magma
that was erupted. For example, during the 1982 eruption of El Chicón, Mexico,
40 times more gas was measured than was estimated from the erupted magma
with petrological methods (Shinohara, 2008)
		\begin{figure}
		\centering
		\includegraphics[width=0.7\linewidth]{Bilder/Simon/Bilder_Tung/BrO_Plume}
		\caption{}
		\label{fig:broplume}
	\end{figure}
	\subsection{Volcanic gases and their impact on the climate}
	Volcanoes emit large amounts of gases into the atmosphere. In descending
	magnitude the most important gases are H2O, CO2, various sulphur species led
	by SO2 and H2S and halogen compounds (see Table 2.1). While water vapour is
	the most abundant gas in a volcanic plume, and responsible for the condensation
	of the plume that can be observed at many quietly degassing volcanoes from
	large distances, the total abundance is negligible when compared to evaporation
	from oceans. CO2 emissions are thought to have played an important role
	during the formation of Earth’s atmosphere. However, today CO2 emission
	from volcanoes are several orders of magnitude smaller than anthropogenic CO2
	emissions. While anthropogenic CO2 emissions were roughly 35 Gt (gigatons)
	in 2010, volcanic CO2 emission estimates ranged between 0.13 - 0.44 Gt per
	year, comparable to the CO2 emissions from countries as Pakistan (0.18Gt),
	Kazakhstan (0.25 Gt) or South Africa (0.44 Gt) (Gerlach, 2011, and references
	therein). However, on time scales of 105 a CO2 degassing has an important role
	in the Earth system, since it offsets CO2 burial in the sediments. SO2 is the
	third most abundant gas and can be measured by remote sensing in the UV. It
	is therefore often measured as a proxy for volcanic activity. The yearly global
	SO2 emissions by volcanoes range between 7.5 - 10.5TgSyr–1 (Halmer et al.,
	2002) and are therefore of the same magnitude as anthropogenic SO2 emissions
	(55TgSyr–1 for the year 2000 were given in IPCC, 2013).
	SO2 and BrO are the two trace gases in volcanic plumes that are the main topic
	of this thesis. This chapter therefore gives a brief overview over the chemistry
	of SO2 and BrO in the atmosphere after emission from the volcanic vent. The
	impacts of volcanic gas emissions on the environment and especially the influence
	of SO2 on the Earth’s climate will be discussed following Robock (2000). The
	part about SO2 summarizes parts of the review paper by Oppenheimer et al.
	(2011) and the chemistry of BrO is based on the papers of Bobrowski et al. (2007)
	and von Glasow (2010). Both parts also contain ideas from von Glasow et al.
	(2009).
	Regardless of the path of SO2 scavenging,
	dry or wet deposition or oxidation
	to sulfuric acid, the molecule
	can influence the environment in several
	ways. A local effect, especially
	compared to the influences on the
	Earth’s climate, are direct influences
	of SO2 on humans (respiratory problems,
	asthma) and plants in the areas
	affected by volcanic gases. Acid rain
	caused by volcanic plumes can prevent
	the growth of plants. Darrall (1989)
	reviewed the influences of SO2 on photosynthesis,
	and found that long-term
	exposures with SO2 concentrations between
	100 - 250 ppb can inhibit photosynthesis
	of plants. Delmelle et al.
	(2002) measured SO2 concentrations
	around Masaya volcano, Nicaragua.
	The authors found concentrations of
	up to 230 ppb and an area downwind
	of the volcano with a decreased number
	of plant communities. Similar
	studies were done by Longo et al.
	(2005), who studied SO2 concentrations
	and fine aerosol downwind of Kilauea
	volcano on Hawaii. The authors
	found high SO2 concentrations of over
	61.9 ppb with atmospheric sampling in the Kau desert south of Kilauea crater.
	Despite these lower SO2 concentrations, plant growth is largely diminished in the
	Kau desert. Fig. 2.4 gives an impression of plants in the Kau desert. It should be
	noted, though, that while SO2 certainly has negative effects on plant life other
	species in the volcanic plume (e.g., HF) are most likely deposited within close
	proximity of SO2 on the ground as well.
	Sulphur influences on the Earth’s climate
	On a global scale, SO2 and especially its oxidation product sulphuric acid have
	a large impact on Earth’s climate. The total annual SO2 emissions are low
	compared with anthropogenic emissions. Halmer et al. (2002) estimated mean
	annual S emitted as SO2 by volcanoes between 1972 and 2000 to be in the range
	between 7.5 - 10.5TgSyr–1. The estimates for anthropogenic SO2 emissions for
	the year 2000 have a mean value of 55TgSyr–1 (IPCC, 2013). However, Graf et al.
	(1997) suggested that the impact of volcanic SO2 emissions is higher, since they
	are released into the free troposphere or into the stratosphere during eruptions,
	whereas most anthropogenic emissions are released into the planetary boundary
	layer. After emission into the atmosphere SO2, is oxidised to sulphuric acid. The
	lifetime of sulphuric acid in the troposphere is only about 1 week compared with
	up to one year in the stratosphere (IPCC, 2013).
	Sulphuric acid directly influences the climate by backscattering radiation
	from the sun and thus increasing the Earth’s albedo. Even more importantly
	it serves as additional cloud-condensation nuclei (CCN), which results in more
	and finer condensed particles and therefore in whiter, more-stable clouds and
	also an increased albedo of Earth (Twomey, 1974, and Fig. 2.5). This leads
	to more back-scattered radiation (see Fig. 2.5) and more diffusively scattered radiation. The net result of the back-scattered radiation is cooling of the Earth
	atmosphere, which is the dominating radiative effect of volcanic gases (Robock,
	2000). Additionally, aerosol particles can also act as surfaces for heterogeneous
	reaction cycles that destroy ozone, especially in the stratosphere, which leads to
	an increased solar flux to the Earth. Larger aerosol particles can backscatter IR
	radiation emitted from the Earth’s surface and the lower atmosphere and thus
	slightly reduce the net cooling effect in the lower troposphere. Absorption of
	direct UV and IR radiation from the sun as well as of radiation emitted by the
	Earth leads to net heating of the stratosphere. The 2013 IPCC report (IPCC,
	2013) states that the radiative forcing caused by volcanoes was -0.11 Wm-2
	between the years 2008 and 2011 (for comparison, the radiative forcing of CO2
	was given as 1.68Wm–2). Another, counter-intuitive effect is the winter warming
	in the Northern Hemisphere, which is caused by changes of the stratospheric and
	tropospheric circulation patterns due to strong temperature gradients between
	the Arctic and equatorial regions (Robock, 2000).
	\begin{figure}
		\centering
		\includegraphics[width=0.7\linewidth]{Bilder/Simon/Bilder_Tung/Climate_Influence}
		\caption{}
		\label{fig:climateinfluence}
	\end{figure}
	
	\subsection{Volcanic plume chemistry}

	\begin{figure}
		\centering
		\includegraphics[width=0.7\linewidth]{Bilder/Simon/Bilder_Tung/BrO_Explosion}
		\caption{}
		\label{fig:broexplosion}
	\end{figure}
	
	\subsection{Sulphur species}
	2.3.1 Sulphur species
	Sulphur species, mainly SO2 and H2S are the third most abundant species in
	volcanic plumes after H2O and CO2. SO2 contributes up to 25% of the total
	volume of the volcanic plume and H2S ranges between 1 - 10% (see Table 2.1).
	As opposed to the two most dominant species, SO2 does not have considerable
	background levels in the atmosphere. The SO2 concentration in the atmosphere
	is usually below 1 ppb while the concentration in a volcanic plume can easily
	achieve levels of 1 ppm (Oppenheimer, 2003a). The low background values
	together with the stability and the relatively strong UV absorption make SO2
	an easily accessible tracer for volcanic gas emissions (see next Section 2.4).
	After their release from the reducing conditions in the volcanic vent the gases
	enter the atmosphere, which is an oxidising environment. H2S is slowly oxidized
	to SO2 when released into the atmosphere. The first step is the reaction of SH
%	with OH:2
%	H2S + OH −−−−−−−!
%	k=4.7×10−12 H2O + SH (C 2.1)
	The SH radical then undergoes a series of reactions in the atmosphere that
	lead to the formation of SO2 (Seinfeld and Pandis, 2006). Theoretically H2S
	can react with the halogen radicals Cl and Br to form SH, which results in H2S
	lifetimes of only a few seconds (Aiuppa et al., 2007). However, Aiuppa et al.
	(2007) could not reproduce these low H2S lifetimes in measurements that showed
	a constant SO2/H2S ratio at distances up to several kilometres from the vent.
	SO2 is removed from the atmosphere by wet and dry deposition or oxidized to
	sulphuric acid. In the gaseous phase the reaction with OH to sulphuric acid is
	rather slow:
%	SO2 + OH M
%	−−−−−−−!
%	k=1.3×10−12 HSO3 (C 2.2)
%	HSO3 + O2 −−−−−−−!
%	k=4.3×10−13 HO2 + SO3 (C 2.3)
%	SO3 + H2O M
%	−−−−−−−−!
%	k=5.7×104 s−1 H2SO4 (C 2.4)
	The rate constant k in Eq. C2.4 is given for 50% humidity in Atkinson
	et al. (2004). Möller (1980) calculated an SO2 loss rate of k = 1.2 10-6 s-1 for
	homogeneous reactions in the gas phase, or converted to hours a loss of 0.43%
	SO2 per hour. In heterogeneous reactions, such as reactions of SO2 on particles
	or in the liquid phase of particles, the oxidation process is much faster. Möller
	(1980) gave loss rates of k = 0.1- 1 10-5 s-1 for reactions of SO2 on aerosol
	particles and k = 5 10-5 s-1 for liquid phase oxidation. The equations for the reaction of SO2 in the liquid phase are (Möller, 1980):
%	SO2,g + H2Oaq −)−−*− (SO2 · H2O)aq (C2.5)
%	(SO2 · H2O)aq + H2O −−* )−− HSO−3,aq + H3O+
%	aq (C 2.6)
%	HSO−3 + H2O −)−−*− SO2−
%	3 + H3O+ (C 2.7)
%	HSO–3 and SO2–
	3 can be further oxidized to sulphuric acid or sulphate, e.g. by a
	reaction with OH (Platt and Stutz, 2008). The oxidation of SO2 can also take
	place on aerosol particles by reactions with O3, H2O2, HOCl, HOBr or oxygen if
	catalysts (iron, manganes) are available (von Glasow et al., 2009).
	Considering these different pathways and regarding the importance of SO2 as
	a tracer for volcanic activity and for the comparison of SO2 with other trace
	gases, it is worthwhile to compare measured SO2 loss rates with the values given
	above. There are only a few reports of high SO2 loss rates. Jaeschke et al.
%	(1982) and Oppenheimer et al. (1998) reported loss rates between 110−3 s−1
	and 7  10-4 s-1, which would lead to an e-folding time between only 16 and
	23 minutes. However, older data listed in Oppenheimer et al. (1998) showed
%	loss rates between 5  10−5 s−1 and 210−7 s−1, therefore leading to e-folding
	times between 5 hours and several weeks. McGonigle et al. (2004) studied the
	depletion rate of SO2 using the DOAS technique at Masaya volcano, Nicaragua.
	The authors found that there are negligible variations of up to 500 s - 2000 s
	after release from the volcano. Furthermore, the authors argue that Jaeschke
	et al. (1982) had problems with using CO2 as a tracer to compare the SO2
	concentration. Meanwhile Oppenheimer et al. (1998) based the calculated loss
	rates on only a few traverse measurements, thus being influenced by variations
	in the total emission rate from the volcano as well as errors from wind speed and
	direction. Oppenheimer et al. (1998) measured ash plumes during the wet season
	at Soufriere Hills, Montserrat. Repeated measurements at the same volcano with
	more traverses during the dry season and ash-free volcanic plumes, found an one
	order of magnitude slower SO2 loss rate (Rodríguez et al., 2008). Recent studies
	from satellites by Beirle et al. (2013) investigated the SO2 loss rates at Kilauea,
	Hawaii. The authors found mean monthly SO2 life times between 16 and 57 h,
	with the longer lifetimes in summer, when the cloud coverage is lower. While
	satellites are not able to assess potential SO2 loss close to the volcanic vent due
	to restrictions of their spatial and time resolution, this study still shows slow
	SO2 loss rates with robust observations over a long time-scale.
	In this work it is assumed that SO2 is stable during the time frame that usually
	occurs during ground-based remote sensing measurements (a few up to several
	tens of minutes during this thesis).
	\subsection{Bromine oxide}
		2.3.2 Bromine oxide
	Bromine oxide (BrO) was detected for the first time at Soufriere Hills volcano
	(Montserrat) by Bobrowski et al. (2003). Since then, BrO has been detected
	at many volcanoes by ground-based measurements (e.g., Bobrowski and Platt,
	2007; Bobrowski et al., 2007; Oppenheimer et al., 2006; Vogel, 2011) or from
	airborne platforms (Heue et al., 2011; Kelly et al., 2013). Theys et al. (2009)
	were able to detect BrO from satellite after the Kasatochi eruption in 2008, and
	Hörmann et al. (2013) were able to find BrO in volcanic plumes from 11 erupting
	volcanoes with data from the GOME-2 instrument.
	However, it is actually mainly HBr, not BrO, that is emitted from volcanoes.
	BrO is formed after the volcanic gas mixes with ozone-rich ambient air. Gerlach
	(2004) and Martin et al. (2006) used thermodynamical equilibrium calculations
	to assess the source of bromine and found that there is not enough Br or BrO
	in hot magmatic gases to explain the BrO concentration measured, e.g. at
	Soufriere Hills by Bobrowski and Platt (2007). Instead, the volcanic gases mix
	with atmospheric air at high temperatures, in the so-called effective source region
	(Bobrowski et al., 2007), see Figure 2.6. This can be compared to a chimney,
	where hot air released from the volcano pulls in atmospheric air through the
	permeable edifice (Gerlach, 2004). The mixing process at high temperatures
 changes the gas composition and leads to an increase in bromine
	species other than HBr such as Br, Br2 and BrO (Martin et al., 2006; von Glasow,
	2010). Further mixing with atmospheric air leads to cooling down to ambient
	temperatures where the so-called Bromine Explosion starts. The term Bromine
	Explosion originates from BrO observations in polar regions, where a similar
	non-linear increase in BrO concentration has been observed (Hönninger and
	Platt, 2002; Lehrer et al., 1997; Wennberg, 1999).
	The Bromine Explosion mechanism can be summarized as:
%	HBrgas −−! Br−
%	aq + H+
%	aq (C 2.8)
%	HOBr(gas) −−! HOBr(aq) (C 2.9)
%	HOBr(aq) + Br−
%	(aq) + H+
%	(aq) −−! Br2(aq) + H2O (C 2.10)
%	Br2,aq −−! Br2,gas (C 2.11)
%	Br2 + h −−! 2 Br (C 2.12)
%	Br + O3 −−! BrO + O2 (C 2.13)
%	BrO + HO2 −−! HOBr + O2 (C 2.14)
%	BrO + BrO −−! 2 Br + O2 (C 2.15)
%	BrO + BrO −−! Br2 + O2 (C 2.16)
	C2.8 and C2.9 describe the uptake of HBr and HOBr on the surface of an
	aerosol particle and subsequent transformation into the liquid phase. At low pH
	< 6.5 (Fickert et al., 1999) or at cold temperatures H+, Br– and HOBr react to
	form Br2 (C 2.10) that is then released back into the gas phase (C 2.11). Br2
	reacts to Br via photolysis if sunlight is available (C 2.12). The necessity of solar
	radiation for the Bromine Explosion in volcanic plumes was verified by Kern et al.
	(2009), who could measure elevated BrO during daytime, but not at night at
	Masaya volcano, Nicaragua. In the next step Br that was formed from Br2 and
	O3 reacts to BrO (C 2.13). The role of O3 in Eq.C2.13 was shown by Bobrowski
	et al. (2007) and Louban et al. (2009) who measured higher BrO/SO2 ratios at
	the edges of the volcanic plume rather than in the middle of the plume, where
	less ambient (ozone-rich) air is available. Kelly et al. (2013) was able to measure
	the evolution of ozone depletion in volcanic plumes from an airborne platform.
	BrO can react with HO2 to form HOBr again leading back from C2.15 to C2.9.
	The result of this reaction cycle is the formation of BrO and the destruction of
	O3. BrO can also react with another BrO molecule to form Br2 or Br (C. 2.15
	and 2.16), which in turn would react again to BrO. The Bromine Explosion
	reaction cycle is schematically shown in Figure 2.7.
	Considering the multitude of chemical reactions that take part in the formation
	of BrO and because BrO is a secondary volcanic gas, there is some discussion as to
	whether BrO/SO2 ratio can be a useful indicator of volcano activity. If BrO/SO2 ratios are used as an indicator of volcanic activity, influences from the formation
	of BrO in the atmosphere have to be ruled out or characterized carefully. SO2 can
	be regarded as stable during times that are typically observed in ground-based
	remote sensing observations (see above). Perspectives on the time scales of BrO
	formation have changed over the last couple of years. Bobrowski et al. (2007)
	and Roberts et al. (2009) estimated an increase in the BrO/SO2 ratio up to
	several tens of minutes after release from the vent. Newer publications and an
	improving dataset led Bobrowski and Giuffrida (2012) to the conclusion that the
	BrO/SO2 ratio increases in the first minutes after release from the volcano, and
	then stays constant for some time (see Figure 2.8). Vogel (2011) and Gliß (2013)
	were able to measure the evolution of the BrO/SO2 ratio in the young plume of
	Pacaya volcano and Etna volcano respectively using horizontal DOAS scanning
	measurements and found a strong increase within the first five minutes after
	release. Vogel (2011) also measured the BrO/SO2 ratios in ageing plumes at Mt.
	Etna and found a constant ratio for plume ages of up to two hours. It has not
	been sufficiently examined if the BrO/SO2 ratio depends on the ratio of Br/S
	exsolved from the magma. Nevertheless, it is import to rule out influences of
	atmospheric chemistry in order to find a meaningful interpretation of BrO/SO2
	ratios. The influence of the time following the mixing of volcanic gas with the
	ambient atmosphere will be discussed in Chapter 5.
	\subsection{Using volcanic gases to study volcanic activity}
	\begin{figure}
		\centering
		\includegraphics[width=0.7\linewidth]{Zwischenbericht2018/Bilder/so2_bro}
		\caption{}
		\label{fig:so2bro}
	\end{figure}
	
	2.4 Using volcanic gases to study volcanic activity
	There are many factors influencing the degassing of volatiles and the composition
	of gases after and during the release from the volcanic vent. Furthermore the
	plume composition changes when it reacts with the oxidizing atmosphere. Therefore
	interpreting volcanic degassing data is complex. For example, Oppenheimer
	et al. (2011) reported in their review paper on SO2 degassing, that decreasing
	SO2 fluxes have been ascribed to:
	• depletion of volatiles in a magma body
	• decreased permeability of the magma in the conduit
	The first process can be regarded as an indicator of decreasing volcanic activity.
	The second process can be caused by, for example, the formation of a plug when
	degassed magma starts crystallization. The plug leads to a build up of gas
	pressure and ultimately might lead to an eruption (Clarke, 2013). Therefore the
	observation of decreasing SO2 emission rates can indicate an increase as well as
	a decrease of volcanic activity.
	Despite these complications, measurements of volcanic gases can be an important
	additional tool for the forecasting of volcanic activity to accompany
	the classic monitoring techniques like seismicity or deformation measurements.
	The remote sensing of volcanic gases started with the correlation spectrometer
	COSPEC (Moffat and Millan, 1971; Stoiber et al., 1983) in the 1970 s. COSPEC 
	measurements together with fumarolic sampling helped, for instance, predict the
	eruptions of Mount St. Helens, USA, in August 1980 and June 1981 (Casadevall
	et al., 1983) or Pinatubo, Philippines, in 1991. The remote sensing of volcanic
	SO2 experienced further spreading world-wide with the availability of miniature
	spectrometers (Galle et al., 2003) that measure the SO2 column density by
	using Differential Optical Absorption Spectroscopy (DOAS, Platt and Stutz,
	2008). Besides being smaller and cheaper, these miniature spectrometers using
	the DOAS technique have additional advantages for fieldwork in rugged
	environments, as they are lighter and consume less power. Additionally the
	availability of more spectral information allows for the correction of radiative
	transfer problems (Kern, 2009, and also Chapter 3) and to retrieve other trace
	gases (e.g. O3, NO2 or BrO). An important step towards continuous monitoring
	of volcanic SO2 emission rates was made with the Network for Observation
	of Volcanic and Atmospheric Change (NOVAC, Galle et al., 2010). During
	this EU-funded project scanning DOAS instruments were installed at several
	volcanoes world-wide. The spectroscopic data for the BrO/SO2 ratios evaluated
	within this thesis were recorded by the NOVAC network, which will be discussed
	in more detail in Chapter 5. Besides the monitoring instruments and the network
	itself, the detection of abnormally high SO2 emission rates at Santa Ana volcano
	(El Salvador) before the eruption in 2005 was the first big success of the project
	(Olmos et al., 2007). These measurements helped to plan the evacuation of
	thousands of people living in the vicinity of Santa Ana volcano. Scanning DOAS
	instruments need approximately 5 – 15 minutes for one scan across the complete
	sky. These techniques are therefore rather suited to measure long-term variations
	in the SO2 emission rates.
	A lot of additional information can be gained when using instruments that
	allow emission rate measurements with a higher time resolution. Boichu et al.
	(2010) used two spectrometers with an optical system that led to a wide fieldof-
	view covering the complete volcanic vent at Mount Erebus, Antartica. This
	set-up allowed SO2 emission rate measurements with a time resolution of the
	order of 1 Hz. Frequency analysis of the data revealed patterns with periods
	between 11 and 24 minutes that allowed discussion of different models for the
	degassing at Mt. Erebus. The SO2 camera, which is also used in large parts of
	this thesis, makes SO2 emission rate measurements with a similar time resolution
	possible. Holland et al. (2011) used SO2 camera measurements at Santiaguito,
	Guatemala to identify shear-fracturing as the main process leading to cyclic
	patterns of explosive eruptions rather than building of a viscous plug. Tamburello
	et al. (2013) applied SO2 cameras at Etna, Italy, using a similar approach as
	Boichu et al. (2010), and suggested that periodic signals with periods between
	40 – 250 s and 500 – 1200 s are caused by bursting of rising gas bubbles. First
	steps in the direction of the analysis of high time resolution SO2 camera data
	are presented in Chapter 4 of this thesis.
	In addition to the SO2 emission rate, the ratio between different trace gases can
	be indicative of volcanic activity as well. The solubility of volatiles depends, e.g.
	on depth and chemical composition of the magma (see Chapter 2.1), therefore
	the composition of volcanic gases has been studied in depth. A good introduction
	to gas compositions and their interpretation can be found in Giggenbach (1996).
	A more recent review article about halogens in volcanic system was published by
	Aiuppa et al. (2009). Initial studies (e.g. Noguchi and Kamiya, 1963) investigated
	the Cl/S ratio by direct sampling. The authors found a decrease of the Cl/S
	ratio before an eruptive period (Noguchi and Kamiya, 1963; Stoiber and Rose,
	1970). Later Pennisi and Le Cloarec (1998) found that Cl and F degas in a
	similar manner, while the Cl/S ratio varied between non-eruptive and eruptive
	periods. The authors argue that chlorine is exsolved from the magma at deeper
	levels (Cl/S > 1) while at shallower levels sulphur degassing dominates (Cl/S
%	 0.1). Burton et al. (2007) measured volcanic gas composition (H2O, CO2,
	CO, SO2 and HCl) by remote sensing using an open-path Fourier transform
	infrared spectrometer (OP-FTIR) at Stromboli, Italy. The authors found that
	the ratio of CO2/SO2 and SO2/HCl were 3-5 times higher during explosions than
	during quiescent degassing. These observations paired with higher equilibrium
	temperatures lead the authors to the conclusion that the explosions are driven by
	gas slugs that originate from a deeper level than gas observed during quiescent
	periods. The authors then used melt inclusion data and a chemical model
	simulating degassing of magma to interpret the data. The S/Cl and H2O/CO2
	ratios show that bigger explosions originate from a depth of approximately
	2.7 – 3km while smaller eruptions originate from depths as shallow as 0.8 km
	below the crater at sea level.
	The BrO/SO2 ratio has been measured using DOAS remote sensing measurements
	by several authors (e.g., Bobrowski and Platt, 2007; Bobrowski et al.,
	2003, 2007; Hörmann et al., 2013; Kelly et al., 2013; Oppenheimer et al., 2006;
	Theys et al., 2009; Vogel, 2011). However, despite naming the possibility to
	use BrO/SO2 ratios as an additional tracer for volcanic activity, most of these
	studies focused on the formation of BrO in the atmosphere. First long-time
	measurements at Etna, Italy, were published by Bobrowski and Giuffrida (2012).
	The authors measured the BrO/SO2 ratio covering the years 2005 - 2009 and
	found higher ratios during non-eruptive periods. During four eruptive periods
	BrO/SO2 ratios were higher three months before the eruption compared to the
	ratios observed one month before the eruption. Studies on the seasonal variability
	and the influence of humidity did not reveal obvious correlations. The authors
	suggested a model in which bromine is released earlier than sulphur during the
	magma’s ascent to explain the observed behaviour.
	\section{Remote sensing of volcanic gases}
	Remote sensing of volcanic gases
	In this thesis, SO2 and BrO in volcanic plumes are measured with two different
	remote sensing techniques, Differential Optical Absorption Spectroscopy (DOAS,
	Platt and Stutz, 2008) and SO2 camera. Both techniques make it possible to
	measure volcanic gases by examining their interaction with radiation. This
	chapter will briefly describe absorption and scattering of radiation in the atmosphere
	in Section 3.1. The concepts of the two measurement techniques will
	be described in Sections 3.2 and 3.3. The chapter concludes with the effects of
	radiative transfer on remote sensing measurements in Section 3.4.
	\subsection{Absorption spectroscopy}
	\subsection*{Beer-Lambert Law}
	Atoms and molecules exist in several energy states with varying electron configuration.
	Molecules can additionally have different rotational and vibrational
	states. For simplicity, the basics of radiation transport in the atmosphere will
	be explained only for molecules in the following paragraphs.
	If the energy of a photon matches the energy gap between two states of a
	molecule, the molecule can absorb the photon and enter the more energetic
	state. The transition can only occur if the lower energetic state is occupied and
	if the selection rules are fulfilled. The molecule can return to a lower state by
	collision with other molecules or emission of a photon. However, the direction
	of the emitted photon is usually not the direction of the incoming radiation.
	Therefore, the radiation intensity measured behind an absorbing medium (e.g.
	gases) decreases with increasing light-path through the medium.
	This decrease is described by the Beer-Lambert law. For radiation of wave-
	length  and an initial intensity I0 traversing a medium, the Beer-Lambert
	law gives the light intensity Iafter passing through a path of length L,

	where c(l) is the location-dependent concentration of the trace gas of interest
	and  is its absorption cross section. The absorption cross section is
	unique for each molecule and depends on pressure p as well as on temperature T.
	The quantity measured with many optical remote sensing techniques is the
	optical density ,

	where S is the column density, the concentration of the trace gas integrated
	along the light-path:

	The concentration c for a constant trace gas distribution can be directly calculated
	from the column density S if the path length L is known.
	For measurements in the atmosphere, the situation is more complex, with
	a multitude of different absorbers and scattering processes, such as Rayleigh
	scattering (see Section 3.1.2) and Mie scattering (see Section 3.1.2) that have
	to be taken into account. This is done by treating the scattering effects as
	pseudo-absorbers with their respective extinction coefficients R for Rayleigh
	scattering and M for Mie scattering. The extended Beer-Lambert Law reads:

	This equation is valid for radiation traversing a medium of total length L, with
	several absorbing species j that can have variable concentrations cj depending
	on the position l in the light path. The first two terms in the exponential
	function in Eq. 3.4 describe the extinction due to Rayleigh and Mie scattering in
	the atmosphere; the third term describes the absorption of various molecules.
	For simplicity, inelastic scattering and effects arising from turbulences in the
	atmosphere are neglected here. The most important effect of inelastic scattering,
	the Ring effect, will be described in Section 3.1.2 and 3.2.2.
	\subsection{Scattering processes in the atmosphere}
Rayleigh scattering
Rayleigh scattering describes elastic scattering (the photon energy does not
change) on particles much smaller than the wavelength of the incident radiation.
It has a strong wavelength dependency, scattering at shorter wavelengths is
stronger, which leads to the blue colour of the sky. Platt and Stutz (2008) gave
a simplified estimate of the Rayleigh scattering cross section:

Rayleigh scattering is (as the name implies) a scattering and not an absorption
process. However, in the narrow-beam approximation it can be treated as an
absorption process. This approximation assumes that the probability of a photon
that was scattered out of the light beam is scattered back into the light beam
is negligible. The Rayleigh extinction coefficient R for a number density of air
molecules Nair is given by:

The angular distribution of Rayleigh-scattered photons is given by the Rayleigh
phase function:

I0
(3.7)
Forward or backward scattering is stronger by up to a factor of 2, when
compared to a scattering angle of  = 9
Scattering and absorption by particles
Photons can also interact with particles that have a size comparable to the
wavelength of the incident radiation (via scattering or absorption). For spherical
particles this process is described by the Mie theory. Because these particles can
absorb and scatter radiation, the Mie extinction coefficient is divided into two
parts

with the particle radius r, the scattering coefficient s and the absorption
coefficient a. Mie theory in general is very complex, even more for complex
particle shapes. Therefore, only a few general remarks from Platt and Stutz
(2008) will be given here.
The single scattering albedo (SSA) of an aerosol is defined as:

It defines the amount of radiation scattered by Mie particles, compared to the
amount that was scattered or absorbed. An SSA of 1 describes pure scattering,
whereas an SSA of 0 describes an aerosol that absorbs all radiation.
Mie scattering is, similar to Rayleigh scattering, not an absorption process;
however, it can be described as an absorption process in the narrow-beam
approximation as well. The wavelength dependency of the Mie extinction
coefficient M is:

%Here  is the angstroem exponent (angstroeem, 1929, 1961), which is inversely
related to the size of the aerosol particles. In general, Mie scattering exhibits a
smaller wavelength dependency then Rayleigh scattering. Spinetti and Buongiorno
(2007) measured the optical properties of aerosols in the plume of Etna,
Italy and found angstroem exponents between 0.13 and 2.42 during episodes of
quiescent degassing.
The Mie phase function, and thus the scattering direction, depends on the
%size parameter s:


(3.11)
In general it can be said, that forward scattering is more dominant in Mie
scattering when compared to Rayleigh scattering, especially for increasing values
%of s.
Raman scattering
In addition to elastic scattering, inelastic Raman scattering (i.e. the energy of the
photon changes during the scattering process) may also occur in the atmosphere
if the atom/molecule changes its excitation state during the scattering process.
The photon can transfer energy to the molecule (Stokes process) or take up energy
from the molecule (anti-Stokes process). If only the rotational state  of
the molecule changes the process is called Rotational Raman scattering is used.
When the vibrational excitation state changes as well, it is called rotationalvibrational
Raman scattering. The cross sections for Raman scattering are orders
of magnitudes smaller than those for Rayleigh scattering (Platt and Stutz, 2008).
However, the influence of Raman scattering has to be considered for remote
sensing applications. Due to the energy exchange during the scattering event,
the incident photon has a different wavelength after the scattering process. This
causes an effect known as the “Ring effect”, which is named after Grainger and
Ring (1962). The Ring effect describes a decrease in the optical depth of the
Fraunhofer lines due to photons that are inelastically scattered. The Ring effect
is described in more detail in Section 3.2.2.
	\subsection{Differential Optical Absorption Spectroscopy(DOAS)\label{DOAS}}
	Eq. 3.4 cannot be applied to real measurements because obtaining the background
	intensity I0 without absorbers would require removing the atmosphere.
	Additionally it is impossible to distinguish between the various broad-band
	effects, like scattering in the atmosphere or instrumental factors that influence
	the measured spectra. Differential Optical Absorption Spectroscopy (DOAS)
	is a technique that was invented in the late 1970s by Perner and Platt (1979)
	and that overcomes these drawbacks by using only the narrow-band absorption
	features of molecules to measure their column densities. This chapter will give
	a brief overview of the basic concepts applied in this thesis. More detailed
	information can be found in the comprehensive work by Platt and Stutz (2008),
	which was the foundation for this section.
	3.2.1 The DOAS principle
	The DOAS technique takes advantage of the fact that scattering effects as well
	as instrumental properties (i.e. the wavelength dependency of the spectrometer’s
	optical system and grating) have a broad-band structure. The absorption cross
	section of trace gases on the other hand has spectral broad-band as well as narrow-band features. It can be divided into the following two parts: a broadband
	part  that only varies weakly with wavelength and a narrow-band
	part which varies strongly with wavelength:

	The Beer-Lambert law (Eq. 3.4) can now be rewritten with the exponential
	function separated into one part that contains all broad-band effects (e.g.,
	scattering as well as broad-band absorption features) and another part that only
	contains the narrow-band absorption features of the trace gases:

	The newly defined intensity I0
is the intensity without differential absorption,
	it differs from I0 only in broad-band structures. Using I0
	0, the differential
	optical density can be defined (note that in this equation the differential
	absorption cross sectionsj are used, see also Fig. 3.1):

	Eq. 3.14 can now be solved for the column densities Sj . For the simple case of
	only one trace gas with constant concentration c and a well-known light path L,
	the concentration can be directly calculated from 

	(3.15)
	In a DOAS measurement only the difference of the column density SM in the
	measurement spectrum  and the column density SR of the
	reference spectrum  is obtained. The optical density is given
	by:

	In general for two spectra the obtained column density is called differential
	slant column density dS. In the case that the reference reference spectrum IR
	does not contain the trace gas of interest (SR = 0) it is called the slant column
	density (SCD), the column density along some light path. In volcanological setting the ideal light-path is a straight line through the volcanic plume. In this
	thesis, it is usually assumed that the reference spectrum is gas free. Whenever
	the term column density is used it refers to the SCD. Only in Chapter 8 the
	distinction between dS and the SCD is explicitly made.
	3.2.2 Technical implementation of the DOAS approach
	The theory described above is based on an ideal instrument. In reality, instruments
	have limitations; spectrometers only have a finite optical and spectral
	resolution and temperature dependencies can further influence the signal. The
	chapter above also neglects the effect of inelastic scattering that causes the Ring
	effect and also the solar I0 effect, an effect that influences the convolution of the
	high-resolution reference cross sections. All these effects will be described in
	this section. The section ends with a brief description of the DOASIS software’s
	fitting routine (Kraus, 2006), which was used for the evaluation of spectroscopic
	data in this thesis.
	Optical and spectral resolution of the spectrometer
	The optical and spectral resolution of spectrometers is finite. This leads to a
	spectrum arriving at the detector that can be described as the convolution
	of the incident spectrum  with the instrument line function (ILF) :

	To correctly retrieve the column densities Sj , all reference cross sections j
	used in Eq. 3.15 must have the same spectral resolution as the instrument used for
	recording the spectra. This can be achieved, for example, by recording absorption
	spectra and placing a calibration cell containing a known concentration of the
	trace gas of interest in the light path. However, this is difficult to achieve in many
	cases. Some species are too chemically reactive (e.g. the equilibrium between
	NO2 and N2O4 depends on the temperature) and additionally calibration cells
	tend to leak. For O3, the temperature dependency of the absorption cross
	section presents another problem. Therefore, cross sections 
 at instrument
	resolution are often calculated from high-resolution absorption cross sections
 measured in a laboratory by convolving them with the ILF H():

	In DOASIS, the convolved spectrum is interpolated to the spectral grid of the
	spectrometer after the convolution. Equation 3.18 is only an approximation that
	can be used for small optical densities. A more accurate solution is shown in the
	subsection about the solar I0 effect.
	The ILF H that is needed for the convolution can be approximated by
	measuring the lines of a Mercury lamp. The spectral width of these lines is typically only a few pm. Therefore, these lines can be seen as a delta peak in
	good approximation when compared to the resolution of a typical spectrometer
	(on the order of 0.6nm for the instruments used in this thesis).
	Effects from the detector
	The spectrumis recorded by a detector that only has discrete pixels. Therefore,
	a wavelength-interval is mapped onto a pixel i:
	I0(i) =

	For the DOAS retrieval, it is important to know the relationship between the
	channels of the detector and the wavelength of the spectrum. The so-called
	wavelength-pixel-mapping for a detector with q channels can be described with
	a polynomial:

	The parameters 
k describe which wavelength corresponds to which pixel. 
0
	corresponds to a shift of the spectrum, and a change in 
1 describes a squeeze or
	stretch of the spectrum. The wavelength-pixel-mapping as well as the optical
	properties of the spectrometer depend on the instrument’s temperature and the
	ambient pressure in many cases. The temperature dependency is discussed in
	more detail in Chapter 7.
	To determine the wavelength-pixel-mapping, the peaks of a Mercury line
	spectrum are measured in many applications. From the known wavelength of
	the peaks and the pixels that correspond to these peaks the wavelength-pixelmapping
	can be determined. More recently the wavelength-pixel-mapping is
	determined by comparing a high-resolution solar spectrum (Chance and Kurucz,
	2010), which is convolved to the instrument’s resolution with the measured
	spectrum (Lehmann, 2014; van Rozendael, 2013). The latter approach is used
	in this thesis, as Mercury line spectra for the NOVAC instruments were only
	available at room temperature.
	Two other important temperature-dependent effects of the detector are offset
	and dark current. The offset is a low voltage that is added to the CCD signal
	to prevent negative signals that could result from noise at very low intensities.
	The analog-to-digital converter that converts the signal to a digital value cannot
	handle negative values. The offset structure is temperature dependent and each
	instrument has an individual offset structure. It can be removed by subtracting
	an offset spectrum, which can be created by recording n spectra with minimal
	exposure time and no radiation entering the spectrometer. The offset spectrum
	has to be rescaled to match the number of spectra summed up in the measurement
	spectrum.
	The dark current of the CCD is caused by thermally excited electrons in the semiconductor’s depletion zone. The dark current signal increases linearly with
	increasing exposure times and exponentially with increasing temperatures and can
	be greatly reduced by cooling the detector. Additionally, it is signal dependent
	(Stutz and Platt, 1993), but a first-order correction can be made by subtracting
	a dark current spectrum from the offset-corrected measurement spectrum. The
	dark current spectrum should be recorded at the same temperature as the
	measurements, with no radiation entering the spectrometer and an exposure
	time longer than the measurement’s exposure time. It has to be rescaled to
	match the measurements exposure time.
	The complete process of how radiation is measured and the incident spectrum
 is altered by the instrument’s optical system and subsequently mapped
	onto discrete channels recorded by the detector is depicted in Figure 3.2.
	The Ring effect
	Inelastic scattering leads to the Ring effect (named after Grainger and Ring, 1962)
	that can be observed as a filling of the Fraunhofer lines in spectra of scattered
	solar radiation when compared to direct sunlight measurements. Today, it is
	believed that the Ring effect is caused by rotational Raman scattering mainly on
	O2 and N2 in the atmosphere (Bussemer, 1993; Solomon et al., 1987). Solomon
	et al. (1987) suggested treating the Ring effect as a pseudo-absorber in the
	evaluation of spectroscopic data based on the calculation listed below.
	The radiation arriving at the instrument Imeas consists of parts that were
	Rayleigh, Mie or Raman scattered in the atmosphere:
	Imeas = IRayleigh + IMie + IRaman = Ielastic + IRaman (3.21)
	The logarithm of Imeas can be expanded when taking into account that Raman
	scattering is orders of magnitude weaker than elastic scattering processes:


	The Ring spectrum that can be included as a pseudo-absorber in the DOAS
	evaluation is defined as:
	IRing = IRaman
	Ielastic
	(3.23)
	The Raman Spectrum is needed for the calculation of the Ring effect. It can be
	measured by comparing the intensities of light that is polarized perpendicular
	and parallel to the scattering plane (Bussemer, 1993; Solomon et al., 1987). Another
	approach is to calculate the Raman spectrum from a measured spectrum
	using input knowledge on the concentrations of N2 and O2 in the atmosphere (Bussemer, 1993; Chance and Spurr, 1997). In this thesis, the latter approach
	is used with the Ring spectrum being calculated from the DOASIS software
	package (Kraus, 2006).
	In cases with multiple Rayleigh scattering in the atmosphere or scattering on
	clouds and aerosol particles, the wavelength dependency of the amplitude of the
	Ring spectrum can vary (Wagner et al., 2009). For these cases a second Ring
	spectrum, which is calculated from the original Ring spectrum by multiplying it
	by a wavelength dependent term , is included in the evaluation:
	IRing,2 = IRing 
	In this work, the second Ring spectrum is orthogonalised against the first Ring
	spectrum in the retrieval range.
	The solar I0 effect
	As outlined earlier in this section, typical spectrometers have a finite resolution
	that cannot resolve all the structures of the Fraunhofer lines. The absorption
	structures as seen by the spectrometer can therefore differ slightly from the absorption
	cross sections that were recorded in the laboratory with an unstructured
	light source and later convolved to match the instrument’s resolution (Platt
	et al., 1997). Two effects called the solar I0-effect and the saturation effect
	can occur. Both effects are caused by narrow spectral structures that are not
	accurately taken into account when using cross sections as outlined in Eq. 3.18.
	For the saturation effect, these are strong absorption lines that are smoothed out
	during the convolution of the cross section. For the solar I0-effect, the causes
	are narrow Fraunhofer lines that weight a cross section differently in reality.
	Mathematically this can be shown by comparing an ideal measurement (infinite
	resolution) with one trace gas with a real measurement. For an ideal measurement,
	the DOAS approach would yield the optical density:

	In reality, however, both intensities are measured with the instrument’s spectral
	resolution (see Eq. 3.17) and the optical density  is:

	Therefore, for strong absorbers or a structured light source (as the sun in
	passive DOAS measurements), a correction has to be applied. Following Platt
	and Stutz (2008) and Lehmann (2014), the corrected absorption cross sections 3 Remote sensing of volcanic gases
	can be calculated as:


	In this equation 
	0,K is calculated by convolving a high resolution solar spectrum
	(e.g. Chance and Kurucz, 2010):

	For the second spectrum 
	K an absorbing term according to Beer-Lambert law
	is applied before the convolution:

	A good estimate of the true column density is therefore needed to create the
	corrected absorption cross sections. In this work, whenever solar I0-corrected
	cross sections are used, the estimate was obtained by performing a fit using cross
	sections according to Eq. 3.18. The obtained column densities were then used as
	input parameters for Eq. 3.27.
	The DOAS retrieval
	The DOAS approach used in this thesis differs slightly from Eq. 3.14 and 3.15.
	The fitting routine from the DOASIS software (Kraus, 2006) was used to evaluate
	all spectroscopic data in this thesis.
	Eq. 3.4 can be rewritten to:

	differ between the background spectrum I0 and the measurement spectrum I.
	The DOAS fitting routine varies the polynomial and the column densities

	The DOASIS fit routine uses a combination of a standard least-squares fit and a
	Levenberg-Marquard algorithm to minimize 
	\begin{figure}
		\centering
		\includegraphics[width=0.7\linewidth]{Bilder/Simon/Bilder_Tung/DOAS_Intensity}
		\caption{}
		\label{fig:doasintensity}
	\end{figure}
	
	\part{Evaluation of the Data of Tungurahua and Nevado Del Ruiz}
	
	
	\chapter{Network for Observation of Volcanic and Atmospheric Change \label{NOVAC}}
	%	
		\begin{figure}[h]
			\centering
			\includegraphics[width=0.8\linewidth]{Bilder/NOVAC2015}
			\caption{Global map of the volcanoes monitored by NOVAC. Used with friendly permission of Santiago Arellano.}
			\label{fig:novac2015}
		\end{figure}
		Network for Observation of Volcanic and Atmospheric Change (NOVAC) is a network of instruments monitoring volcanoes over the hole world. 
		The aim of NOVAC is to gain another tool for risk assessment, for gas emissions and geophysical researches. Also many other scientific purposes are build on the data from NOVAC.\\
		\Cref{fig:novac2015} shows a map, with all volcanoes of the Network for Obersavation of Volcanic and Atmospheric Change.\\
		%
		NOVAC was originally funded by the European Union on the first October in 2005. The aim of NOVAC is to  establish  a  global  
		network  of  stations  for  the  quantitative  measurement  of  volcanic gas  emissions. At the beginning NOVAC encompassed observatories of 15 volcanoes in Africa America and Europe, including some of the most active and strongest degassing volcanoes in the world. Although the EU-funding has stopped, the network has been constantly growing since it was founded. In 2017 more than 80 Instruments are installed at over 30 volcanoes in more than 13 countries.\\
		The great advantage of the data monitored in NOVAC is the fact
		that NOVAC provides continues gas emission data over many years. Therefore one is able to get more statistical stable results.\\
		The instruments used in NOVAC are scanning UV-spectrometer : Mini Doas instruments. \\
		The  Mini-DOAS  instrument  represents  a  major  breakthrough  in  volcanic  gas	monitoring as it is capable of real-time semi-continuous unattended measurement of the total emission fluxes of  SO2	and BrO from a volcano. Semi-continues means in this case that the measurement is only possible during day time when enough Sun light is there.\\
		%
		\begin{figure}
			\subfigure[Bezeichnung der linken Grafik]{\includegraphics[width=0.49\textwidth]{Bilder/Simon/Bilder_Tung/NOVAC_Instrument}}
			\subfigure[Bezeichnung der rechten Grafik]{\includegraphics[width=0.49\textwidth]{Bilder/Simon/Bilder_Tung/NOVAC_scan_geo}}
			\caption{Titel unterm gesamten Bild}
		\end{figure}
		The  basic  mini-DOAS  system  consists  of  a  pointing  telescope  fiber-coupled  to  a  spectrograph.  
		Ultraviolet light from the sun, scattered from aerosols and molecules in the atmosphere, is collected by 
		means  of  a  telescope  with  a  quartz  lens  defining  a  field-of-view  of  12  mrad.
		\ref{NOVAC Seite} \\
		The spectrometers measure in the UV region in a wavelength range of 280 to 420 nm. In this range are the differential structures of SO2 and BrO dominant.
		\\
 		The Novac-instruments need to be very robust to stand the conditions around volcanoes. Therefore the design of the instruments is rather simple, this means the instruments do not have internal stabilisation features like temperature stabilization to keep the measurement independent of external parameters (for example Temperature).
		This comes along with a reduced precision of the data, but the huge amount of data produced by NOVAC compensates this disadvantage.  
	
	
	\section{Measurement Routine}
	\begin{figure}[h]
		\centering
		\includegraphics[width=0.6\linewidth]{Bilder/Simon/Bilder_Tung/Map_Tungurahua2}
		\caption{}
		\label{fig:maptungurahua2}
	\end{figure}
	The Instruments are set up five to ten km downwind of the volcano of the volcano. To cover most of the occurring wind directions two to five instruments are installed at each volcano. Ideally the measurement plane is orthogonal to the plume, to get the best measurement results. In reality the measurement plane could be twisted.\\
	The Instruments record spectra in different viewing angles covering a the hole sky from horizon to horizon from 
	-90$^{\circ}$ to 90$^{\circ}$. The zenith is at 0$^{\circ}$.
	The measurement routine starts with a spectrum in zenith direction: The pre-reference.
	Afterwards the dark current spectrum is recorded.\\
	Then the Instrument turns automatically to the side, recording spectra at the Elevation Angle from -90$^{\circ}$ to 90$^{\circ}$ with steps of 3.6$^{\circ}$. \\
	One hole measurement takes from 6 to 15 minutes.

	
	\chapter{Evaluation Routine}
	\section{NOVAC-Evaluation}
		\begin{figure}[h!]
		\centering
		\includegraphics[width=0.5\linewidth]{Bilder/Simon/Bilder_Tung/Algorithm}
		\caption{}
		\label{fig:algorithm}
	\end{figure}
	\begin{figure}
		\subfigure[Bezeichnung der linken Grafik]{\includegraphics[width=0.49\textwidth]{Bilder/Simon/Bilder_Tung/SO2_Scan}}
		\subfigure[Bezeichnung der rechten Grafik]{\includegraphics[width=0.49\textwidth]{Bilder/Simon/Bilder_Tung/BrO_Scan}}
		\caption{Titel unterm gesamten Bild}
	\end{figure}
	In the following we describe the technical implementation of the DOAS approach using the data of NOVAC instruments:\\
	The first important task is to locate volcano plume and the reference region using the data from te measurement routine described above.
	
	To do so we use the pre-reference (the spectra recorded at an elevation angle of  0$^{\circ} $) to evaluate spectra for SO2 at every elevation angle as described in chapter \ref{DOAS}, that means we divide each recorded spectra by the pre-reference and take the logarithm  to get rid of the Frauenhofer structures and to be able to just look at the important structures of the plume. To get the gas amounts of the evaluated spectra, one fits the absorption spectrum of all important gases on the spectrum. In our case we take all gases written in tab. \ref{tab: 1} into account. The result will be an SO2 curve as it is shown in fig. \ref{fig so2}.
	Figure \ref{fig so2} shows the relative SO2 column density to the pre-reference as a function of the elevation angle. We can clearly observe a a maximum of SO2 and a minimum. Inside the plume the SO2 amount is much higher than in the outside the plume. Therefore we assume that the location of the SO2 maximum match with the location of the plume. We assume that the minimum of the SO2 curve refers to a region outside of the plume which is in most times the case. The SO2 amount in the earths atmosphere is negligible so we take it as a region of zero SO2. Now it is possible to locate the plume region as the SO2 maximum, whereas the minimum of the SO2 curve the reference region is. \\
	To technically detect the plume region we use a gauss fit of the So2 curve.
	To increase the quality and to get a more robust result the sum over several plume spectra is taken. If the gauss curve is too wide we use only the 10 spectra with the highest SO2 amount. For the reference we use the sum of 10 spectra with the lowest SO2 amount.\\
	%
	The so found reference spectrum is used to fit it on the SO2 absorption lines of Gases to get the absolute column densities of SO2 and BrO in the plume spectrum\\
	\\
	Since the BrO column density is much lower than the SO2 column density and lies just slightly above the detection limit the plume is hard to detect using the BrO column density as it is shown in fig. \ref{ fig bro}. 
	Therefore we use plume location we found by using SO2 to evaluate the BrO column density.\\
	For the evaluation we use the data of more than one measurement, to increase the fit quality.\\
	We are mainly interested in the BrO/SO2 ratio, with the calculations described above it is now possible the get this ratio.
	In \cref{...} is the NOVAC Evaluation visualized.\\
	Taking the BrO/SO2 ratio if the column densities are close to zero yields unpredictable and unrealistic results. Thus spectra measured outside of the volcano plume need to be excluded.
	This could be achieved by setting a BrO or/and a SO2 threshold. A reasonable BrO threshold need to be at least in the order of the DOAS fit error. But this could lead to elevated BrO/SO2 ratios, since the BrO error is often close to the detection limit, and thus exclude all low BrO column densities from the evaluation.
	%
	
	\textcolor{red}{	
	To avoid this problem, an SO2 threshold of $7\cdot 10^{17} \frac{molec}{cm^2}$ was used
	to select spectra for the evaluation of the BrO/SO2 ratio. This threshold is
	a relatively high SO2 column density. However, for the lower values of the
	BrO/SO2 ratio in  this would result in a BrO column
	density as low as $4\cdot 10^{13} \frac{molec}{cm^2}$, a value only slightly higher than the
	average DOAS retrieval error for BrO. This approach assures that scans not seeing
	significant amounts of volcanic gas are filtered out and thus will not significantly
	influence the BrO/SO2 ratio. \citealp{lubcke2014bro}}
	
	\section{Contamination Problem}
	\begin{figure}
		\centering
		\includegraphics[width=0.7\linewidth]{Bilder/contaminated}
		\caption{}
		\label{fig:contaminated}
	\end{figure}
	---------------------- genaue angebane tungurhaua u NEVADO\\
	\\
	It might occur that in rare (ca. 10\% of the data) scenarios, the
	volcanic plume covers the whole scan region.
	This could happen if for example the volcanic plume of the day before still extend over the hole scan area as a consequence of windless conditions.
	In consequence, the reference	is contaminated with volcanic trace gases. Thus the gas amount is underestimated by the NOVAC-Evaluation: In \cref{fig:contaminated} we see an example from April 2011 (Tungurahua) where the reference region is contaminated by volcanic trace gases. The blue SO2 curve shows our calculations with the NOVAC-Evaluation, but since there is still SO2 in the reference region, therefore the assumption, that the SO2 amount could be set to zero in the reference region is wrong. The red curve shows the real SO2 curve, and we will underestimate the total SO2 amount of the plume. Contamination occur in approximately 10$\%$ of the data.\\
	\\
	If the reference region is for any reason
	contaminated by volcanic trace gases, the reference spectrum has to be
	replaced by a volcanic-gas-free reference. Alternative spectra are a
	theoretical solar atlas spectrum (the use of a solar atlas spectrum will be described in \cref{kuruz}) ore a a volcanic-gas-free reference
	spectrum recorded by the same instrument.\\ 
	%
	\\
	%
	In the following we will discuss both of these options:
	%
	\subsection*{Evaluation using a Solar Atlas Spectrum \label{kuruz}}
	An alternative to choose the region with the lowest column density as reference region is to use a theoretical high resolution solar atlas spectrum as reference \cite{chance2010improved}.
	The use of a theoretical solar atlas spectrum as a reference which is completely volcanic-trace-gases-free was first proposed by \cite{lubcke2014bro}.
	The advantage of using a solar atlas spectrum as reference is, that we know that there are no volcanic trace gases, we do not need to assume, that the minimum SO2 amount is zero. The disadvantage is, that using a solar atlas spectrum comes along with a drawback of precision: A theoretical solar atlas spectrum is far more precise than the spectra of the NOVAC instruments therefore the instrument functions need to be modeled and added to the retrieval.\\ 
	The reduction of precision is acceptable for the
	SO2 retrieval but not suitable for a BrO retrieval because then most data would be below the detection limit.\\
%
\\
%
	Possible contaminations can be checked
	by a theoretical solar atlas spectrum to evaluate the SO2 amount in the reference.

	\subsection*{Evaluation using a Spectrum of the same Instrument}
	An alternative reference spectrum coud be a a volcanic-gas-free reference
	spectrum recorded by the same instrument. When using such a reference several problems occur:\\
	As described in \cref{NOVAC} the instruments used in NOVAC do not include features like temperature stabilisation due to that the measurements are not independent from external parameters. 
	So we need to choose a reference recorded at similar conditions with respect to meteorology and	radiation as well as in the temporal proximity due to instrumental changes with time and ambient conditions. Ideally the external conditions should be equal to the conditions when the plume was recorded.\\
	\\
	%
	\\
	In this work we will combine both options in order to
	achieve both, enhanced accuracy but still maximum possible precision of
	the SO2 and BrO retrievals. So we use the solar atlas spectrum to check for 
	contamination and a reference spectrum recorded in temporal proximity by the same instrument as reference.\\
	\\
	%
	\\
	In the following we will discuss how to find the an optimal reference from another scan automatically.
	
	\begin{figure}
		\centering
		\includegraphics[width=0.7\linewidth]{E:/Masterarbeit/Analyse_Contamination/contaminationdependency_so2}
		\caption{}
		\label{fig:contaminationdependencyso2}
	\end{figure}
	\begin{figure}
		\centering
		\includegraphics[width=0.7\linewidth]{E:/Masterarbeit/Analyse_Contamination/contaminationdependency_bro}
		\caption{}
		\label{fig:contaminationdependencybro}
	\end{figure}
	\begin{figure}
		\centering
		\includegraphics[width=0.7\linewidth]{E:/Masterarbeit/Analyse_Contamination/contaminationdependency_ratio}
		\caption{}
		\label{fig:contaminationdependencyratio}
	\end{figure}
	\chapter{Limitations for the evaluation of  BrO}
    Since the SO2 amount in a volcano plume is rather high (magnitude of SO2 at Tungurahua $\approx 1e^{18}$, \cite{WarnachSimon}) , the evaluation of SO2 is unproblematic compared to BrO.\\
	Evaluating BrO is more difficult since the amount is much smaller and the measurement error relative to the column density much larger. Since we want to get the BrO/SO2 we need to maximize the accuracy of BrO.
	Therefore the aim is to choose the reference with respect to the BrO error, to minimize the BrO Error and to increase the amount of reliable BrO/SO2 ratio data.\\
	We figured out, that the BrO Error depends strongly on the surrounding conditions when recording the plume and the reference. In the following, we will take a closer look at the dependence of the BrO error on external parameters. 
	%
	\section{BrO Error dependence on external parameters}
	%HOW IS THE ERROR CALCULATED AND WHY AND DETECTION LIMIT (PLATT \& STUTZ)
	The measurement and evaluation depends on the surrounding conditions like temperature or cloudiness \cite{lubcke2014optical}\\
	If choosing a new reference we need to take the surrounding conditions into account\\
	The better the surrounding conditions of the time where the reference is measured coincide with the conditions of the time when the plume is measured, the lower is the BrO error \\
	The surrounding conditions we take into account are temperature, colorindex, exposure time, elevation-angle, daytime and the temporal difference.\\
	In almost all cases (99\%) the absolute BrO Error is minimal when using the reference recorded at the same time as te plume spectrum. So we won't be able to get an BrO Error which is smaller than the "Same Time Error".   

	
	\subsection{Time}
	Due to instrument drifts the fit quality decreases with the time difference between recording the plume and the reference. Therefore it is better to use an reference in temporal proximity.\\
	%
	\begin{figure}[h]
		\centering
		\includegraphics[width=0.7\linewidth]{Bilder/Simon/Bilder_Tung/Drift_Komplett_NEW}
		\caption{}
		\label{fig:driftkomplettnew}
	\end{figure}
	%
	\Cref{fig:driftkomplettnew} shows the Instrumental drift as a function of time, to create \cref{fig:driftkomplettnew} we used Tungurahua data, 2008 from June to November. We can observe that the drift changes with time. If we use the reference and plume spectra of the same time, we do not need to care about these effects, since the shift is equal for the plume and reference spectrum, but if the recording time is not the same the quality of the fit changes with the differences in wavelength shift which increases with the time difference.\\
	In \cref{fig:dat} the BrO Error as a function of the time difference between recording the plume and the reference is shown. The running mean is drawn with a black line. The BrO Error increases with time difference.\\
	To evaluate the maximal time difference, were we still get reliable results we calculated for all possible reference-plume pairs the corresponding BrO Error. With this data we are able to find for all plume spectra the associated reference where the BrO Error is minimal. In \cref{Histogram} a histogram is plotted with the probability of picking the best reference as a function of the time difference. Obviously the best results are if the day of measuring the reference is the same day as measuring the reference that means, if the time difference is smaller than one day. We allow all time difference which are in one sigma area. 
	
	We found out that the time interval where it is still reasonable to use references is about 14 days. Therefore we only use references where the recording time difference between plume and reference is smaller than two weeks. When using a references with a temporal difference to the plume of more than 14 days the probability, that the fit quality and thus the BrO error increases to much for our purposes. \\
	\\
	\begin{figure}
		\centering
		\includegraphics[width=0.7\linewidth]{Bilder/Datum_100h}
		\caption{}
		\label{fig:datum100h}
	\end{figure}	
	\begin{figure}[h!]
		\includegraphics[width=1.1\linewidth]{Bilder/Datum}
		\caption{}
		\label{fig:dat}
	\end{figure}
	%
	\begin{figure}[h!]		
		\subfigure[Data of Tungurahua]{\includegraphics[width=0.49\textwidth]{Bilder/Simon/Bilder_Tung/D2J2140_Before}}
		\subfigure[Data of Nevado Del Riz]{\includegraphics[width=0.49\textwidth]{Bilder/Simon/Bilder_Tung/D2J2140_After}}
		\caption{Titel unterm gesamten Bild}
		\label{fig:shorttermshift}
	\end{figure}
	\subsection{Temperature}
	The Instrument design of the NOVAC instruments compromise between accuracy and longetivity as explained in \cref{NOVAC}. In particular there are no internal thermal stabilizations installed as an attempt to reduce the need for power. This can influence the recorded spectra.\\	
	Each pixel of the spectrometer, which is used for the DOAS experiment, collects photons of a certain wavelength range.\\
	The calibration for the wavelength to pixel mapping (WMP) is commonly done with a Mercury lamp or by the comparison with the high defined Kuruz spectrum.
	As the WMP depends on the optical alignment of the spectrometer, which itself depends on the temperature, it is not constant.
	Changes in the spectrometers temperature can cause changes in the instrument line function and shifts in the WMP (\cite{pinardi2007influence}). 
	Moreover, \cite{WarnachSimon} show that, short term shifts are related to the instrument temperature (see \Cref{fig:shorttermshift}).\\
	The above discussed temperature dependence of the WMP causes a reduction of the fitquality with increasing instrument temperature between plume and reference. Thus the BrO Error increases as well with the temperature difference. To quantify the BrO error dependency on the temperature all plume spectra of Tungurahua from August 2008 to August 2009 (Nevado del Ruiz from .... to ....) where evaluated with all plume spectra of the same time. In this time span 1647 "multi-add" spectra from tree different instruments where recorded, so we get approximately 1646$^2$ plume reference pairs and their corresponding BrO error and temperature. The BrO error as function of the temperature difference can be seen in \cref{fig:difftemp}. The blue dots shows the mean BrO error at the specific temperature difference, the standard deviation is illustrated with gray bars.\\
	When compare the data of Tungurahua  and Nevado Del Ruiz it is noteworthy that the BrO error on temperature dependence of the Data of Nevado Del Ruiz is stronger and the deviation is weaker than at Tungurahua, this may occur due to the larger temperature fluctuation at Nevado Del Ruiz. ?????????? quellen bitte!\\
	When looking at all discussed external parameters, temperature has  the strongest impact on the BrO error due to the strong impact on the WMP.
	\begin{figure}[h!]			
		\subfigure[Data of Tungurahua]{\includegraphics[width=0.49\textwidth]{Bilder/DiffTemp_Tungu}}
		\subfigure[Data of Nevado Del Riz]{\includegraphics[width=0.49\textwidth]{Bilder/DiffTemp_Nevad}}
		\caption{Titel unterm gesamten Bild}
		\label{fig:difftemp}
	\end{figure}[h!]
	\begin{itemize}
		\item The BrO error has the strongest dependence on the temperature difference. At Tungurahua (Nevado Del Ruiz) the BrO error increases by factor of $3.53\cdot10^{12}$  per degree.
		\begin{align*}
		\rightarrow&  BrO_{Error} = f(ext. P)+ 3.53\cdot10^{12}\cdot\frac{\Delta T}{1C^{\circ}} + \mathcal{O}\left(\right) & Tungurahua\\
		\rightarrow&  BrO_{Error} = f(ext. P)+7.56\cdot10^{12}\cdot\frac{\Delta T}{1C^{\circ}} + \mathcal{O}\left(\right) & Nevado Del Ruiz\\
		\end{align*}
	\end{itemize}
	\subsection{Daytime}
	During the day o al lot of external parameters like temperature, solar altitude etc. change. In particular the solar altitude could have an impact on the fit quality since the light path of the sun is much longer at the evening than at noon. \Cref{fig:diffdaytime} shows the dependency of the BrO error on the daytime. The data are calculated as described for the temperature. As for the temperature the dependence of Nevado Del Ruiz is much larger than of Tungurahua, this might occur during the larger distance from the equator of Nevado Del Ruiz -> besser beschreiben
	\begin{figure}[h!]			
		\subfigure[Data of Tungurahua]{\includegraphics[width=0.49\textwidth]{Bilder/Diffdaytime_Tungu}}
		\subfigure[Data of Nevado Del Riz]{\includegraphics[width=0.49\textwidth]{Bilder/Diffdaytime_Nevad}}
		\caption{Titel unterm gesamten Bild}
		\label{fig:diffdaytime}
	\end{figure}
	\begin{itemize}
		\item We found a dependency of the BrO error on the daytime. We assume, that this dependency comes from other external parameters which change during the day. 
		\item The BrO Error increases with the daytime differences like: \\
		\begin{align*}
		\rightarrow&  BrO_{Error} = f(ext. P)+1.33\cdot10^{12}\cdot\frac{\Delta DT}{1h}  + \mathcal{O}\left(\right)& Tungurahua\\
		\rightarrow&  BrO_{Error} = f(ext. P)+1.58\cdot10^{13}\cdot\frac{\Delta DT}{1h} + \mathcal{O}\left(\right) & Nevado Del Ruiz\\
		\end{align*}
		
	\end{itemize}
	\subsection{Colorindex}
	\begin{figure}[h]		
		\subfigure[Data of Tungurahua]{\includegraphics[width=0.49\textwidth]{Bilder/DiffColidx_Tungu}}
		\subfigure[Data of Nevado Del Riz]{\includegraphics[width=0.49\textwidth]{Bilder/DiffColidx_Nevad}}
		\caption{Titel unterm gesamten Bild}
		\label{fig:diffcolidx}
	\end{figure}
	Clouds  have  a  strong  influence  on  the  atmospheric  radiative  transfer  and  thus  affect  the  interpretation  and  analysis of DOAS - observations \cite{wagner2014cloud}.
	
	Clouds can be identified by several measurement quantities that they influence.
	As Mie scattering is dominant in clouds the wavelength of the light that is scattered is different than the Rayleigh sky. Thus, clouds can be easily identified by their white color.
	Therefore, the cloudiness of the sky can be quantified in a scalar measure defined by the ratio of the measured intensity at two wavelengths, the so-called colour index.
	\cite{wagner2014cloud} showed that for a zenith-looking instrument the measured radiation intensity is enhanced by clouds. Thus, clouds can cause large errors for the retrieved gas column density and the corresponding uncertainties. 
	Cloud effects are especially severe if the cloudiness for the recorded plume and reference spectra strongly defer. Also for broken clouds the described effect can be observed as measurements at some elevation angles might be influenced by clouds while others are not.
	In this work the Colour Index (CI) is the ratio between the intensities at 320nm and 360 nm.
	These two wavelengths are as far apart as the filter used for stray-light prevention in the spectrometers allows.
	%% I don’t understand 	
	On the other hand, the lower wavelength avoids the deep UV range where SO2 and O3 absorption plays a dominant role.
	%% I don’t understand 	
	The Mie scattering in the clouds is responsible for the higher amount of radiation from larger wavelengths. This results in a decrease of the CI (\cite{lubcke2014optical}).
	\\
	We evaluated the CI at the zenith, to increase the stability of the fit we added in each cases 10 intensitys. Using always the zenith to evaluate the colour index makes the colour index more comparable, but if broken clouds occur, the CI of the reference and the plume could differ from the calculated CI of the zenith. This could be a reason for the large deviations of the mean BrO error as function of the colour index (see \cref{fig:diffcolidx})

	\begin{itemize}
		\item 	The BrO Error increases with the Colorindex differences as \\
		\begin{align*}
		\rightarrow&  BrO_{Error} = f(ext. P)+ 1.01\cdot10^{13}\cdot\frac{\Delta Cidx}{0.1} + \mathcal{O}\left(\right) & Tungurahua\\
		\rightarrow&  BrO_{Error} = f(ext. P)+  4\cdot10^{13}\cdot\frac{\Delta Cidx}{0.1} + \mathcal{O}\left(\right) & Nevado Del Ruiz\\
		\end{align*}
	\end{itemize}
	\subsection{Elevation Angle}
		\begin{figure}[h!]			
		\subfigure[Data of Tungurahua]{\includegraphics[width=0.49\textwidth]{Bilder/DiffElevAngle_Tungu}}
		\subfigure[Data of Nevado Del Riz]{\includegraphics[width=0.49\textwidth]{Bilder/DiffElevAngle_Nevad}}
		\caption{Titel unterm gesamten Bild}
	\end{figure}
	The elevation angle describes the angle between the horizon and the zenith. When using the plume spectrum and the reference spectrum of the same time, the difference in elevation angle cannot be zero, since the plume is always somewhere else than the reference located.\\
	The BrO error doesn't depend significantly on the difference between the Elevation Angles. This could have several reasons. One problem is, that the Elevation Angle of Plume and Reference spectrum is not the same. This could also be a reason of uncertainty of the evaluations of the plume spectrum.




	\subsection{Exposure Time}
	\begin{figure}		
		\subfigure[Data of Tungurahua]{\includegraphics[width=0.49\textwidth]{Bilder/DiffExpTime_Tungu}}
		\subfigure[Data of Nevado Del Riz]{\includegraphics[width=0.49\textwidth]{Bilder/DiffExpTime_Nevad}}
		\caption{Titel unterm gesamten Bild}
		\label{fig:diffexptime}
	\end{figure}
	The Exposure Time is a degree of sky lightness.The  exposure time is the length of time the sensor of the NOVAC instrument is exposed to light. The amount of light that reaches the film or image sensor is proportional to the exposure time. The exposure time is adjusted in the way that the maximum intensity does not overly the capacity of the sensor.\\
	We can observe an small dependency of the BrO error on the Exposure time at Tungurahua and Nevado Del Ruiz as it is shown in \cref{fig:diffexptime}
	

	\begin{itemize}
		\item The BrO Error increases with the exposure time differences as\\
		\begin{align*}
		\rightarrow&  BrO_{Error} = f(ext. P)+ 1.92\cdot10^{12}\cdot\frac{\Delta ET}{10^{-2}s} + \mathcal{O}\left(\right) & Tungurahua\\
		\rightarrow&  BrO_{Error} = f(ext. P)+ 1.0\cdot10^{13}\cdot\frac{\Delta T}{10^{-2}s} + \mathcal{O}\left(\right) & Nevado Del Ruiz\\
		\end{align*}
	\end{itemize}
	%--------------------------------------------------------------------------------------------------------------
	\chapter{Method}
	Based on the findings about the influence of external parameters on the BrO error we developed an algorithm which is able to pick an appropriate volcanic-trace-gas free reference.\\ 
	The first step is, to evaluate every reference with solar atlas spectrum, to check for contamination.	A Spectrum is treated as contaminated if the SO2 column density of the reference (evaluated with a solar atlas spectrum) is larger as 2$\cdot 10^{17}\frac{molec}{cm^2}$.\\
	\\
	If the reference is contaminated:
	\begin{itemize}
		\item We have a list of possible references where all references are not contaminated and the temporal distance to the plume date is no longer than 14 days.
		\item we calculate of all possible references the differences in the external parameters
		\item We use the analysis of external parameters described above to estimate the BrO error of all references
		\item We choose the reference with the smallest estimated BrO error as new reference
		\item We evaluate the plume spectra with the new reference.
	\end{itemize}

	%
	The assumption is, that the BrO error $\epsilon_{BrO}$ can be described as the sum of $\epsilon_{0}$ and the deviation of $\epsilon_{BrO}$ with respect to all external parameters. $\epsilon_{0}$ is the BrO error when evalute the plume spectrum with the "same-time-reference", it is determined due to the accurateness of the NOVAC-instruments.
	\begin{align}
		\epsilon_{BrO} &=  \epsilon_{0}+\frac{d\epsilon}{dt}+\frac{d\epsilon}{d ^{\circ}}+\frac{d\epsilon}{dT}+\frac{d\epsilon}{ddt} +\frac{d\epsilon}{dc} + \mathcal{O}\left(OE\right) \\
		\rightarrow \Delta \epsilon_{BrO} &= \epsilon_{BrO} - \epsilon_{0} =\frac{d\epsilon}{dt}+\frac{d\epsilon}{d ^{\circ}}+\frac{d\epsilon}{dT}+\frac{d\epsilon}{ddt} +\frac{d\epsilon}{dc} + \mathcal{O}\left(OP\right) 
		\label{calc:err}
	\end{align}
	With $\epsilon_{BrO}$ describes the BrO Error, t: time between plumetime and referencetime, T, temperaure; dt: daytime, c: colorindex, OP: other excluded external parameters\\
	The task occurring at this stage is to find the best representation for the deviations. An then find the reference which minimize $\Delta \epsilon_{BrO} $\\
	%
	\\
	%
	The easiest way is to just calculate the BrO error of all possible references for every plume. Using this method we would be able to just choose the reference where the BrO error is minimal. But this takes to much time since the evaluation would be proportional to the number of possible references because the evaluation need to be done for every plume-reference pair. Doing the evaluation for every plume-reference pair would make it impossible to do the evaluation in real, or near real time.\\
	%
	But we use this optimal evaluation to rate our model and compare them among each other. The optimal evaluation always choose the reference with the smallest absolute error. We don't use the relative error due to his vulnerability. Using the relative error could lead to a less preciseness.\\
	%
	\\
	Hier ein Bild, das eine Plume gegen viele referencen auswertet und hier die Abweichungen zeigt\\
	\\
	%
	The results of the algorithm which chooses the reference automatically are described relative to an optimal evaluation. If the relative error is larger than 5 we don't use the data.\\
	\\
	We tried several methods for choosing the best reference based on the analysis of external parameters. Fitting the data with a first order polynomial brought the best results.
	
	
	\section{Fit data}
	When looking at the analysis of the external parameters (see fig. \ref{fig:difftemp}-\ref{fig:diffexptime}) we can observe curves which are symmetric to the zero position. Therefore we conclude that it makes no difference whether the difference of a specific external parameter is positiv or negativ. Thus we take the absolute values for our calculations.\\
	\\
	%
	If we assume that all differentiations are linear, than we can write \cref{calc:err}
	as:  
	
	\begin{equation}
		\Delta \epsilon_{BrO} = a_{t}\cdot\Delta t+a_{^{\circ}}\cdot\Delta ^{\circ}+a_{T}\cdot\Delta T+a_{dt}\cdot\Delta dt +a_{c}\cdot\Delta c + \mathcal{O}\left(OP\right)
		\label{calc:delterr}
	\end{equation}
	%
	We used the same data as in \cref{fig:difftemp}-\ref{fig:diffexptime} to get the coefficients $a_{x}$ of \cref{calc:delterr}. We used on ordinary least square linear regression to get the coefficients $a_{x}$. In particular we used the python function LinearRegression from the library sklearn \cite{SKlearn}.  

	
	

	

		
		The constants for Tungurahua and Nevado Del Ruiz are:
		\begin{table}
			\subfigure[Data of Nevado Del Riz D2J2201\_0]{
			\begin{tabular}{c|c|c}
				\toprule
				Constant &value & importance\\
				\toprule
				$a_{T}$&  7.338e+12&0.840\\
				\midrule
				$a_{ET}$&1.545e+10&0.045\\
				\midrule
				$a_{t}$&-2.6e+09& 0.0\\
				\midrule
				$a_{dt}$&1.805e+12&0.091\\
				\midrule
				$a_{c}$&2.301e+13& 0.031\\
				\bottomrule
			\end{tabular}}
			%	\caption{Data from Nevado Del Ruiz from the D2J2201\_0 instrument. All external parameter where taken into account. $\epsilon_{0} = =  5.404e+12$}
			\qquad
			\subfigure[Data of Nevado Del Riz D2J2200\_0]{
			\begin{tabular}{c|c|c}
				\toprule
				Constant &value & importance \\
				\toprule
				$a_{T}$& 1.162e+13&0.908 \\
				\midrule
				$a_{ET}$&2.811e+10& 0.046\\
				\midrule
				$a_{t}$&-1.7e+09& 0.0\\
				\midrule
				$a_{dt}$&1.076e+12&0.034\\
				\midrule
				$a_{c}$& 3.587e+13& 0.016\\
				\bottomrule
			\end{tabular}}
					\subfigure[Data of Nevado Del Both Instruments]{
			\begin{tabular}{c|c|c}
				\toprule
				Constant &value & importance \\
				\toprule
				$a_{T}$& 1.073e+13&0.973 \\
				\midrule
				$a_{ET}$&3.478e+10&  0.070\\
				\midrule
				$a_{t}$&-9.1e+08& 0.0\\
				\midrule
				$a_{dt}$& 1.523e+11&0.006\\
				\midrule
				$a_{c}$& -6.811e+13& -0.047\\
				\bottomrule
		\end{tabular}}
			\caption{(a)Data from Nevado Del Ruiz from the D2J2201\_0 instrument. All external parameter where taken into account. $\epsilon_{0} = =  5.404e+12$
				(b)Data from Nevado Del Ruiz from the D2J2200\_0 instrument. All external parameter where taken into account. $\epsilon_{0} = =  1.105e+13$ (c) Data from Nevado Del Ruiz from both instrument. All external parameter where taken into account. $\epsilon_{0} = 1.260e+13$}
		\end{table}	


	\begin{itemize}
	
		\item Plume data are reliable if the SO2 column density is larger as 7 $\cdot 10^{17} \frac{molec}{cm^2}$
		\item Data are above the detection limit if the column density as two times larger than the fit error.
		\item If the reference is contaminated:

	\end{itemize}
	\begin{figure}
		\centering
		\includegraphics[width=0.7\linewidth]{E:/Masterarbeit/Analyse_minSO2/percentage_minSO2}
		\caption{}
		\label{fig:percentageminso2}
	\end{figure}
	\begin{figure}
		\centering
		\includegraphics[width=0.7\linewidth]{E:/Masterarbeit/Analyse_minSO2/data_minSO2}
		\caption{}
		\label{fig:dataminso2}
	\end{figure}
			\begin{table}[h]
		%		\begin{tabular}{|p{2cm}|p{2.5cm}|p{1.5cm}|p{1.5cm}||p{1cm}|}
		%		%	\toprule
		%			&& Error & Amount of Data&davon gültig\\
		%			\toprule
		%			\multirow{2}{5em}{All Variables} 
		%				& independent& 1.51 & 95$\%$&10,5$\%$\\
		%				& dependent & 1.40&176 &14\\
		%			\midrule
		%			\multirow{2}{5em}{Exposure Time}
		%					&  independent &1.47&174&18\\
		%					& All &1.39&176&13\\
		%			\midrule
		%			\multirow{2}{5em}{Exp.Time u Coloridx}
		%					&  independent &1.4&175&20\\
		%					& All &1.35&176&13\\
		%			\bottomrule
		%		\end{tabular}
		\begin{tabular}{|p{2cm}|p{2.5cm}|p{1.5cm}|p{1.5cm}||p{1cm}|}
			%	\toprule
			&& Error & Amount of Data&valid data\\
			\toprule
			\multirow{2}{5em}{All Variables} 
			& independent& 1.51 & 95$\%$&10,5$\%$\\
			& dependent & 1.40&98$\%$ &8$\%$\\
			\midrule
			\multirow{2}{5em}{Exposure Time}
			&  independent &1.47&97$\%$&10$\%$\\
			& All &1.39&98$\%$&7$\%$\\
			\midrule
			\multirow{2}{5em}{Exp.Time u Coloridx}
			&  independent &1.40& 98$\%$&11\\
			& All & 1.35& 98$\%$&7$\%$\\
			\bottomrule
		\end{tabular}
	\end{table}


	\begin{table}[h]
		\centering
		{Evaluation of all contaminated data from Nevado Del Ruiz}
		\vspace*{1cm}
		\begin{tabular}{|p{2.2cm}|p{4.4cm}|p{5.0cm}|}
			\toprule
			Instrument& Dev from opt eval. (mean/median) & valid data\\
			\toprule
			$D2J2201\_0$ &1.14 / 0.82293& 128/283 = 45.2\%\\
			\midrule
			$D2J2200\_0$ &1.5 / 0.89965&954/1109 = 86.0\%\\
			\midrule 
			Both &1.26/0.847&1073/1392 = 77.1\%\\
			\bottomrule
		\end{tabular}
	\caption{Data from Nevado Del Ruiz from the D2J2201\_0 instrument. All external parameter where taken into account. $\epsilon_{0} = =  5.404e+12$}
	\end{table}
	\section{Other approaches}

	$\bullet$ In the optimal results are 15$\%$ valid data
	\begin{itemize}
		\item We also tried other possibilities than fitting to find the reference where the BrO error is minimal. In the following we present two additional possibilities but compared to fitting the results are not as good.
	\end{itemize}
	\subsection{Nearest neighbours}
	\begin{itemize}
		\item Description of the Nearest Neighbours Method
	\end{itemize}
		\textcolor{red}{Nearest neighbor search (NNS), as a form of proximity search, is the optimization problem of finding the point in a given set that is closest (or most similar) to a given point. Closeness is typically expressed in terms of a dissimilarity function: the less similar the objects, the larger the function values. Formally, the nearest-neighbor (NN) search problem is defined as follows: given a set S of points in a space M and a query point q$\in$ M, find the closest point in S to q. Donald Knuth in vol. 3 of The Art of Computer Programming (1973) called it the post-office problem, referring to an application of assigning to a residence the nearest post office. A direct generalization of this problem is a k-NN search, where we need to find the k closest points.	
		Most commonly M is a metric space and dissimilarity is expressed as a distance metric, which is symmetric and satisfies the triangle inequality. Even more common, M is taken to be the d-dimensional vector space where dissimilarity is measured using the Euclidean distance, Manhattan distance or other distance metric. However, the dissimilarity function can be arbitrary. One example are asymmetric Bregman divergences, for which the triangle inequality does not hold.}
	\subsection{Iterative}
	\begin{itemize}
		\item Description of the iterative Method
	\end{itemize}
	The idea of the iterative method was, that the importance of the individual external parameters are very different, that means if we have the list of possible references, we took all referenes where the temperature difference is minimal, so we get a new, much smaller list of possible referenecs. From this list we choose all references where the next external parameter for example the daytime is minimal and get again a new list. We proceed this way with the following external parameters. We experiment with the sequence of the parameters, to increase the success of the method. The final sequence was:
	\begin{equation*}
	Temperature \bullet ................
	\end{equation*} 
	
	\chapter{Comparison with NOVAC Evaluation}
	We want to 
	\begin{itemize}
		\item Results only for contaminated data
		\begin{itemize}
			\item Difference in SO2 data evaluated with NOVAC-method and contamination-based evaluation
			\item Difference in BrO data evaluated with NOVAC-method and contamination-based evaluation
			\item Difference in BrO/So2 Ratio data evaluated with NOVAC-method and contamination-based evaluation
		\end{itemize}
		\item Amount of BrO data more than before (valid and not valid and above dection limit)
		\item Amount of SO2 data more than before (valid and not valid and above dection limit)
		\item Amount of BrO/SO2 data more than before (valid and not valid and above dection limit)
		\item More BrO data: 51\%
		\item  More valid BrO data: 38\%
		\item Compare the dayly means: how many more data? due to higher S02 values
		\item 
	\end{itemize}

	\begin{figure}[h]		
		\subfigure[Data of Tungurahua]{\includegraphics[width=0.49\textwidth]{Bilder/Tungurahua_Pic/tung_bro_diff_novac_conbased}}
		\subfigure[Data of Nevado Del Riz]{\includegraphics[width=0.49\textwidth]{Bilder/NevadoDelRuiz_Pic/bro_diff_novac_conbased}}
		\subfigure[Data of Tungurahua]{\includegraphics[width=0.49\textwidth]{Bilder/Tungurahua_Pic/tung_so2_diff_novac_conbased}}
		\subfigure[Data of Nevado Del Riz]{\includegraphics[width=0.49\textwidth]{Bilder/NevadoDelRuiz_Pic/so2_diff_novac_conbased}}
		\subfigure[Data of Tungurahua]{\includegraphics[width=0.49\textwidth]{Bilder/Tungurahua_Pic/tung_ratio_diff_novac_conbased}}
		\subfigure[Data of Nevado Del Riz]{\includegraphics[width=0.49\textwidth]{Bilder/NevadoDelRuiz_Pic/ratio_diff_novac_conbased}}
	\caption{The dependency of the Difference between contamination based data and NOVAC to the data evaluated with the NOVAC data }
	\label{fig:diffcontbase}
	\end{figure}
	%
	\begin{figure}[h]		
		\subfigure[Data of Tungurahua]{\includegraphics[width=0.49\textwidth]{Bilder/Tungurahua_Pic/tung_bro_novac_conbased}}
		\subfigure[Data of Nevado Del Riz]{\includegraphics[width=0.49\textwidth]{Bilder/NevadoDelRuiz_Pic/bro_novac_conbased}}
		\subfigure[Data of Tungurahua]{\includegraphics[width=0.49\textwidth]{Bilder/Tungurahua_Pic/tung_so2_novac_conbased}}
		\subfigure[Data of Nevado Del Riz]{\includegraphics[width=0.49\textwidth]{Bilder/NevadoDelRuiz_Pic/so2_novac_conbased}}
		\subfigure[Data of Tungurahua]{\includegraphics[width=0.49\textwidth]{Bilder/Tungurahua_Pic/tung_ratio_novac_conbased}}
		\subfigure[Data of Nevado Del Riz]{\includegraphics[width=0.49\textwidth]{Bilder/NevadoDelRuiz_Pic/ratio_novac_conbased}}
		\caption{The dependency of the Difference between contamination based data and NOVAC to the data evaluated with the NOVAC data }
		\label{fig:diffNovac}
	\end{figure}
\begin{figure}
	\centering
	\includegraphics[width=0.7\linewidth]{Bilder/BarPlot}
	\caption{}
	\label{fig:barplot}
\end{figure}

	\chapter{Results}
	Interpretation of the BrO/SO2 ratio time-series
	\section{Tungurahua}
	\begin{figure}
		\centering
		\includegraphics[width=0.7\linewidth]{Bilder/Results/Results_Tungurahua}
		\caption{}
		\label{fig:resultstungurahua}
	\end{figure}
	
	\begin{small}	
	\begin{align*}
	&Menge an Daten insgesamt: &5883 &\equiv &1\\
	&\hspace*{1cm} Davon: (NOVAC Auswertung) über plume limit&712 &\equiv & 0.121\\
	&\hspace*{1cm}Davon: Menge an Daten, die nicht Kontaminiert sind: &5504 &\equiv & 0.936\\
	&\hspace*{2cm}Davon im Plume-limit:  &599   &\equiv & 0.102\\
	&\hspace*{2cm}Davon über dem Detection Limit:&36  &\equiv & 0.006\\
	&\hspace*{1cm}Davon sind kontaminiert:  &379  &\equiv & 0.064\\
	&\hspace*{2cm}Davon (mit NOVac ausgewertet) über plume limit: &114  &\equiv &0.301 \\
	&\hspace*{2cm}Davon (Neue Auswertung) über plume limit &185  &\equiv &0.488\\
	\end{align*}	
	Dh in den kontaminierten daten sind mit NOVAC ausgewerteten daten \underline{2.485} häufiger über dem plume limit\\	

	\end{small}
	\section{Nevado Del Ruiz}
		\begin{figure}
		\centering
		\includegraphics[width=0.7\linewidth]{"Bilder/Results/Results_NevadoDelRuiz (1)"}
		\caption{}
		\label{fig:resultsnevadodelruiz-1}
	\end{figure}
	\begin{small}	
	\begin{align*}
		&Menge an Daten insgesamt:  &8962 &\equiv &1\\
		&\hspace*{1cm} Davon: (NOVAC Auswertung) über plume limit&142  &\equiv &0.016\\
		&\hspace*{1cm} Davon: nicht kontaminierte daten: &8596 &\equiv &0.959\\
		&\hspace*{2cm} Davon im Plume-limit:   &123  &\equiv &0.014\\
		&\hspace*{2cm} Davon über dem Detection Limit: &53   &\equiv &0.006\\
		&\hspace*{1cm} Davon sind kontaminiert:  &366  &\equiv &0.041\\
		&\hspace*{2cm} Davon (mit NOVac ausgewertet) über plume limit: &20   &\equiv &0.055\\
		&\hspace*{2cm} Davon  (Neue Auswertung) über plume limit&179 &\equiv &0.489\\
		\end{align*}
	Dh in den kontaminierten daten sind mit NOVAC ausgewerteten daten \underline{3.449} häufiger über dem plume limit\\
 	
 \end{small}
	\chapter{Issues of our method}
	
	\section{Contamination of the plume}
	\begin{itemize}
		\item As discussed above it might occur, that, that the reference is contaminated for example by the plume of the day before. If that happens, we underestimate the gas amount by using a contaminated reference. But another possibility is, that the plume is also contaminated. This might be the case if the volcanic gas of the volcano is not taken away by the wind, but accumulates in the plume. If this is the case, using an other reference would lead to an overestimation of the column density of gases.
	\end{itemize}

	\chapter{Conclusion}
	....
	
	
	



  \part{Appendix}
  \begin{appendix}
    \chapter{Lists}
    \listoffigures
    \listoftables
    \bibliography{references}{}
    \citestyle{egu}
    \bibliographystyle{plainnat}
    \include{deposition}
  \end{appendix}
\end{document}




