%% Template for Master thesis
%% ===========================
%%
%% You need at least KomaScript v3.0.0,
%% e.g. available in Texlive 2009
\documentclass  [
  paper    = a4,
  BCOR     = 10mm,
  twoside,
  fontsize = 12pt,
  fleqn,
  toc      = bibnumbered,
  toc      = listofnumbered,
  numbers  = noendperiod,
  headings = normal,
  listof   = leveldown,
  version  = 3.03
]                                       {scrreprt}

% used pagages
%\usepackage[margin=10pt,font=small,labelfont=bf,labelsep=endash]{caption}
\usepackage     [utf8]          {inputenc}
\usepackage     [T1]            {fontenc}
\usepackage                     {color}
\usepackage                     {amsmath}
\usepackage                     {graphicx}
\usepackage     [english]       {babel}
\usepackage     [round]         {natbib}

\usepackage                     {hyperref}
\usepackage						{cleveref}
\usepackage						{subfigure}
\usepackage						{multirow}
\usepackage						{booktabs} % Allows the use of \toprule, \midrule and \bottomrule in tables
\usepackage						{multicol} % Required for creating multiple columns in slides
\usepackage		[version=4]		{mhchem}
%\usepackage{cvpr}
%\usepackage{times}
\usepackage{epsfig}
\usepackage{placeins}
\usepackage{amssymb}
% links
\definecolor{darkblue}{rgb}{0.0,0.0,0.4}
\definecolor{darkgreen}{rgb}{0.0,0.4,0.0}
\hypersetup{
    colorlinks,
    linkcolor=black,
    citecolor=darkgreen,
    urlcolor=darkblue
}
%\renewcommand{\partname}{}
%\renewcommand{\thepart}{}

\begin{document}
  %% title pages similar to providet template instead of maketitle
  %% this will generate title pages similar to the template provided
%% by the Department of Physics and Astronomy Heidelberg
%%
%% More information:
%% http://www.physik.uni-heidelberg.de/aktuelles/studium/
%% (PDF link: ...studium/download/145/Vorlage_Diplomarbeit_Formular.pdf)

%% Titleintro
\thispagestyle{empty}
\begin{center}
  \renewcommand{\baselinestretch}{2.00}
  \Large\sffamily
  Department of Physics and Astronomy\\
  \large University of Heidelberg
  \par\vfill\normalfont
  Master thesis\\
  in Physics\\
  submitted by\\
  Elsa  Wilken\\
  born in Hamburg\\
  2018
\end{center}
\newpage

%% Titlepage
\thispagestyle{empty}
\begin{center}
  \renewcommand{\baselinestretch}{2.00}
  \Large\bfseries\sffamily
    Retrieval Advances of BrO/\ce{SO2} Molar Ratios from NOVAC\\
  \par
  \vfill
  \large\normalfont
  This Master thesis has been carried out by Elsa Wilken\\
  at the\\
  Institute for Environmental Physics, University of Heidelberg, Germany\\
  under the supervision of\\
  Prof. Ulrich Platt
  %% additionally insert second supervisor here if carrying out an
  %% external diploma thesis. Reduce vspace in L. 44 accordingly.
\end{center}\par
\vspace{5\baselineskip}

% reset baselinestretch
\renewcommand{\baselinestretch}{1.00}\normalsize % or english title page
  %% Abstract page
%% =============
%%
%% Content of abstract pages has been put into seperate pages to simplify
%% word counting. Use e.g. the unix command
%%   wc abstract-ger.tex
%% or
%%   wc abstract-eng.tex
%% to get the number of words contained in these files.
\thispagestyle{empty}
\begin{center}
\begin{minipage}[c][0.49\textheight][t]{0.95\textwidth}
	\small
	%\textbf{Optimierte Bestimmung von molaren BrO/\ce{SO2} Verhältnissen aus NOVAC Daten
	%}\par
	\textbf{Zusammenfassung}\par
	\vspace{\baselineskip}
	%% Latex markup und Zitate funktionieren auch hier


Die Messung der absoluten Menge und von Konzentrationsverhältnissen vulkanischer Gas Emissionen geben Einsicht in magmatische Prozesse. Das Network for Observation of Volcanic and Atmospheric Change (NOVAC) besteht aus einem System von automatisierten UV-Spektrometern, welche die Gas Emissionen der Vulkane aufzeichnen. Die Emission von BrO und \ce{SO2} kann mithilfe von Differenzieller optischer Absorptionsspektroskopie (DOAS) aus den aufgenommen Spekren bestimmt werden wobei die optische Absorption in der Fahne mit einem Reference Spectrum verglichen wird. Dies setzt voraus, dass das Reference Spectrum frei von Vulkanische Gasen ist. Typischerweise wird das Reference Spectrum für einen Scan bei einem Elevationswinkel aufgenommen welcher welcher so gewählt wird dass das Instrument nicht in die Fahne schaut. Es hat sich jedoch gezeigt, dass auch diese Spektren noch durch Vulkanische Emissionen verunreinigt sein können. Als alternative Referenzspektren könnten 1) ein theoretisches Solar Atlas Spektrum oder 2) ein nicht verunreinigtes Referenz Spektrum des selben Messgeräts dienen. Option 1) hat den Nachteil einer verringerten Messgenauigkeit, da Instrumenteneffekte hier modelliert werden müssen und ist daher nur für das typischerweise in hoher Konzentration vorkommende \ce{SO2} anwendbar. Option 2) setzt voraus, dass das Referenzspektrum unter ähnlichen Wetter- und Strahlungsbedingungen aufgenommen wurde. Wir verwenden die erste Methode um (\ce{SO2}) Kontaminierung zu identifizieren und greifen für die Bestimmung der Gas Konzentration auf die zweite Methode zurück um eine hohe Qualität der Messung sicher zu stellen. Im Folgenden stellen wir unsere Methode für NOVAC Daten von den Vulkanen Tungurahua und Nevado Del Ruiz vor.
\end{minipage}
\vfill
  \begin{minipage}[c][0.49\textheight][b]{0.95\textwidth}
	\small
%	\textbf{Retrieval advances of BrO/\ce{SO2} molar ratios from NOVAC
%	}\par
	\textbf{Abstract}\par
	\vspace{\baselineskip}
	%% Latex markup and citations may be used here


Measurements of magnitude and composition of volcanic gas emissions allow insights in magmatic processes. Within the Network for Observation of Volcanic and Atmospheric Change(NOVAC) automatically scanning UV-spectrometers are monitoring gas emission at  volcanoes. The emissions of BrO and \ce{SO2} can be retrieved from the recorded spectra by applying Differential Optical Absorption Spectroscopy(DOAS) and comparing the optical absorption of the volcanic plume to the background. Therefore, the background spectrum must not be affected by volcanic influence. Classically, the background spectrum is taken from the same scan but from an elevation angle which has been identified to be outside of the volcanic plume. However, experience shows those background spectra can still be contaminated by volcanic gases.  Alternatively reference spectra can be derived from 1) a theoretical solar atlas spectrum or 2) a volcanic-gas-free reference spectrum recorded by the same instrument. 1) comes with a drawback of reduced precision, as the instrumental effects have to be modeled and added to the retrieval. For 2), the alternative reference spectrum should be recorded at similar conditions with respect to meteorology and radiation. We use the first option to check for contamination and the second to evaluate the spectra to maintain a god fit quality. We present our approach and its results when applied on NOVAC data from Tungurahua and Nevado Del Ruiz.

\end{minipage}\par
%\vfill
\end{center}


  \tableofcontents
  %% Put your contents here
	\chapter{Introduction}	
	
%% Introduction page
%% =============
%%
Volcanic activities on Earth have  always shaped the earth surface and influenced atmospheric processes. Volcanoes are often particularly recognized by their dramatic consequences of a major volcanic eruption. But volcanoes influence our lives in more than this way. Volcanic gases can effect the weather (timescales of days to weeks) or the climate (timescales of months to years) \cite{schmidt2015volcanismarticle}.
Examples are the lake eruption in Iceland (1783-1784) followed by a very hot summer and a cold winter in central Europa \cite{thordarson2003atmospheric} and the Tambora eruption, indonesia in 1815 which caused the "year without summer" in 1816.\\
%
\newline
%
Considering the plate tectonics of earth  most volcanoes are caused by diverging or converging of the continental plates and therefore located at the margins of the continental plates.
Another possibility for occurrence of volcanoes is the the interior of continental or oceanic shelves. \cite{schmincke2000vulkanismus}\\
The most abundant volatile species released during a volcanic eruption are water vapour (H$_2$O; relative amount of the plume: 50\%-90\%) and carbon dioxide (CO$_2$; relative amount of the plume: 1\%-40\%) \cite{platt2015quantification}. But the short effects of those two gases are rather low since there effect on atmospheric composition is negligibly due to the high abundance of atmospheric H$_2$O and CO$_2$. But on timescales of the age of the earth the volcanic emission of H$_2$O and CO$_2$ are the source of our current atmosphere. \cite{schmidt2015volcanism}\\ 
A typically volcanic plume consists of many different gases alongside H$_2$O and CO$_2$  sulfur dioxide (SO$_2$) contributes with 1\%-25\% to the plume, hydrogen sulfide (H$_2$S) with 1\%-10\% and hydrogen chloride with (HCl) 1\%-10\%. Furthermore there are trace gases for example carbon disulfide (CS$_2$), carbon sulfide (COS) carbon monoxide (CO) hydrogen fluoride (HF) and hydrogen bromide (HBr) \cite{platt2015quantification}\\
%
A decrease of stratospheric ozone (O$_3$) has been observed after the eruption of  El Chickon in 1982 and the eruption of mount Pinatubo 1991. A depletion stratospheric O$_3$ results in ozone holes. The depletion comes from volcanic aerosols which serve anthropogenic chlorine/bromine into more reactive forms \cite{solomon1998ozone}. 
%
Volcanic gases can alter the radiative balance of the earth in timescales relevant for climate change due to scatter and absorption of solar radiation \cite{schmidt2015volcanism}.\\
%
The gas composition of the volcano plume change with activity and could be a indication for the processes inside the earth.\\ 
%
In this work we are particularly interested in the ratio of BrO and SO$_2$. The halogen sulfur ratio is a proxy for volcanic processes. Therefore we make the assumption
that the ratio of BrO and SO2 contains informations about its degassing source depth. A change in BrO/SO2 prior to eruption was observed at Etna and Nevado del Ruiz.\\
%
\newline
%
To gain further knowledge about the volcanoes the Network for Observation of Volcanic and Atmospheric Change (NOVAC) was installed. NOVAC is a Network of DOAS Instruments located next to about 30 volcanoes in America, Africa and Europe. At every Volcano there are two to four DOAS Instruments installed, recording record back-scattered solar radiation spectra at different viewing angles.\\
NOVAC is a network which produces a large amount of data and we have the chance to evaluate long time periods which is a unique opportunity to study correlations of the trace gases.\\
Since the conditions at volcanoes are rough, the instruments need to be rather simple to keep the maintenance cheap and to assure a longer lifetime of the instruments. So we need to waive on temperature stabilization even at the expense of the quality of the data.\\
%
\newline
%
One possibility to measure the volcanic trace gases is to use Differential Optical Absorption Spectroscopy \cite{platt2008differential}. DOAS exploit the wavelength dependency of the absorption of light. Here the gas emissions can be retrieved
from the quotient of the absorption signal of the volcanic plume and a
reference region. This will be explained in a further chapter.\\
%
\newline
%
The reference region, is usually treated as free of
volcanic trace gases. If the reference region is for any reason
contaminated by volcanic trace gases, the reference spectrum has to be
replaced by a volcanic-gas-free reference. Alternative spectra could be for example a
theoretical solar atlas spectrum or a volcanic-gas-free reference
spectrum recorded in the temporal proximity(eg. a day before) by the same instrument. 
The first option comes with the drawback of reduced precision, as the
instrumental effects have to be modeled and added to the retrieval. The
reduction in precision is acceptable for the SO2 retrieval, but not suitable
for a BrO retrieval because then most data would be below the detection
limit. For the second option, the alternative reference spectrum should
have been recorded at similar conditions with respect to meteorology and
radiation as well as in the temporal proximity due to instrumental changes
with time and ambient conditions. We combined both options in order to
achieve both, enhanced accuracy but still maximum possible precision of
the SO2 and BrO retrievals. We present an algorithm which finds the
optimal reference spectrum automatically. As first step, a possible SO2
contamination of the standard reference is checked by a comparison with
the theoretical solar atlas. If a contamination is detected, as second step,
the algorithm picks a volcanic-gas-free reference (beforehand
automatically checked for contamination) from another scan.\\
%
\newline
%
In this work we are mainly dealing with data from Tungurahua in Ecuador in the timespan of 01.08.2008 to 30.07.2009. Later on, we will also show the results of Nevado del Ruiz a volcano located in Colombia.\\
    \part{Theoretical background}
\chapter{Volcanism and volcanic chemistry}

\section{Volcanism}
The high thermal energy in the deep interior of the earth is mostly well separated from the earth’s surface by the earth’s crust. A volcano is a geological structure that allows magma to reach the earth’s surface. Such a phenomenon can occur in various ways. In the following paragraphs the different types of volcanoes are described.
\paragraph{ Mid-ocean ridge volcanism}
The mid-ocean ridge volcanism can be traced back to tectonic processes of oceanic plates. The spreading of two plates, that are pulled apart, leads to a thinning of the oceanic earth crust. This way solid material from the upper mantel (lower than 100 km) can ascend to depths of approximately 50 km. As the pressure at this depth is much lower, the mantle material starts to melt to basaltic magma that fills the gap between the two plates.
\paragraph{ Continental rift zone volcanism}
Similar to mid-ocean ridge volcanism continental rift zone volcanism are located at two continental plates that are pulled apart.
\paragraph{ Subduction zone volcanoes}
Subduction zone volcanoes occur if an oceanic plate converges under another plate (oceanic or continental). This way the descending plate penetrate into the lower mantle. At a depth of 80-150 km the water of this plate evaporates and rises and causes the mantle material above to melt. The resulting water-rich magma mainly consists of andesite. Subduction zone volcanoes are known for their violent eruptions caused by the low viscosity magma.
\paragraph{ Hot-spot volcanoes} Hot-spot volcanoes occur on continental or oceanic plates. This type of volcanoes arises from a hot spot at the coremantle boundary inside earth that leads to a plume in the mantle where solid material can rise. This material melts to basaltic magma at a depth of 100-150 km. Through a futher rise also other types of magma (e.g. rhyolitic, more-viscous magma) can arise.


\section{Volcanic gases and their impact on the climate}
\begin{figure}
	\centering
	\includegraphics[width=0.8\linewidth]{Bilder/Simon/Bilder_Tung/Climate_Influence}
	\caption{Influence of volcanic eruptions and quiet degassing on earth climate. Redrawn on the basis of \cite{robock2000volcanic}}
	\label{fig:climateinfluence}
\end{figure}
Volcanoes emit various gases (see \cref{tab.volcemissions}) in the atmosphere. This can occur due to volcanic eruptions or due quiet degassing. Gas emitted by quiet degassing remains in the troposphere while eruptions can inject volcanic gases up to the stratosphere \citet{robock2000volcanic}. The larger lifetime in the stratosphere and a larger sensitivity of the stratospheric chemistry to volcanic gases leads to an higher impact on the earth climate of these gases.
Volcanic gases have a large influence on the earth climate especially \ce{CO2} and \ce{SO2}  or more specific its oxidation product sulfur acid.\\ 
The relevance of  \ce{CO2} for the climate is a subject of many discussions about the climate change. Compared to other atmospheric \ce{CO2} sources, the share of volcanic  \ce{CO2} is rather low.\\
\\
Even though the \ce{SO2} emissions during eruptive episodes are up to one order higher than during quite degassing episodes, \cite{halmer2002annual} estimates that quiescent degassing contributes 40\% of the accumulated \ce{SO2} between 1972 to 2000.\\
\cite{halmer2002annual} estimated the mean annual \ce{SO2}  emitted from volcanoes from 1972 to 200 as 7.5 to 10.5TgSyr$^{-1}$, while the anthropogenic \ce{SO2}  amount for 2000 is estimated as 55TgSyr$^{-1}$ \citep{IPCC}. Despite the less \ce{SO2}  occurring from volcanoes the impact may be higher as the impact of the anthropogenic \ce{SO2}. \cite{graf1997volcanic} supposed that the volcanic \ce{SO2}  has a higher impact on the climate since its reaches up to the stratosphere while the anthropogenic \ce{SO2}  is mostly located in planetary boundary layer. In the lower troposphere sulphuric acid has a lifetime of about a week whereas the lifetime in the stratosphere is about a year \citep{IPCC}.\\
Sulphuric acid in the atmosphere increases the earth albedo due to direct backscattering radiation. Additional the condensation on sulphuric acid particles leads to finer droplets and thus to more stable and more white clouds. This increases the albedo as well \citep{twomey1974pollution}.
Volcanic particles can be surfaces for heterogeneous reaction. The result is a depletion of stratospheric ozone, and thus a more high energetic solar flux on the earth surface.
Large particles may backscatter IR radiation from the earth surface and the lower atmosphere, leading to a small reduction of the net cooling of the lower troposphere.
In the upper troposphere or stratosphere absorption of IR or UV radiation results in a net heating in the stratosphere and a cooling at the earth surface.
\Cref{fig:climateinfluence} shows the above described effects and their localization in the atmosphere.
The dominating radiative effect of volcanic gases is a cooling of the earth atmosphere due to  more backscattered radiation, more diffusive scattering\citep{robock2000volcanic}.
\cite{IPCC} records a volcanic radiative forcing of $-0.11Wm^{-1}$ between 2008 and 2011. For comparison the radiative of  \ce{CO2} is estimated as  $1.68Wm^{-1}$.





\section{Volcanic degassing}
\section{Volcanic plume chemistry}
Volcanoes are emitter of many gases and aerosols particles, which are rising and forming the volcanic plume. 
The temperature where the gases and aerosols are emitted is approximately 500$^{\circ}C$ \citet{gerlach2004volcanic}.
Due to the high temperatures the gas raises, cools down and mix up with ambient air. This process leads to many chemical reactions. The large amount of aerosols catalyses heterogeneous reactions.\\
The volcanic gases are listed in \Cref{tab.volcemissions}.\\
In the following the chemical reactions of the plume constituents \ce{BrO}  and \ce{SO2} will be discussed. This are the gases observed in this thesis. 
	\begin{table}[h]
	\begin{tabular}{p{2cm}p{1.0cm}p{1.0cm}p{1.0cm}p{1.0cm}p{1.0cm}p{1.0cm}p{1.0cm}p{1.0cm}p{1.0cm}}
		\toprule
		%	\toprule
		Species	&  \ce{H2O}  & \ce{CO2}  & \ce{SO2} &  \ce{H2S} &  \ce{COS} & \ce{SC2} & \ce{HCl} & \ce{HBr} & \ce{HF} \\
		\toprule

		\multirow{ 3}{*}{\% / vol} & 50 & 1 & 1 & 1 & $10^{-4}$ & $10^{-4}$ & 1 & &\\
		&-&-&-&-&-&-&-&?&<\\
		& 90 & 40 & 25 & 10 & $10^{-2}$  & $10^{-2}$  & 10 &  & $10^{-3}$  \\ 
		\midrule
		\multirow{ 3}{*}{Tg / year} &  & & 1.5 & 1 &0.005 & 0.007 & 0.4 &0.0078 &0.06\\
		&?&75&-&-&-&-&-&-&-\\
		&&& 50 & 2.8 & 0.1 & 0.096 &11  & 0.1  & 6\\ 
		\bottomrule
	\end{tabular}
		\caption{Volcanic gas constituents at the emission vent and global estimated source strength. Adapted from \cite{textor2004emissions}.}
		\label{tab.volcemissions}
\end{table}

\subsection{Sulphur species\label{chap:so2}}
Sulphur species are the third most abundant gases in volcanic plumes. Hereby \ce{SO2} contributes with about 25\% and \ce{H2S} with 1 to 10\%. Only \ce{H2O} and \ce{CO2} have a larger share on the volcanic gases in the plume \Cref{tab.volcemissions}.\\
Outside of the volcanic plume the \ce{SO2}  amount is negligible with approximately 1ppb. In contrast the \ce{SO2}  amount inside he plume can easily reach 1ppm \cite{Coppenheimer 2003}
When \ce{H2S} escapes from the volcano vent, it enters the oxidizing conditions in the atmosphere. The conversion of \ce{H2S} into \ce{SO2} starts with:
\begin{equation*}
\ce{H2O}+\ce{OH}\rightarrow \ce{H2O}+\ce{SH}
\end{equation*}
The \ce{SH} radical goes through a series of reactions, leading to the \ce{SO2} formation \cite{Seinfeld}\\
\ce{SO2} is removed from the atmosphere by dry or wet
deposition. At homogeneous reactions the lifetime is from 1-3 weeks \cite{robock2000volcanic}. Heterogeneous reactions that take place on particles or liquid phases lead to much faster depletions. But this has not yet been observed in volcanic plume measurements.\\
Further discussions of the stability of \ce{SO2}  in the atmosphere can be found at \cite{lubcke2014optical}.\\
However, \ce{SO2} can be considered as stable several hours after the release of the volcanic vent.\\ The long lifetime  alongside with the negligible amount of \ce{SO2} in the atmospheric background makes \ce{SO2} a good tracer of the volcanic plume.
Therefore relative to other trace gases \ce{SO2} may be used to examine their evolution independent of the plume dispersion.\\
One attempt to use \ce{SO2} to examine other trace gases is made by \cite{bobrowski2007reactive}. They found a higher BrO/\ce{SO2} ratio at the edges of volcano plumes (Mt. Etna on Sicily, Italy, in August–October 2004 and May 2005 and Villarica in Chile in November 2004) and concluded that the BrO amount is higher at the edges due to the insufficient mixing with ozone rich air inside of the plume (see \ref{fig:broplume}).\\
In this thesis it is assumed that \ce{SO2}  is stable on time-scales occurring with ground based remote sensing measuring of about 20 minutes. \\



\subsection{Bromine oxide}
\begin{figure}
	\centering
	\includegraphics[width=0.8\linewidth]{Bilder/Simon/Bilder_Tung/BrO_Plume}
	\caption{schematic sketch of a Bromine Explosion.
		Release of \ce{HBr} at the volcanic vent. Mixing with ambient air in the effective source region leads to \ce{Br} formation. This resulting Bromine species react to \ce{BrO} with ozone from the plume. Adapted from \cite{bobrowski2007reactive}}
	\label{fig:broplume}
\end{figure}
The amount of bromine in volcanic plumes is rather low compared to \ce{SO2}. The first time Bromine monoxid (BrO) was observed at a volcano was 2013 at the soufriere hills by \cite{bobrowski2003detection}. Since then many others were able to detect \ce{BrO}  using ground based remote sensing measurement techniques (DOAS: see \Cref{DOAS}) for example:
\citet{bobrowski2007so2}, \citet{bobrowski2007reactive},\citet{vogel2011volcanic} and \cite{lubcke2014bro}
\\
The main Bromine formation which is released from the volcano is  \ce{HBr}. \ce{BrO}  is formed due to mixing with the ozone rich atmosphere at ambient temperatures \cite{bobrowski2007reactive}.\\
Due to the raising of hot air in the volcano vent , ambient air is pulled into the vent. There temperatures of  600$^{\circ}C$ to 1200$^{\circ}C$    prevent the formation of \ce{BrO}. Only Br is formed. \ce{BrO}  occur after further cooling and mixing while rising. When the temperature cooles down to ambient conditions the so called "Bromine Explosion" causes a non linear formation of \ce{BrO}.
The "Bromine Explosion" is illustrated in \Cref{fig:broexplosion} and can be described with the following reaction cycle:
\begin{align}
\ce{HBr}_{gas} &\rightarrow \ce{Br}^{-}_{aq} + \ce{H}^{+}_{aq} \label{eq1}\\
\ce{HOBr}_{(gas)} &\rightarrow \ce{HOBr}_{(aq)} \\
\ce{HOBr}_{(aq)} + \ce{Br}^{-}_{(aq)} + \ce{H}^{+}_{(aq)} &\rightarrow
\ce{Br}_{2(aq)} +  \ce{H2O}\label{eq2}\\
\ce{Br}_{2,aq} &\rightarrow \ce{Br}_{2,gas} \\
\ce{Br2} + h\nu &\rightarrow 2 \ce{Br} \\
\ce{Br} + \ce{O3} &\rightarrow \ce{BrO} + \ce{O2} \\
\ce{BrO} + \ce{HO2} &\rightarrow \ce{HOBr} + \ce{O2} \\
\ce{BrO} + \ce{BrO} &\rightarrow 2 \ce{Br} + \ce{O2} \\
\ce{BrO} + \ce{BrO} &\rightarrow \ce{Br2} + \ce{O2}
\end{align}
The gaseous \ce{HBr} emitted by the volcano is splited heterogeneously int H$^{+}$ and Br$^{-}$ (\cref{eq1}). Inside of an aerosol it forms with HOBr \ce{Br2} and \ce{H2O} (\cref{eq2}). \ce{Br2} evaporates to the gaseous phase and splits photolytic  into 2Br. Including an ozon molecule (\ce{O3}) the two Br react to 2BrO. The last step o the circle visualized in \Cref{fig:broexplosion} with blue lines is the reaction of a BrO with \ce{H2O} to an HOBrO molecule condensing into the liquid phase  and thuse closing the circle. The non linear explosion occurs due the formation of two BrO particles from one \ce{Hbr} from the volcano.\\
The BrO formation is slightly diminished due to self reaction of BrO molecules marked with the red lines in \cref{fig:broexplosion}. The 2BrO react with themselves and form \ce{Br2} or may split photolyticaly into 2Br.\\
The BrO concentration reaches a maximum approximately 5 minutes after emission and then remains constant for the next 25 minutes \citet{lubcke2014bro}.
 \begin{figure}
 	\centering
 	\includegraphics[width=0.7\linewidth]{Bilder/rat_diff}
 	\caption{BrO/\ce{SO2} ratio as a function of the distance from the volcanic vent with a constant wind speed of 10 m/s. From \cite{lubcke2014optical}.}
 	\label{fig:ratdiff}
 \end{figure}
 
\begin{figure}
	\centering
	\includegraphics[width=0.7\linewidth]{Bilder/Simon/Bilder_Tung/BrO_Explosion}
	\caption{Bromine reactions inside of a volcanic vent. The release of \ce{Hbr} at the volcaninc vent is drawn  in orange. Inside of aerosols heterogeneous dissociation with HOBr forms \ce{Br2}. Then \ce{Br2} splits photolytically into single Br radicals. \ce{BrO}  results through a reaction with  \ce{O3}upon mixing with ambient air. Reactions with  \ce{H2O}  forms \ce{HOBr} creating an autocatalytic cycle. The reaction cycle along the blue lines are called Bromine explosion. From \cite{WarnachSimon}.}
	\label{fig:broexplosion}
\end{figure}

\section{Using the \ce{BrO}/\ce{SO2}  ratio to study volcanic activity}
\begin{figure}
	\centering
	\includegraphics[width=0.9\linewidth]{Zwischenbericht2018/Bilder/so2_bro}
	\caption{Dependency of the ratios of different volcanic trace gases on depth. Data originate from Stromboli volcano. From \cite{lubcke2014optical} reproduced from Burton et al. (2007)}
	\label{fig:so2bro}
\end{figure}    
	
	Volcanic degassing is influenced by many factors, which can be exploit to study volcanic activity by using the gas composition of the volcano plume. Therefore remote sensing should be an additional tool for forecasting of volcanic activity next to classical monitoring techniques like seismographic and deformation measurements.\\
	Inside of volcanoes volatiles are in solution in magmatic melt. The Henry law \Cref{Henrylaw} describes the necessary conditions for gas formation:
	\begin{equation}
	P = K_{H}\cdot c
	\label{Henrylaw}
	\end{equation}
	Here P is the partial pressure at equilibrium of the solute, c is the concentration and $ K_{H}$ is the Henry constant which is anti proportional to the solubility $\alpha$ ($\alpha = \frac{1}{ K_{H}}$).
	If the partial pressure of the gas solute (in this case a magmatic gas constituent) exceeds the pressure of the surrounding solvent, a formation of gaseous bubbles occur. Otherwise, if the partial pressure of the gas in the solution is below the surrounding pressure the formation of gas bubbles stops.\\
	The solubility $\alpha$ depends on the temperature, the chemical composition and on the solvent (here magma). Whereas the partial pressure of the constituent depends on the surrounding pressure. The pressure below the volcanic vent increases with depth, this leads to a correlation between the partial pressure of the constituents and the depth.
	The result is, that the gas starts exsolving at a certain depth depending on the  partial pressure of the constituent. Thus the gas bubble formation increases with rising magma. But at a certain depth the percentage of solved gas is different for each volcanic gas. The result is, that the composition of the gases changes with depth. So gas ratios contain information about its originating source depth.\\
	Prior to volcanic eruptions the magma starts raising since the gas is mostly less dense than the magma, it raises faster and could be therefore a indicator for its origin source depth thus a indicator for the volcanic activity.\\
	\Cref{fig:so2bro} shows the ratios of  \ce{H2O} / \ce{CO2},S/CL, \ce{CO2}/S as a function of the pressure respectively on the depth.
	%
	\citet{noguchi1963prediction} found decrease Cl/S prior to eruptive periods
	%
	\citet{pennisi1998variations} observed lower CL/S ratio during eruptive periods than not eruptive periods at Mt. Etna.
	%
	\citet{burton2007magmatic}  found \ce{CO2}/\ce{SO2} and \ce{SO2}/hCl ratios 3-5 times higher during explosions  compared to quiet degassing episodes.
	%
	the authors compared these data to gas formation simulations for different degassing source depth they concluded that these eruptions were driven by gas slugs from deeper levels where the ratios were higher while quiet degassing originates from from shallow magma\\
	%
	Especially halogen-sulfur ratios are interesting as possible tracer for the volcanic activity since the ambient air concentrations are negligible
	BrO/\ce{SO2}  curves equivalent as in \Cref{fig:so2bro} are yet not available due to lack of  bromine solubility curve but the following observations were made:
	%
	Changes of \ce{BrO}/\ce{SO2}  were found by \citet{bobrowski2006bromine}: multiple eruptions between 2006 and 2009 highest ratios 2-3 month before the eruptions the ratio then decreased and was lowest during eruptive phase. Therefore it could be concluded that bromine exsolves earlier, at lower depth than sulphur.
	%
	\citet{lubcke2014bro} found decrease of \ce{BrO}  5 month prior to th eruption 2012 at Nevado Del Ruiz can also be attributed to a earlier exsolution of bromine during rising magma.
	%
	Despite the lack of the solubility curve of \ce{BrO}  until now, the \ce{BrO}/\ce{SO2}  has a great potential for investigations of the volcanic activity. The first reason is, that both gases can be measured with remote sensing by DOAS instruments. For examples ground based measurements by \cite{bobrowski2007reactive}, \cite{lubcke2014optical} or satellite based measurement by \cite{Hörman et al 2013} or \cite{Beirle at al 2014}. The advantage of remote sensing techniques is the possibility of measuring during eruptions which is with in situ measurements not always possible.
	Secondly due to the NOVAC network (See \ref{NOVAC}) continues measurements are possible.\\
	Another reason for the research on \ce{BrO}/\ce{SO2}  ratios at volcanoes is the constance of the ratio from 5 to at least 30 minutes after release \citep{bobrowski2007reactive};\citep{lubcke2014optical}. As well as the constance from 5 to 20 km off the volcano see \ref{fig:ratdiff}. This ensures that the data measured from different positions or at different conditions are comparable.\\
	\\
	This chapter motivated the research on the \ce{BrO}/\ce{SO2}  ratio as a tracer for volcanic activity.
	


	\subsection*{Tungurahua \label{Tung}}
	\begin{table}
		\centering
		\begin{tabular}{|p{4cm}|p{3cm}|p{3cm}|p{3cm}|}
			%	\toprule
			Instrument	&D2J2140&I2J8546& I2J8548\\
			\toprule
			compass&90.0	&	30.0	&	360.0	\\
			latitude&-1.453584	&	-1.543492	&-1.404719	\\
			longitude&-78.513047	&-78.517179	&	-78.494777	\\
			altitude&2609.980	&	2782.352	&	2911.253	\\
			volcano&Tungurahua	&Tungurahua	&	Tungurahua	\\
			site&pillate	&	bayushig	&	huayrapata	\\
			observatory&igepn	&	igepn	&igepn	\\
			serial&D2J2140	&	I2J8546	&	I2J8548	\\
			spectrometer&S2000	&	S2000	&S2000	\\
			instrumenttype&gothenburg	&gothenburg	&gothenburg	\\
			version&2.1	&2.1	&	2.1	\\
			\bottomrule
		\end{tabular}
		\caption{Technical data of the instruments installed at the Tungurahua volcano.}
		\label{tab:TInstruments}
	\end{table}
	Tungurahua is a steep-sided andesitic-dacitic subduction zone volcano located in the Ecuadorian Andes (Lat: 1.467$^{\circ}$S; Long: 78.442$^{\circ}$W). 2014 Tungurahua was one of the most active volcanoes in southern America, since then the activity was decreasing. Tungurahua is 5023m high and is one of the defining volcanoes of the eastern volcanic rows in Ecuador \citep{hall1999tungurahua}.
	\\
	The modern volcano was formed by a sequentially construction of three major edifices on a basement of metamorphic rocks. It has a total diameter of 12 km. Tungurahua I was roughly located at the same place as today and was build up in the mid-Pleistocene, mainly by andesitic and dacitic lava flows as well as interbedded tephra. Tungurahua II was formed in the past 14,000 years as a result of the collapse of the initial edifice.\\
	3000 years ago the Tungurahua II edifice collapse. The collapse of Tungurahua II produced a large debris-avalanche deposit and a horseshoe-shaped caldera which is open to the west side. Tungurahua III the current glacier caped stratovolcano was constructed in this caldera \citep{GlobalVolcanismProgram}.\\
	The current ongoing long term eruption started in October 1999. The eruptive ohase was preceded by hydrothermal tremors between 1994 and 1997 \citep{samaniego}\textcolor{red}{reference finden}.\\
	From September 1998 to July 1999 an increase of seismic activity like volcano tectonic earthquakes indicates the raising of magma. This eruptive phase is characterized by alternating periods of high and low volcanic activity. \\

	At Tungurahua three instruments described  in \Cref{tab:TInstruments}, with data recorded in the time span from July in 2008 to August in 2009 are used in this thesis.
	\Cref{tab:TInstruments} shows the exact position, of the instruments, the altitude, and other specifications of the instrument. \\

	\subsection*{Nevado Del Ruiz}
	Nevado Del Ruiz is a glacier-covered, subduction zone volcano which is located in the Central Cordillera of Colombia, 140 km west of Bogota
	(Lat: 4.892$^{\circ}$N; Long: 75.324$^{\circ}$W). Nevado De Ruiz has a hight of 5389 m and covers an area of more than 200 km$^2$.
	Three major edifices has been constructed since the beginning of the Pleistocene, consisting of andesitic and dacitic lavas and pyrolastics \citep{GlobalVolcanismProgram}. \\
	The current cone is build within the caldera of older edifice and consists of a cluster of lava domes. The crater on the summit the name is Arenas crater, with a diameter of 1 km and a depth of 240 m. 
	The last big eruption was in 1985. This was South America's deadliest eruption.\\
	In this thesis the data of two NOVAC instruments (see \Cref{tab:NVDRInstruments}) in the time from the end of 2009 to the end of 2011 are used. \\
	\\
	\begin{table}
		\centering
	\begin{tabular}{|p{4cm}|p{3cm}|p{3cm}|}
		%	\toprule
		Instrument	&D2J2200&D2J2201\\
		\toprule
		compass&115.0		&59.0		\\
		tilt&0.0		&0.0		\\
		latitude&4.900917		&4.876183		\\
		longitude&-75.335134		&-75.353408		\\
		altitude&4866.500		&4494.259		\\
		volcano&Nevado Del Ruiz		&Nevado Del Ruiz		\\
		site&bruma	&alfombrales\\
		observatory&ingeominas		&ingeominas		\\
		serial&D2J2200		&D2J2201		\\
		spectrometer&S2000		&S2000		\\	
		instrumenttype&gothenburg		&gothenburg		\\
		version&2.2		&2.2		\\
		softwareversion&1.82		&1.82		\\
		compiledate&Feb 19 2009		&Feb 19 2009		\\
		\bottomrule
	\end{tabular}
	\caption{Technical data of the instruments installed at the Nevado Del Ruiz volcano.}
	\label{tab:NVDRInstruments}
\end{table}
	\chapter{Remote sensing of volcanic gases}

	In this thesis we are interested in the volcanic trace gases \ce{SO2} and \ce{BrO}, both measured with the Differential Optical Absorption Spectroscopy (DOAS) a remote sensing technique proposed by \cite{platt1980observations}

	

	\section*{Beer-Lambert Law}
	The Beer-Lambert law describes the attenuation of light when travelling through a material.\\
	This section will give an overview about the reasons for decreasing light intensity when going through a medium.\\
	The Beer-Lambert law describes the attenuation of light when travelling through a material.\\
                                                                                                                                                                                                                                                                                       ,
	Atoms and molecules exists in several energy states, depending on the different electron configuration. Moreover molecules have additionally rotation and vibration states, also enclose to the energy states. If a photon energy matches the energy gap between two possible energy states, this includes, that the lower energy state is occupied and the selection rules are fulfilled  the molecule could absorb the photon, remaining in a higher energy state.\\
	The additional energy could be loosed by collision with another molecule or by emission. But since the direction of the emitted photon is mostly not the same direction of the absorbed photon the intensity I$_{0}$ of the light before passing the medium is higher than the intensity I after travelling the distance L through the medium.\\
	This can be described as:\\ 
	\begin{equation}
	I\left(L,\lambda\right) = I_{0}\left(\lambda\right)\cdot exp\left(-\int^{L}_{0}\sigma\left(\lambda,p(l),T(l)\right)\cdot c\left(l\right)dl\right)
	\end{equation}
	where $\lambda$ is the wavelength,$c\left(l\right)$ is the location-dependent concentration of the trace gas of interest. $\sigma\left(\lambda,p,T\right)$ is the absorption cross section, $\sigma\left(\lambda,p,T\right)$ is unique for each molecule and depends on pressure p and on the temperature T.\\
	%
	An important quantity used in many optical remote sensing techniques is the optical density $\tau$. The optical density is a measure for the weakening of radiation when going through a material. At a volcano the variation of temperature and pressure in different viewing angles is negligible, thus, $\sigma\left(\lambda,p(l),T(l)\right)$ is independent of $l$ required to consider it within the integral. Then  $\tau$ can be calculated using the Beer-Lambert law:
	\begin{equation}
	\tau = ln\left(\frac{I_{0}\left(\lambda\right)}{I\left(\lambda\right)}\right) = \sigma\cdot S
	\end{equation}
	The column density $S$ can be calculated as:
	\begin{equation}
	S = \int_{0}^{L}c\left(l\right)dl
	\end{equation}
	The column density is the concentration of the trace gas when integrating along the light path, the dimension of $S$ is therefore the number of molecules divided by an area: $\frac{molec}{cm^2}$.\\
	%
	\begin{figure}
		\centering
		\includegraphics[width=0.7\linewidth]{Bilder/DOASFunction}
		\caption{Schematic sketch of the DOAS measurement of volcanic plume constituents. The column density of the plume constituent of interest is retrieved by comparing the spectrum $I\left(\lambda \right)$ which is measured through the plume with the spectrum  $I_0^{'}\left(\lambda \right)$ measured outside of the plume.}
		\label{fig:doasfunction}
	\end{figure}
	
	When measuring in atmosphere, the situations gets more complex, since we need to deal with several absorbers and scattering processes have to be taken into account. 
	Scattering processes in the atmosphere can be roughly grouped in Rayleigh scattering, scattering at very small particles and Mie scattering, scattering at larger particles (radius$\approx \lambda$). The effects on the spectrum caused by scattering need to be considered in the calculations.  One possibility is to treat scattering effects as pseudo absorbers with the respective extinction coefficients for Rayleigh ($\epsilon_R$) and  Mie ($\epsilon_M$) scattering. 
	%\textcolor{red}{das muss ausführlicher eingeführt werden}
	%
%	\begin{equation}
%	I\left(L,\lambda\right) = I_{0}\left(\lambda\right)\cdot expt\left(-\int^{L}_{0}\sum_{j}\sigma_{j}\left(\lambda,p,T\right)\cdot
%	c_{j}\left(l\right)+\epsilon_R\left(\lambda,l\right)+\epsilon_{M}\left(\lambda,l\right)dl\right)
%	\label{eq:lbe}
%	\end{equation}
	\begin{equation}
	\tau = \int^{L}_{0}\sum_{j}\sigma_{j}\left(\lambda,p,T\right)\cdot
	c_{j}\left(l\right)+\epsilon_R\left(\lambda,l\right)+\epsilon_{M}\left(\lambda,l\right)dl
	\label{eq:lbe}
	\end{equation}
%	\textcolor{red}{Warum nicht die tau formel? Die ist übersichtlicher und ja sogar schon extra eingeführt worden.}
	The first term of \Cref{eq:lbe}: multiple absorbers $j$ are considered, the corresponding concentration depends on the position l of the light path.
	The last two terms describe the extinction due to Rayleigh and Mie scattering in the atmosphere.\\
	Inelastic scattering (for example the Ring effect) and effects due to turbulences in the atmosphere are neglected in \Cref{eq:lbe}. 

	\section{Differential Optical Absorption Spectroscopy (DOAS)\label{DOAS}}
		\begin{figure}[h]
			\centering
			\includegraphics[width=0.8\linewidth]{Bilder/Simon/Bilder_Tung/DOAS_Intensity}
			\caption{Basic idea of the DOAS principle: Light attenuate due to broad band and narrow band effects. The broad band extinction is caused by aerosols and Raylight scattering $\left(I_0\rightarrow I^{'}\right)$. The measured intensity $I$ is formed by narrow band effects due to differential absorption structures by trace gases with the optical density $\tau^{'}$. Adapted from \cite{kern2009spectroscopic}}
			\label{fig:doasintensity}
		\end{figure}
	Differential Optical Absorption Spectroscopy (DOAS) was invented in the late 1970s by \cite{perner1979detection}. This section will give an overview about the DOAS technique. More detailed information ca be found in the work of \cite{platt2008differential}\\
	DOAS uses the fact that the broad bands effect do not need to be quantified to determine the column density.
	Therefore it is not necessary to apply \Cref{eq:lbe} to real measurements.\\
	\newline
%	\textcolor{red}{Das hier ist ist unter (mathematischer)Debatte. Zumindest bei MAX-DOAS wird das nämlich eigentlich gar nicht gemacht. Der DOAS fit unterteilt gar nicht in Breitband und Schmalband. Die Erklärung hier ist eher eine Illustration wie man sich den DOAS fit vorstellen kann. IM Prinzip ist dieser Abschnitt redundant. Das steht (leider) so nicht im DOAS Buch. Ich bin hier aber voreingenommen. Zu diesem Abschnitt solltest du dir andere Meinungen einholen. ;-) : nicole sat dazu: den Kommentar verstehe ich nicht, falls doch stimmt das geasagte aber nicht }
	Differential Optical Absorption Spectroscopy uses the fact, that absorption can be divided into broad-band parts and narrow-band parts. Broad band parts are effects that only changes weakly with the wavelength,  i.e. scattering and instruments effects have a broad-band structure. 
	The narrow band part includes effects that strongly depends on the wavelength.
	Within the DOAS-Method only narrow-band absorption features of molecules are used to obtain their column densities.
	The absorption cross section of trace gases $j$ have broad-band ($\sigma_b\left(\lambda \right)$) and narrow band ($\sigma{'}\left(\lambda \right)$) features, only the narrow-band structures are used in DOAS.
	\begin{equation}
	\sigma\left(\lambda \right) = \sigma_b\left(\lambda \right) + \sigma{'}\left(\lambda \right)
	\end{equation}

	%
	With this considerations the Beer-Lambert law \Cref{eq:lbe} can be rewritten
	dividing the exponential part into a narrow-band part and a broad-band part:

	\begin{align}
	I\left(\lambda,L\right) = &\overbrace{I_{0}\left(\lambda\right)\cdot exp\left(-\int^{L}_{0}\sum_{j}\sigma_{b,j}\left(\lambda,p,T\right)\cdot c_{j}\left(l\right)+\epsilon_R\left(\lambda,l\right)+\epsilon_{M}\left(\lambda,l\right)dl\right)}^{=I^{'}_0\left(\lambda\right)} \cdot \nonumber \\
	&exp\left(-\int^{L}_{0}\sum_{j}\sigma_{j}^{'}\left(\lambda,p,T\right)\cdot c_{j}\left(l\right)dl\right)
	\label{eq:bb}
	\end{align}	
		%
	The so defined $I^{'}_0\left(\lambda\right)$ differs from $I_0\left(\lambda\right)$ only by broad band effects. With $I^{'}_0\left(\lambda\right)$ a differential optical density $\tau^{'}$ can be defined:
	\begin{equation}
	\tau^{'} = ln\left(\frac{I^{'}_0\left(\lambda\right)}{I\left(\lambda\right)}\right) = \int_{0}^{L} \sum_{j} \sigma^{'}_{j} \cdot \left(\lambda\right) \cdot c_{j}\left(l\right)dl = \sum_{j}\sigma^{'}_{j}\left(\lambda\right)\cdot S_{j}
	\label{eq:taustrich}
	\end{equation}
	
	The optical density can now be calculated by using the difference of the column density $S_{M}$ in the measurement spectrum to the column density $S_{R}$ of a reference spectrum. From \Cref{eq:bb} we know:	
	\begin{equation}
	I_{P,R} = I^{'}_{0}\cdot exp\left(-S_{P,R}\cdot\sigma\left(\lambda\right)\right)
	\label{eq:smr}
	\end{equation}
	In general the obtained column density $S_{M}$ is called differential slant column density: "dSCD". If the reference spectrum does not contain the trace gas of interest (is not contaminated with trace gases) that means $S_{R} = 0$, $S_{M}$ is called the slant column	density (SCD). 
	With \Cref{eq:smr} the optical density can be derived by:
	\begin{equation}
	\tau\left(\lambda\right) = -ln\left(\frac{I_{M}}{I_{R}}\right) = \sigma\left(\lambda\right)\cdot\left(S_{M}-S_{R}\right)
	\end{equation}
	%
	
	
	\subsection{Technical implementation of the DOAS approach}
	\textcolor{red}{Hier könnte man schon mal expliziter aufzeigen, wie eine andere Temperatur T + delta T zu einem Fehler führt. Als Vorbereitung für deine späteren Untersuchungen von delta T}
	The theory explained above only describes the ideally situation. In real measurements more problems occur due to instrument limitations inelastic scattering causing the Ring effect and due to impacts of external parameters like temperature.\\
	In the following a short overview about these problems and their consequences for our retrieval is given. Further information can be found in \cite{lubcke2014optical}.\\
	\subsubsection*{Optical and spectral resolution of the spectrometer}
	The resolution of the spectrometer is finite, thus, the detector receives a spectrum $I^{*}\left(\lambda\right)$ which can be retrieved with a convolution of the incident spectrum $I\left(\lambda\right)$ with the instrument function H$\left(\lambda\right)$:
	\begin{equation}
	I^{*}\left(\lambda\right) = I\left(\lambda\right)*H\left(\lambda\right)=\int I\left(\lambda-\lambda{'}\right)\cdot H\left(\lambda-\lambda{'}\right)d\lambda{'}
	\end{equation} 
	For the evaluation all $\sigma_{j}$  of the trace gases of interest need to have the same spectral resolution as the instrument used for recording the spectra. In this work we will use high resolution cross sections and convolute them with the instrument function H:
	\begin{equation}
	\sigma{*}\left(\lambda\right) = \sigma\left(\lambda\right)*H\left(\lambda\right)
	\end{equation}
	\textcolor{red}{bessere Darstellung zu $\sigma{*}$}
	The instrument function H can be approximated by using a the spectral lines of an mercury lamp since the width of those lines is only a few pm, they could be treated as delta peaks when comparing it to the resolution of the spectrometers.
	
	\subsubsection*{Effects of the detector}
	The detector only has discrete pixels, therefore a wavelength interval is mapped to a pixel $i$.
	
	\begin{equation}
	I^{'}\left(i\right) = \int_{\lambda(i)}^{\lambda(i+1)}I^{*}\left(\lambda{'}d\right)d\lambda{'}
	\end{equation}
	For the retrieval the relationship between the detector channels and the wavelength of the spectrum need to be known.
	The wavelength to pixel mapping (WPM) for a detector with q channels can be calculated as:

	\begin{eqnarray}
	\lambda(i) = \sum_{k=0}^{q-1}\gamma_{k}\cdot i^{k}
	\end{eqnarray}
	Hereby, is $\gamma_{0}$ a shift of the spectrum and $\gamma_{1}$ is a squeeze (respectively stretch) of the spectrum.
	The wavelength to pixel mapping can be discovered by using a mercury lamp again and compare pixel-position with the well known wavelength of the individual HG-lines of the mercury lamp.\\
	The wavelength to pixel mapping depends on the instrument temperature as well as on the ambient pressure \citep{lubcke2014bro}.
	\subsubsection*{Ring effect}
	As mentioned above inelastic scattering causes the Ring effect (named after Grainger and Ring, 1962).
	The Ring effect is observable through a filling of the Fraunehofer lines in spectra of scattered solar radiation, (e.g. if the sunlight travels through the earth atmosphere). When compared to direct sunlight measurements (e.g. outside of the earth atmosphere).
	(\cite{bussemer1993ring},\cite{solomon1987interpretation}) identified rotational Raman scattering mainly of
	\ce{O2} and \ce{N2} in the atmosphere as the origin of the Ring effect.
	\cite{solomon1987interpretation} suggested to treat the Ring effect as a pseudo-absorber. 


	
	\part{Evaluation of the data of Tungurahua and Nevado Del Ruiz}
	
	
	\chapter{Network for Observation of Volcanic and Atmospheric Change \label{NOVAC}}

	%	
	

%% NOVAC
%% =============
%%

		\begin{figure}[h]
			\centering
			\includegraphics[width=0.8\linewidth]{Bilder/NOVAC2015}
			\caption{Global map of the volcanoes monitored by NOVAC. Used with friendly permission of Santiago Arellano.}
			\label{fig:novac2015}
		\end{figure}
		The Network for Observation of Volcanic and Atmospheric Change (NOVAC) is a network of instruments monitoring volcanoes over the whole world. 
		%
		NOVAC was installed to add a monitoring parameter for volcanic activity by installing automatised instruments measuring \ce{SO2} emissions during daytime.\\
		%
		NOVAC was originally an European funded research project from 2005 until 2010. The aim of NOVAC is to  establish  a  global  
		network  of  stations  for  the  quantitative  measurement  of  volcanic gas  emissions in particular \ce{SO2}. At the beginning, NOVAC encompassed observatories of 15 volcanoes in Africa, America and Europe, including some of the most active and strongest degassing volcanoes in the world. Although the EU-funding has stopped, the network has been constantly growing since it was founded. In 2018 more than 100 instruments are installed at over 40 volcanoes in more than 13 countries.
		\Cref{fig:novac2015} shows a map, with all volcanoes of the Network for Observation of Volcanic and Atmospheric Change.\\
		
		The great advantage of the data monitored in NOVAC is the fact that NOVAC provides continues gas emission data over many years. This ensures statistically meaningful results for the data evaluation.\\
		The instruments used in NOVAC are scanning UV-spectrometer named Mini Doas instruments. \\
		The  Mini DOAS  instrument  represents  a  major  breakthrough  in  volcanic  gas	monitoring as it is capable of real-time semi-continuous unattended measurement of the total emission fluxes of  \ce{SO2} and BrO from a volcano. Semi-continues in this case means that the measurement is only possible during daytime.\\
		%
		\begin{figure}
			\centering
		%	\subfigure[ ]
		 \includegraphics[width=1\textwidth]{Bilder/Simon/Bilder_Tung/NOVAC_Instrument}
		%	\subfigure[Different Scan geometrys of NOVAC-Instruments From \cite{galle2010network}]{\includegraphics[width=0.49\textwidth]{Bilder/Simon/Bilder_Tung/NOVAC_scan_geo}}
			\caption{schematic sketch of a NOVAC instrument. From \cite{galle2010network}}
		\end{figure}
		The  basic  Mini DOAS  system  consists  of  a  pointing  telescope  fiber-coupled  to  a  spectrograph.  
		Ultraviolet light from the sun, scattered from aerosols and molecules in the atmosphere, is collected by 
		means  of  a  telescope  with  a  quartz  lens  defining  a  field-of-view  of  12~mrad
		\citep{NOVACsite}. \\
		The spectrometers measure in the UV region in a wavelength range of 280 to 420~nm. In this range the differential structures of \ce{SO2} and BrO structures are dominant.\\
		The lack of temperature stabilization at the instruments used by NOVAC comes with a reduced precision of the data, but the huge amount of data produced by NOVAC compensates for this limitation.  
%		\textcolor{red}{naja und es kommt evben auch immer drauf an was fuer genauigkeiten man bracuht - wenn ich eine groessenordnung in der emissionsmessungen sehen will muss ich nicht auf ein promille genau messen,..}
		

		\section{Measurement routine}
		\begin{figure}[h]
			\hspace*{-0.8cm}
			\subfigure{
			\includegraphics[width=0.488\linewidth]{Bilder/Simon/Bilder_Tung/Map_Tungurahua2}}
		\subfigure{
		\includegraphics[width=0.542\linewidth]{Bilder/NdR_map}}
			\caption[Topographig Map of the Tungurahua Volcano ( from \cite{hidalgo2015so2}) and Nevado del Ruiz volcano]{Topographig Map of the Tungurahua Volcano (left, from \cite{hidalgo2015so2}) and Nevado del Ruiz volcano (right). The predominant plume direction is shaded in yellow.  The NOVAC stations are shown as red squares, the corresponding scanning geometry is sketched with black lines. The prevalent plume dispersion is taken from the wind rose shown in \citet{Windrose} the topographic map is taken from google maps.}
		\label{fig:maptungurahua2}
	\end{figure}
	
		The instruments are set up five to ten km downwind of the volcano. To cover most of the occurring wind directions two to five instruments are installed at each volcano. Ideally, the measurement plane is orthogonal to the plume, to get the best measurement results. In reality, the measurement plane might be rotated.\\
		
		For the calculations of gas data from the DOAS retrieval a spectrum of the plume (plume spectrum) and a scan without any volcanic trace gases (reference spectrum) is needed.  This is done without any knowledge of the plume location by scanning the whole sky. 
		The measurement routine starts with a spectrum in zenith direction: the pre-reference. The exposure time of the pre-reference will be used for the whole scan.
		Afterwards, the dark current spectrum is recorded for the correction of the dark and offset.\\
		Then the telescope turns automatically to the side, recording spectra at the elevation angle from -90$^{\circ}$ to 90$^{\circ}$ with steps of 3.6$^{\circ}$. \\
		The instruments records 53 spectra per scan, the pre-reference, the dark current spectrum and 51 spectra at different elevation angles.
		One whole measurement cycle from horizon to horizon takes 6 to 15 minutes depending on the current illumination conditions.
		
		
				\subsection*{Tungurahua \label{Tung}}
		\begin{table}
			\centering
			\begin{tabular}{|p{4cm}|p{3cm}|p{3cm}|p{3cm}|}
				%	\toprule
				Instrument	&D2J2140&I2J8546& I2J8548\\
				\toprule
				compass&90.0	&	30.0	&	360.0	\\
				latitude&-1.453584	&	-1.543492	&-1.404719	\\
				longitude&-78.513047	&-78.517179	&	-78.494777	\\
				altitude&2609.980	&	2782.352	&	2911.253	\\
				volcano&Tungurahua	&Tungurahua	&	Tungurahua	\\
				site&pillate	&	bayushig	&	huayrapata	\\
				observatory&igepn	&	igepn	&igepn	\\
				serial&D2J2140	&	I2J8546	&	I2J8548	\\
				spectrometer&S2000	&	S2000	&S2000	\\
				instrumenttype&gothenburg	&gothenburg	&gothenburg	\\
				version&2.1	&2.1	&	2.1	\\
				\bottomrule
			\end{tabular}
			\caption{Technical data of the instruments installed at the Tungurahua volcano.}
			\label{tab:TInstruments}
		\end{table}
		Tungurahua is a steep-sided andesitic-dacitic subduction zone volcano located in the Ecuadorian Andes (Lat: 1.467$^{\circ}$S; Long: 78.442$^{\circ}$W). 2014 Tungurahua was one of the most active volcanoes in southern America, since then the activity is decreasing. Tungurahua is 5023m high and is one of the defining volcanoes of the eastern volcanic rows in Ecuador \citep{hall1999tungurahua}.
		\\
		The modern volcano was formed by a sequentially construction of three major edifices on a basement of metamorphic rocks. It has a total diameter of 12 km. Tungurahua I was roughly located at the same place as today and was build up in the mid-Pleistocene, mainly by andesitic and dacitic lava flows as well as interbedded tephra. Tungurahua II was formed in the past 14,000 years as a result of the collapse of the initial edifice.\\
		3000 years ago the Tungurahua II edifice collapse. The collapse of Tungurahua II produced a large debris-avalanche deposit and a horseshoe-shaped caldera which is open to the west side. Tungurahua III the current glacier caped stratovolcano was constructed in this caldera \citep{GlobalVolcanismProgram}.\\
		The current ongoing long term eruption started in October 1999. The eruptive ohase was preceded by hydrothermal tremors between 1994 and 1997 \citep{samaniego2003peligros}.\\
		From September 1998 to July 1999 an increase of seismic activity like volcano tectonic earthquakes indicated the raising of magma. This eruptive phase was characterized by alternating periods of high and low volcanic activity. \\
		
		At Tungurahua three instruments described  in \cref{tab:TInstruments}, with data recorded in the time span from July in 2008 to August in 2009, are used in this thesis.
		\Cref{tab:TInstruments} shows the exact position, of the instruments, the altitude, and other specifications. \\
		
		\subsection*{Nevado del Ruiz}
		Nevado del Ruiz is a glacier-covered, subduction zone volcano which is located in the Central Cordillera of Colombia, 140 km west of Bogota
		(Lat: 4.892$^{\circ}$N; Long: 75.324$^{\circ}$W). Nevado De Ruiz has a hight of 5389 m and covers an area of more than 200 km$^2$.
		Three major edifices has been constructed since the beginning of the Pleistocene, consisting of andesitic and dacitic lavas and pyrolastics \citep{GlobalVolcanismProgram}. \\
		The current cone is build within the caldera of older edifice and consists of a cluster of lava domes. The crater on the summit with the name Arenas crater, has diameter of 1 km and a depth of 240 m. 
		The last big eruption was in 1985. This was South America's deadliest eruption.\\
		In this thesis the data of two NOVAC instruments (see \cref{tab:NVDRInstruments}) in the time from the end of 2009 to the end of 2011 are used. \\
		\\
		\begin{table}
			\centering
			\begin{tabular}{|p{4cm}|p{3cm}|p{3cm}|}
				%	\toprule
				Instrument	&D2J2200&D2J2201\\
				\toprule
				compass&115.0		&59.0		\\
				tilt&0.0		&0.0		\\
				latitude&4.900917		&4.876183		\\
				longitude&-75.335134		&-75.353408		\\
				altitude&4866.500		&4494.259		\\
				volcano&Nevado del Ruiz		&Nevado del Ruiz		\\
				site&bruma	&alfombrales\\
				observatory&ingeominas		&ingeominas		\\
				serial&D2J2200		&D2J2201		\\
				spectrometer&S2000		&S2000		\\	
				instrumenttype&gothenburg		&gothenburg		\\
				version&2.2		&2.2		\\
				softwareversion&1.82		&1.82		\\
				compiledate&Feb 19 2009		&Feb 19 2009		\\
				\bottomrule
			\end{tabular}
			\caption{Technical data of the instruments installed at the Nevado del Ruiz volcano.}
			\label{tab:NVDRInstruments}
		\end{table}
	
	\chapter{Evaluation routine\label{Chap:evalroutine}}
	

%% NOVAC
%% =============
%
In the following, I describe the technical implementation of the DOAS approach using the data of NOVAC instruments:\\
%
The first step is to correct each spectrum of the scan for dark current and offset using the dark spectrum.
The next task is to locate the spectra within and outside of the volcanic plume.
First a "pre-reference" (the spectrum recorded at an elevation angle of  0$^{\circ} $) is used to perform the evaluation of the scan spectra recorded at every elevation angle.
For every spectrum of the scan the \ce{SO2} differential slant column density (dSCD) with respect to the pre-reference is calculated using \Cref{eq:F} by the DOASIS fit routine.

The result is \ce{SO2} dSCDs as a function of the elevation angle. This way the elevation angle corresponding to the maximum and the minimum of the \ce{SO2} column density can be determined. The location of the \ce{SO2} maximum defines the location of the plume. The assumption is that the minimum of the \ce{SO2} curve corresponds to a region outside of the plume which is true in most cases. The background \ce{SO2} amount in the Earth's atmosphere around Tungurahua is usually negligible (see  \Cref{chap:so2}). \\
We use a gauss fit of the \ce{SO2}-elevation-angle-curve to define the plume region.
%
\begin{figure}
	\centering
	\includegraphics[width=1\linewidth]{Bilder/NOVAC_Eval}
	\caption{NOVAC evaluation: (a) Measurement at the volcano (b) Evaluation of the spectral data with the DOAS routine using the absorption cross sections of \ce{BrO}  and \ce{SO2}. (c) Finding the location of the plume and reference(taken from \cite{WarnachSimon}). With the so found plume and reference spectra, the BrO/\ce{SO2} can be calculated. }
	\label{fig:NOVAC_Eval}
\end{figure}
The sum over all plume spectra is taken, which are in the elevation angle interval of the gauss peak plus minus one sigma, to increase the photon statistic and to reduce the residuum. If the Gauss curve is too wide, what this means in specific is that more then 10 spectra are added within the gauss evaluation, The running mean is calculated and the 10 spectra with the highest \ce{SO2} amount are used for the retrieval. As reference, the sum of the 10 spectra with the lowest \ce{SO2} amount is used.\\
The absolute slant column densities (SCDs) of \ce{BrO}  and \ce{SO2} can now be calculated with the previously defined reference and plume spectrum.
In \Cref{fig:plumeref} (a) an example \ce{SO2} SCD as a function of the elevation angle is shown. The \ce{SO2} curve has a maximum at the position of the plume at an elevation angle of approximately $-30^{\circ}$ to $0^{\circ}$  and a reference region at an elevation angle of $40^{\circ}$ to $70^{\circ}$. \Cref{fig:plumeref} (b)  illustrates that  extrema of the \ce{BrO}  curve are not as distinct as it is the case for the \ce{SO2} curve.\\
Since the \ce{BrO} column density is much lower than the \ce{SO2} column density, and just lies slightly above the detection limit, the plume is hard to detect using the \ce{BrO} column density as it is shown in fig. \ref{fig:plumeref} (b). 
Therefore  \ce{BrO} is only in the plume location determined by using \ce{SO2} evaluated.\\
To avoid a distortion of the data as a consequence of relatively bad scans, only scans with a $\chi^2$ (BrO fit) below $10^{-3}$ ($\approx$    95 \% of all suitable scans) are considered.
In a further step, multiple reference and plume spectra of successive measurements are added to further increase the fit quality.
\Cref{fig:NOVAC_Eval} visualizes the different steps described above in the retrieval of the BrO/\ce{SO2} ratios.\\
\begin{figure}
	\subfigure[ ]{\includegraphics[width=0.53\textwidth]{Bilder/Simon/Bilder_Tung/SO2_Scan}}
	\subfigure[ ]{\includegraphics[width=0.47\textwidth]{Bilder/Simon/Bilder_Tung/Algorithm}}
	\caption{(a) \ce{SO2} SCD as a function of the elevation angle. The co-added plume region is marked with red diamonds, and the co added reference region with green stars. From \cite{WarnachSimon}. (b) Flow chart of the \ce{BrO}  and \ce{SO2} evaluation. From \cite{lubcke2014optical}.}
	\label{fig:algorithm}
\end{figure}
\Cref{fig:algorithm} (b) shows the routine of adding multiple spectra of consecutive measuring times. In the following the spectra resulting from the multi adding technique will be referred to as "Multi Add Spectra". The algorithm for co-adding is visualized in \Cref{fig:algorithm} (b) was invented by \citet{vogel2011volcanic} and \citet{lubcke2014bro}. At Tungurahua periods with significant degassing have a mean temporal resolution of 19 and up to 45 "Multi Add Spectra" per day (resulting from three instruments)\\

The mean standard deviation of BrO SCD is about $2.6\cdot10^{13} \frac{molec}{cm^{2}}$. Hereby the standard deviation is estimated as 2 times the DOAS fit error of the multi-scan BrO fit \citet{Stutz96} \\
%
The SO2    detection limit is in the order of $6\cdot10^{16}\frac{molec}{cm^{2}}$.\\
%
Taking the \ce{BrO}/\ce{SO2} molar ratios if the column densities are close to zero yields unpredictable and unrealistic results. 
Thus, spectra measured in a thin volcano plume need to be excluded.
This could be achieved by setting a \ce{BrO} or/and a \ce{SO2} threshold. A reasonable \ce{BrO} threshold needs to be at least in the order of the DOAS fit error. The BrO detection limit can be enhanced by a daily averaging, then one gets a detection limit of BrO$_{DT}=\frac{2.6\cdot10^{13}}{n}\cdot\frac{molec}{cm^{2}}$ (n: number of Multi Add Scans per day, the minimum amount is four): However most BrO SCDs are below the detection limit.
Rejecting all BrO SCDs below the detection limit leads to an drastical decreases of data points since only the high BrO SCDs will be maintained and thus to systematic elevated \ce{BrO}/\ce{SO2} ratios  \citep{lubcke2014bro}.\\
%
The other possibility is to set an \ce{SO2} threshold to ensure only measuring in strong degassing periods. In this thesis an \ce{SO2} threshold (plume limit) of $7\cdot 10^{17} \frac{\text{molec}}{\text{cm}^2}$ is used for the selection of spectra for the evaluation of the \ce{BrO}/\ce{SO2} ratio. \\
%
Within the data above a \ce{SO2} plume limit of $7\cdot 10^{17} \frac{\text{molec}}{\text{cm}^2}$ the maximum detection limit of the BrO/\ce{SO2} ratio is estimated as $(BrO/SO2)_{DT}    =    \frac{\ce{BrO}_{DT}}{SO2_{thres}}=\frac{4    }{\sqrt{n}}\cdot10^{-5}     \leq     1\cdot10^{-5}$.
A plume limit of $7\cdot 10^{17} \frac{\text{molec}}{\text{cm}^2}$ is a high threshold for the column density. However, this approach assures that only strongly significant gas amounts are accounted \citep{lubcke2014bro}. Choosing the \ce{SO2} threshold in this way is a compromise getween a low BrO/\ce{SO2} detection limit and a sufficient amount of data.\\
\begin{figure}
	\centering
	\subfigure{    \includegraphics[width=0.49\linewidth]{Bilder/tungpercentage_minSO2}}
	\subfigure{\includegraphics[width=0.49\linewidth]{Bilder/percentage_minSO2}}
	\caption[The ce{SO2} SCD  as a function of the decrease of the amount of daily means amount above the plume limit. Data of Tungurahua and Nevado del Ruiz.]{The ce{SO2} SCD  as a function of the decrease of the amount of daily means amount above the plume limit. The \ce{SO2} SCDs below the actual plume limit of 7$\cdot10^{17}\,\frac{\text{molec}}{\text{cm}^2}$ are marked with a yellow shade. Left: Data of three instruments at Tungurahua. Right: data of two instruments at Nevado del Ruiz.}
	\label{fig:percentageminso2}
\end{figure}
Increasing a plume limit leads to a decrease of usable data. The amount of usable  daily means as a function of the plume limit is shown in \Cref{fig:percentageminso2}. A plume limit of 7$\cdot10^{17}\frac{\text{molec}}{\text{cm}^2}$ leads to a ratio of usable data of approximately 10\%.


\section{Retrieval of absolute slant column densities\label{Chap:Cont}}

At the volcano the atmospheric background levels in \ce{SO2} are negligible, thus one can interpret the dSCDs as SCDS. However, this is not the case if any volcanic trace gases are contaminating the reference spectrum.
The conventional evaluation is based on the assumption, that the reference is free of volcanic gases. This assumption was checked by using a volcanic gas free a high-resolution solar atlas spectrum (see below) to evaluate the reference (\cite{lubcke2014optical}; \cite{salerno2009novel}). In some reference spectra, absorption structures of \ce{SO2}  different from zero are found. This can be a result of spectrographic problems or of a non-negligible amount of volcanic trace gases in the reference region. However, the result is an underestimation of the gas amount by the conventional NOVAC-evaluation.
In ca. 10\% of the data, the reference spectrum is found to be contaminated with volcanic trace gases.
This could happen if, for example, the volcanic plume of the day before extends over the whole scan area as a consequence of windless conditions.
In consequence, the reference is contaminated with volcanic trace gases. Thus, the gas amount is underestimated by the conventional NOVAC-evaluation: In \Cref{fig:contaminated} an example from April 2011 (Tungurahua) where the reference region is contaminated by volcanic trace gases is shown. The blue \ce{SO2} curve shows the calculations with the NOVAC-evaluation, but since there is still \ce{SO2} in the reference region, the assumption, that the \ce{SO2} amount could be set to zero in the reference region is wrong. The red curve shows the \ce{SO2} curve, retrieved with a solar atlas spectrum as reference, which lies significantly above the NOVAC -curve.\\
\\    
%
\begin{figure}
	\centering
	\includegraphics[width=0.7\linewidth]{Bilder/contaminated}
	\caption{Scan with a contaminated reference spectrum from April 2011. From \cite{WarnachSimon}}
	\label{fig:contaminated}
\end{figure}
\\
If the reference region for any reason is
contaminated by volcanic trace gases, there are two possibilities: excluding the contaminated data from the evaluation or the reference spectrum has to be
replaced by a volcanic-gas-free reference. Alternative spectra are a
theoretical solar atlas spectrum or a volcanic-gas-free reference
spectrum recorded by the same instrument at another time.\\
A further possibility is to assume, that contamination only occurs for \ce{SO2}, but not for BrO due to the smaller lifetimes of BrO, thus it is possible to use the Solar atlas spectrum for the \ce{SO2} evaluation, but the reference, recorded by the NOVAC-instrument at the same time for the BrO retrieval. Hereby the assumption, that BrO is not contaminated need to be proved. \\
%
\\
%
In the following, I will discuss the two alternative reference spectra.
%
\subsection*{Evaluation using a Solar Atlas spectrum \label{kuruz}}
An alternative for choosing the region with the lowest column density as reference region is to use a theoretical high-resolution solar atlas spectrum as reference \citep{chance2010improved}.
The use of a theoretical solar atlas spectrum as a reference which is completely volcanic-trace-gases-free was first proposed by \cite{salerno2009novel} and evolved by \citep{lubcke2014bro}.
The advantage of using a solar atlas spectrum as reference is, that it is sure that it is not affected by past or current volcanic gas emissions. Thus, it allows for a retrieval of the absolute trace gas SCDs in the volcanic gas plume. The disadvantage is, that using a solar atlas spectrum comes along with a drawback of precision: The spectral resolution of the theoretical solar atlas spectrum is much  higher than of the NOVAC instruments. Therefore the instrument functions would need to be perfectly modeled and added to the retrieval. This is not straightforward, because the instrumental line-shape varies over the wavelength region and is also mathematically often not perfectly described by a simple approach like Gauss, Lorentz,..etc.\\ 
The reduction of precision is acceptable for the
\ce{SO2} retrieval but not suitable for a \ce{BrO} retrieval because then most data would be below the detection limit.\\
%
\\
%
Possible contaminations can be checked
by a theoretical solar atlas spectrum to evaluate the \ce{SO2} amount in the reference.\\

%
\subsection*{Evaluation using a spectrum of the same instrument}
An alternative reference spectrum could be a volcanic-gas-free reference
spectrum recorded by the same instrument at a different time. When using such a reference several problems occur:\\
As described in \cref{NOVAC} the instruments used in NOVAC do not include features like temperature stabilization. Due to that, the measurements are not independent of external parameters. 
So it is necessary to choose a reference recorded at similar conditions with respect to meteorology and radiation as well as in the temporal proximity due to instrumental changes with time and ambient conditions. Ideally, the external conditions should be equal to the conditions at the time when the plume was recorded.\\
%
When performing the evaluation with the Solar Atlas Spectrum as reference, finding the instrument function occur to be a central challenge. If the instrument function for the solar atlas spectrum is found the function is typically used for a few years This could lead to higher errors due to a gradual worse matching instrument function.
Using the reference of the same instrument but recorded at another day, leads also to problems caused by different instrument functions, but compared to the calculated instrument function used for the evaluation with the solar atlas spectrum those differences in the instrument function could be smaller.
\\
In this work, I combine both options in order to
achieve both, enhanced accuracy but still maximum possible precision of
the \ce{SO2} and \ce{BrO} retrievals. So I use the solar atlas spectrum to check for 
contamination and a reference spectrum recorded in temporal proximity by the same instrument as reference.\\
\begin{figure}
	\centering
	\includegraphics[width=0.5\linewidth]{Bilder/ConProb}
	\caption[The possibility of contamination relatively to all not contaminated data recorded at the same time as a function of the daytime. The data are taken from the Nevado del Ruiz volcano.]{ The possibility of contamination relatively to all not contaminated data recorded at the same time as a function of the daytime. The probability is calculated by dividing the number of contaminated scans recorded at a specific daytime by the amount of not contaminated data recorded at the same daytime. The square root of the data amount is taken as the error, the visualized error is then calculated by using error propagation. The local time can be calculated from the daytime as UTC-5h. The data are taken from the Nevado del Ruiz volcano.}
	\label{fig:conprob}
\end{figure}
The probability of contamination changes with the daytime as can be seen in \cref{fig:conprob}. In the Morning the probability of contamination is much lower than at noon. Thus one can conclude that the conditions on the day favor the occurrence of contamination more than the conditions at night. The decrease of contamination in the evening can be a result of fewer data and is therefore not significant.

If contamination occurs it is possible to choose a new reference from a list of gas-free alternative references. In theory, for ideal instruments, all references should lead to
the same results for the gas retrievals. But instruments are imperfect (see Chapter
4) thus the reference needs to be chosen carefully in order to ensure reliable results.\\
%
\\

\subsection{Contamination of the plume}
As discussed above it might occur, that the reference is contaminated for example by the plume of the day before.
If that happens, the gas amount is underestimated by using a contaminated reference. But another possibility is, that the plume itself is also contaminated. This might be the case if the volcanic gas of the volcano is not taken away by the wind, but accumulates at the instrument. If this is the case, using another reference would lead to an overestimation of the column density of gases. With the data retrieved by the NOVAC instruments, it is very difficult to discover whether the plume is contaminated or not. \\
%
%Thus using a gas-free reference in order to get rid of the contamination problem is arguable because it does not consider contamination of the plume. Moreover, I do not consider the different lifetimes of \ce{SO2} and BrO. This leads to a faster depletion of BrO.\\
%
\begin{figure}
	\centering
	\includegraphics[width=0.7\linewidth]{Bilder/Contaminationplume}
	\caption{Visualization of the contamination of the plume. Due to a lack of wind the old plume sinks down an accumulates
		above the instrument. The light path trough the old plume is longer when recording the reference spectra (orange).}
	\label{fig:contaminationplume}
\end{figure}
\begin{figure}
	\centering
	\subfigure{\includegraphics[width=0.49\linewidth]{Bilder/Contaminationplumeinstrinplum}}
	\subfigure{\includegraphics[width=0.49\linewidth]{Bilder/Contaminationplumewideplume}}
	\caption[Visualization of possible scenarios for contamination]{Visualization of possible scenarios for contamination. Left: the plume sinks in a way that the instrument is within the plume, therefore, all elavation angles will contain volcanic trace gases, while the plume is not additionally contaminated. Right: the plume covers the hole sky, thus all elvation angles "see" volcanic trace gases. }
	\label{fig:contaminationplumewideplume}
\end{figure}
%
Plume contamination could occur in a scenario described by  \cref{fig:contaminationplume}.
This might be the case if the volcanic gas of the volcano is not taken away by the wind, but accumulates in the vicinity of the volcano and thus is seen by the instrument. In such a scenario, using a contamination free reference recorded at another time would lead to an overestimation of the gas column densities in the plume.\\
This is one possible occurrence of contamination. As it can be seen gas of the old plume affects the measurement of the reference and the plume. However, for this example, the influence on the measurement of the reference is much larger since the light path through the old plume (colored orange in \cref{fig:contaminationplume}) is longer for the reference than for the measurement of the volcanic plume. Thus, the gas amount is underestimated by not using a gas-free reference, but might be overestimated if a gas-free reference spectrum is used.
The real gas amount might be between the measured amount with and without using a reference measured at another time.\\
%The contamination setup could differ from \cref{fig:contaminationplume}. This would lead to different results. However, the reference region is in most cases at a larger elevation angle than the plume. Thus, the assumption that the light path through the old plume is in average longer for the reference, if both, reference and plume are contaminated.\\
%
With the data retrieved by the NOVAC instruments, it is very difficult or even impossible to discover whether the plume is contaminated or not. \\
\\
\\
\begin{figure}
	\subfigure{
		\includegraphics[width=0.5\linewidth]{Bilder/contaminationdependency_so2}}
	\subfigure{
		\includegraphics[width=0.5\linewidth]{Bilder/contaminationdependency_so2_Nevad}}
	\caption[The strength of contamination as function of the mean \ce{SO2} column density (daily mean) of the day before. Data from Tungurahua and Nevado Del Ruiz.]{The strength of contamination as function of the mean \ce{SO2} column density (daily mean) of the day before. The strength of contamination is defined as the difference in \ce{SO2} SCD  when evaluation with an alternative reference, or neglect the contamination. Left: data from Tungurahua. Right: data from Nevado Del Ruiz. }
	\label{fig:contaminationdependencyso2}
\end{figure}

If the contamination is a result of strong emissions of the day before, a dependency of the strength of contamination on the emission of the day before could appear. 
The mean \ce{SO2} SCDs of a day (daily mean) could be used as a proxy for the total emission. The strength of contamination can be calculated as the difference between the evaluation for \ce{SO2} with a contaminated reference recorded at the same time as the plume spectrum was recorded and using a gas-free reference. Such a plot is shown for Tungurahua and Nevado del Ruiz in \cref{fig:contaminationdependencyso2}. Even though both plots show a slight increase in contamination strength with the mean amount of \ce{SO2} of the day before, the increase is not significant. Another possible proxy for the \ce{SO2} emission would be the maximum \ce{SO2} SCD of the day before contamination occurs. However, taking the maximum does not lead to a more significant relation between the emission and the strength of contamination. \\
Thus the \ce{SO2} emission is, if it influences the possibility of contamination, not the only significant factor. To further examine the reasons for contamination the wind conditions should be studied in particular. 
%\textcolor{red}{hier waere es gut noch andere parameter zu testen - wie in Luebcke et al, zum beispile als funktion von Wind geschwindigkeit,..   Für NdR kann ich die Meteorologischen Daten ab 2014 beisteuern.}
\\
However, this thesis is built on the assumption, that the plume is free of additional contamination. %\textcolor{yellow}{Nicole hat hier noch fragen, was ?}. 
In the following, I discuss how to automatically determine an optimal reference from another scan.




	\chapter{\ce{BrO}  evaluation and its limitations}
	%\ce{BrO}  Evaluation and its limitations
This chapter discusses the evaluation of the \ce{BrO} Column densities, calculated from the spectra recorded by the spectroscopic instruments of NOVAC. The quality of the BrO retrieval is hereby defined by the BrO retrieval error.\\
\Cref{fig:allbroerrordistribution} shows the BrO "Multi Add" retrieval error distribution, which is centered around $1.1e^{13}$ to $1.4e^{13}$. The instruments of Nevado del Ruiz are blue  coloured, while the the instruments of Tungurahua are coloured in yellow to red.\\
\\
The evaluation of the data from NOVAC is parted into the evaluation of SO2 and the evaluation of BrO. While the retrieving of SO2 is relatively easy due to the high amount of SO2 in the plume (magnitude of \ce{SO2} at Tungurahua $\approx 1e^{18}$), the BrO evaluation is much more problematic (as it can be seen in \Cref{fig:plumeref}). The magnitude of BrO SCD is around $\approx 1e^{14}$. \\
%
\begin{figure}
	\subfigure{
	\includegraphics[width=0.49\linewidth]{E:/Masterarbeit/BrO_Error_Distribution/sametimeBrOErrorDistribution}}
\subfigure{
	\includegraphics[width=0.49\linewidth]{E:/Masterarbeit/BrO_Error_Distribution/AllBrOErrorDistribution}}
	\caption{BrO error distribution shown for all in this thesis considered Instrument. The left plot shows the BrO distribution for the "Same Time retrieval", the left plot shows the BrO error distribution for the evaluation with a reference from another time, while the temporal difference between plume and reference is not longer that two weeks. 
	The peaks for the single instrument can be found at: D2J2140\_0: 1.2e+13;
		 I2J8548\_0: 1.3e+13;
		I2J8546\_0: 1.4e+13;
		D2J2200\_0: 1.4e+13;
		D2J2201\_0: 1.1e+13}
	\label{fig:allbroerrordistribution}
\end{figure}
%
This results in a larger uncertainty of the \ce{BrO}  SCD. Most of the \ce{BrO}  data are below the detection limit of: \ce{BrO}$_{err}$/BrO$_{value}$<1/4. In comparison SCD's of \ce{SO2} in almost all cases (99.5\% of the data)  are above the detection limit. \\
%
Choosing a different reference than the reference measured at the same time as the plume results in 99\% of all data in an increasing of the absolute error. 
Thus an \ce{BrO} error which is smaller than the "Same Time Error" is often not  possible to retrieve. 
However, for a contaminated same-time-reference the relative error might decrease due to the underestimation of the gas amount. \\
Due to the large uncertainty of BrO relative to SO2 the optimization of the BrO error is of particular importance. Therefore, the reference is chosen with respect to the BrO error to maximize the quality of the BrO/SO2 ratio. \\
\\
The amount of gas free alternative references is around 1500 per year. To make an optimal choice, it is necessary to examine the conditions which influence the BrO retrieval.\\
Every spectrum is measured under certain conditions, these conditions are in general not the same when measuring the plume and the reference. References where the surrounding conditions e.g temperature or cloudiness are equivalent with the surrounding conditions of the  plume measuring lead to a small error.\\
In the following, we take a closer look at the dependence of the \ce{BrO} error on external parameters. 
%
\subsubsection*{Data used for the analysis}
For this analysis, every plume reference pair of the observed time span is used. Thus 1000 recorded "multi add" spectra result in $1000^2$ possible plume reference pairs and the corresponding differences in the external parameter and their associated \ce{BrO} error.


\section{\ce{BrO} Error dependence on external parameters \label{Chap:BROErr}}
The measurement and evaluation of the spectra monitored with NOVAC depends on the surrounding conditions like temperature or cloudiness \citep{lubcke2014optical}\\
Thus, the surrounding conditions need to be taken into account for choosing a new reference.\\
The better the surrounding conditions for the reference coincide with the conditions for the plume measurement, the lower is the \ce{BrO} error. \\
	\\
The surrounding conditions that are considered in this thesis are: 
	\begin{itemize}
		\item Temporal Difference between measuring the plume and the reference.
		\item Temperature, 
		\item Colorindex, 
		\item Exposure Time, 
		\item Elevation Angle, 
		\item Daytime 	
	\end{itemize}
The analysis of these external parameter are performed for spectra recorded at Tungurahua and Nevado del Ruiz. At Tungurahua three instruments with data recorded in the time span from July in 2008 to August in 2009 are used. Nevado del Ruiz contributes with two instruments in the time from the end of 2009 to the end of 2011.
%	
\subsection{Temporal Difference}
%
\begin{figure}
	\centering
	\includegraphics[width=1\linewidth]{Bilder/Simon/Bilder_Tung/Drift_Komplett_NEW}
	\caption{Wavelength shift over the time. The shift is shown for six NOVAC- instruments from Tungurahua. The red and yellow dots show the running mean about 20 days. Red line indicates a temperature independent long term polynomial. Source: \cite{WarnachSimon}}
	\label{fig:driftkomplettnew}
\end{figure}
%
\begin{figure}
	\subfigure[]{
		\includegraphics[width=0.5\linewidth]{Bilder/Datum}}
	\subfigure[]{
		\includegraphics[width=0.5\linewidth]{Bilder/Datum_100h}}	
	\caption{The \ce{BrO} error as a function of the temporal difference shown for the Pillate instrument from Tungurahua (2008-2009). (a) Temporal differences up to 400 days. (b) Temporal differences up to 120h. The periodical \ce{BrO}  error evolution indicates the impact of the daytime}
	\label{fig:dat}
\end{figure}
%
Due to instrument drifts the fit quality decreases with the time difference between recording the plume and the reference. This could be a result of a wavelength shift over time observed by \citet{WarnachSimon}. As a result of the observed wavelength shift, the instrument function of the plume, does not match with the instrument function of the reference, thus the retrieving errors increase. \citet{WarnachSimon} suggests that the drift is caused by a hysteresis effect. \Cref{fig:driftkomplettnew} shows the wavelength shift as a function of the time for six NOVAC instruments located at Tungurahua in the time between 2008 to 2014. In this thesis for the analysis data of Tungurahua between 2008 to the mid of 2009 are used. \Cref{fig:driftkomplettnew} shows a rather steep drift in this time interval. \citet{WarnachSimon} observed a decrease of the shift after initial negative drift after the first two years at Pillate. Thus, it could be that the temporal difference becomes less important for old instruments.\\
When using reference and plume spectra of the same time, these effects are cut out since the shift is equal for the plume and reference spectrum, For increasing temporal different between reference and plume measurement time the fit quality decreases and thus the \ce{BrO} error\\
To examine the effect of the temporal difference on the retrieved BrO error, for all reference-plume pairs the corresponding BrO error was calculated. Due to the large amount of reference plume pairs within one year, it takes more than a month to evaluate the corresponding BrO error for every possible reference-plume pair of one instrument. We did this for the  D2J2140\_0 instrument installed at Tungurahua:\\
In \Cref{fig:dat} the \ce{BrO} error as a function of the time difference between recording the plume and the reference is shown. The running mean is drawn with a black line. \\
In \Cref{fig:dat} (a) it can be seen that a large temporal differences results in an increase of the \ce{BrO}  error of more than 600\%. \ce{BrO} errors of such magnitudes are too large for our purposes. Therefore, it is useful to set a maximal temporal difference, to prevent too large BrO error and to reduce the calculation time.
%
In \Cref{fig:dat} (a) it can be seen that the evolution of the \ce{BrO}  error with the temporal difference is symmetric around zero. Thus it is not necessary to distinguish between positive or negative temporal differences.
%
\Cref{fig:dat} (b) shows the evolution of the \ce{BrO} error for a maximal absolute temporal difference of 120 hours. It is only possible to record data during daytime. This causes the lack of data in the night time. A periodic decrease of the \ce{BrO} error can be seen. This is a result of a decrease of the \ce{BrO} error when the surrounding conditions coincidence. In this case the daytime coincidence causes the \ce{BrO}  error decrease. This effect is analysed in detail in \Cref{chap:daytime}.\\

The maximal temporal difference should be large enough to ensure a sufficient amount of references to be able to pick a reference with similar conditions. However, the maximal temporal difference should be small enough to prevent too large BrO errors due to long term shifts.\\
\\
To evaluate the maximal time difference, for which we still get reliable results for every plume, where the "Same Time Reference" is contaminated, the alternative reference was chosen, which leads to the minimal BrO error.\\
%
In \Cref{fig:Hist} a histogram with the probability of picking the best reference as a function of the time difference is plotted. Obviously, the best results are achieved, if the day of measuring the plume is the same day as measuring the reference. A Gaussian fit is used to fit the data of the histogram We allow all time differences within two sigma area. Thus, the maximal time difference is 14 days\\
% histogram
\begin{figure}
	\centering
	\includegraphics[width=0.7\linewidth]{Bilder/Hist}
	\caption{Histogram showing the frequency of getting the best reference as function of the temporal difference between plume and reference measuring. A Gaussian-like distribution is retrieved. The red dotted line visualizes a Gaussian fit for the shown histogram.}
	\label{fig:Hist}
\end{figure}
% remaining possible reference spectra
\begin{table}[h]
	\begin{tabular}{|p{2cm}|p{2cm}|p{2cm}|p{2cm}|p{2cm}|p{2cm}|}
		%	\toprule
		Instrument	&D2J2140\_0&I2J8546\_0& I2J8548\_0&D2J2200\_0&D2J2201\_0\\
		\toprule
		Mean&84.6 &163.7 &217.1&284.0 &225.6 \\
		\midrule
		Std&
		35.8&% $\equiv$ 100\%&
		29.9&% $\equiv$ 100\%&
		64.8&% $\equiv$ 100\%&
		69.5&% $\equiv$ 100\%&
		41.2\\% $\equiv$ 100\%\\
		\midrule
		Min&
		8 &%$\equiv$ 100\%&
		113&% $\equiv$ 100\%&
		97 &%$\equiv$ 100\%&
		64&% $\equiv$ 100\% &
		63\\% $\equiv$ 100\%\\
		\midrule
		Max&
		169&% $\equiv$ 100\%&
		214&% $\equiv$ 100\%
		399&% $\equiv$ 100\%
		433&%  $\equiv$ 100\%
		297\\% $\equiv$ 100\% \\
		\bottomrule
	\end{tabular}
	\caption{Amount of possible references when restricting the time span between plume and reference to two weeks. Here in the ”Mean” and “Std” row for each  instrument the average restriction is shown with the corresponding standard deviation. The “Min” and “Max” rows show the extend of restriction in the extreme cases (minimum and maximum amount of available references / restriction ratio).}
	\label{Tab:refstime}
\end{table}	
By restricting the temporal difference to 14 days, the amount of possible gas free references decreases to an average of 195 alternative references per contaminated plume (see \cref{Tab:refstime}). Whereas none of the plumes do not have alternative references. The minimum amount of references is 8.\\
If a continuously evaluation is required, this means the spectra are evaluated directly after the recording, the number of suitable gas free references halves since only references recorded before the plume are available.\\
For the following analysis of the remaining external parameters all temporal differences are below 14 days.
\begin{figure}
	\centering
	\includegraphics[width=0.7\linewidth]{Bilder/DatallInstruments}
	\caption{The BrO measurement error as a function of the temporal difference in days between the reference and the plume is shown for each of the individual instruments at Tungurahua and Nevado del Ruiz. The instruments at Nevado del Ruiz are colored in blue, while the instruments at Tungurahua are colored in red red color tones.  To evaluate the plume spectra all reference spectra with a temporal distance of no longer than two weeks are used. An increase of the BrO error with the absolute difference in temperature is observable. This is quantified by a correlation between the BrO retrieval error and the absolute temporal difference. The plots reveal a symmetry around the axis with zero temperature difference.}
	\label{fig:datallinstruments}
\end{figure}




\subsection{Temperature}
% fig_curves CAPTION
\begin{figure}
	\centering
	\includegraphics[width=0.7\linewidth]{Bilder/DiffTempallInstruments}
	\caption{The BrO measurement error as a function of the difference of temperature between the reference and the plume is shown for each of the individual instruments at Tungurahua and Nevado del Ruiz. To evaluate the plume spectra all reference spectra with a temporal distance of no longer than two weeks are used. An increase of the BrO error with the absolute difference in temperature is observable. This is quantified by a correlation between the BrO retrieval error and the absolute difference in temperature. The plots reveal a symmetry around the axis with zero temperature difference.}
	\label{fig:difftemp}
\end{figure}
% individual text
The instrument design of the NOVAC instruments compromises between accuracy and robustness as explained in \Cref{NOVAC}. In particular, there are no internal thermal stabilizations installed as an attempt to reduce the instruments power consumption. This can influence the recorded spectra.\\	
Each pixel of the spectrometer, which is used for the DOAS experiment, collects photons of a certain wavelength range.\\
The calibration for the wavelength to pixel mapping (WMP) is commonly done with a mercury lamp or by the comparison with the high defined Kuruz spectrum.
As the WMP depends on the optical alignment of the spectrometer, which itself depends on the temperature, it is not constant.
Changes in the spectrometers temperature can cause changes in the instrument line function and shifts in the WMP (\citep{pinardi2007influence}). 
% short term wavelength
\begin{figure}		
	\subfigure[]{\includegraphics[width=0.49\textwidth]{Bilder/Simon/Bilder_Tung/D2J2140_Before}}
	\subfigure[]{\includegraphics[width=0.49\textwidth]{Bilder/Simon/Bilder_Tung/D2J2140_After}}
	\caption{Short term wavelength as a function of the instrument temperature for Pillate 1. The coloring of the scatter points indicate the temporal evolvement. (a) initial period prior to January 2010 (b) after 2010. Source: \cite{WarnachSimon}.}
	\label{fig:shorttermshift}
\end{figure}
Moreover, \cite{WarnachSimon} show that, short term shifts are related to the instrument temperature (see \Cref{fig:shorttermshift}).\\
The above discussed temperature dependence of the WMP causes a reduction of the fit quality with increasing instrument temperature difference between plume and reference. Thus the \ce{BrO} error also increases with the temperature difference.
Compared to the other external parameters the temperate difference has the largest impact on the \ce{BrO} error.\\
\\
% fig_curves
In \Cref{fig:difftemp} the \ce{BrO} error is plotted against the temperature difference between the plume and the reference spectrum. The blue dots show the mean \ce{BrO} error at the specific temperature difference, the standard deviation is illustrated with gray error bars. The mean \ce{BrO} deviation for the sametime evaluation is additionally marked with a red point. The plots reveal a symmetry around axis with zero temperature difference.
% tab_fit CAPTION
\begin{table}[h]
	\begin{tabular}{|p{2cm}|p{2cm}|p{2cm}|p{2cm}|p{2cm}|p{2cm}|}
		%	\toprule
		Instrument	&D2J2140\_0&I2J8546\_0& I2J8548\_0&D2J2200\_0&D2J2201\_0\\
		\toprule
		Slope&4.10e+12 &3.93e+12 &6.50e+12 &1.24e+13&8.17e+12 \\
		\midrule
		Correlation
		& 
		0.593& 
		0.681& 
		0.910& 
		0.886& 
		0.920\\
		\midrule
		Zero point&2.58e+13&2.23e+13&1.60e+13& 1.38e+13& 9.07e+12\\
		\midrule
		$\Delta T_{2}$&6.3&5.7&2.5&1.1&1.1\\
		\bottomrule
	\end{tabular}
	\label{tab:tempe}
	\caption{The BrO measurement error as a function of the difference of temperature between the reference and the plume is fitted with a first order polynomial for each of the individual instruments at Tungurahua and Nevado del Ruiz. This table shows the fitting parameters slope and zero point. Moreover, the correlation between the BrO error and the absolute temperature difference is shown. For the temperature difference this correlation with an average of $0.797$ is the highest compared to the other external parameters. In the $\Delta T_{2}$ row the temperature difference for which the error doubles compared to a temperature difference of zero is shown. This is already the case for a difference of $3.3^\circ C$}
\end{table}
% tab_fit
To quantify the dependency between the BrO error and the difference in temperature the data are fitted with a polynom of the first order. Because of the observed symmetry around zero the absolute temperature difference is used for the fit. The computed fitting parameters slope and zero point for each instrument are shown in \Cref{tab:tempe}. \\
The zero points at Tungurahua vary from 1.6$\cdot10^{13}$ to 2.58$\cdot10^{13}$. The variation at Nevado del Ruiz ranges from  9.07$\cdot10^{12}$ to 1.38$\cdot10^{13}$.\\
Also the correlation between the BrO error and the absolute temperature difference is shown. Here the correlation is calculated using the python library Numpy. The correlation ranges from 0.593 for the instrument D2J2140\_0 to  0.92 for D2J2201\_0 and exhibits a large variation between the instruments.\\
The $\Delta T_{2}$ row in the table shows the temperature difference for which the error doubles compared to a temperature difference of zero.
% tab_ratio CAPTION
\begin{table}
	\begin{tabular}{|p{2cm}|p{2cm}|p{2cm}|p{2cm}|p{2cm}|p{2cm}|}
		%	\toprule
		Instrument	&D2J2140\_0&I2J8546\_0& I2J8548\_0&D2J2200\_0&D2J2201\_0\\
		\toprule
		Mean&
		39.6/46,8\%&
		119.3/72,9\%
		&158.2/72,9\%
		&233.6/82,3\%
		&151.6/67,2\%\\
		\midrule
		Std&
		24.66/68,9\%&
		50.4/168,6\%&
		75.97/117,2\%&
		84.5/121,6\%&
		72.6/176,2\%\\
		\midrule
		Min&
		1/12,5\%
		&8/7,1\%&
		12/12,4\%&
		3/4,7\% &
		6/9,5\%\\
		\midrule
		Max
		&
		130	/76,9\%&
		213	/99,5\%&
		386/96,7\%&
		414	/95,6\% &
		296	/99,7\%\\
		\bottomrule
	\end{tabular}
	\caption{This table shows the absolute amount and the ratio of remaining references if restricting the temperature difference to the mean $\Delta T_{2}$ over all instruments ($Mean(\Delta T_{2}) = 3.3$). Here in the ”Mean” and “Std” row for each  instrument the average restriction is shown with the corresponding standard deviation. The “Min” and “Max” rows show the extend of restriction in the extreme cases (minimum and maximum amount of available references / restriction ratio).}
	\label{tab:decTemp}
\end{table}	
% tab_ratio
If restricting the temperature difference to the mean $\Delta T_{2}$ over all instruments ($Mean(\Delta T_{2}) = 3.3$) the amount of possible references decrease as shown in \Cref{tab:decTemp}. Excluding references with temperature differences above $Mean(\Delta T_{2}) = 3.3$ restricts the amount of potential references to $46.8\%$ for D2J2140\_0 to $82.3\%$ for $D2J2200\_0$.\\
The advantage of restricting the accepted temperature difference is a better control of the choice of the best reference. The disadvantage is that the amount of possible references decreases. Thus, it could occur that a reference is dismissed, which has a large temperature difference but is very similar in the remaining parameters.\\

\subsection{ Daytime \label{chap:daytime}}
	\begin{figure}
	\centering
	\includegraphics[width=0.7\linewidth]{Bilder/DiffDaytimeallInstruments}
	\caption{The BrO measurement error as a function of the difference of daytime between the reference and the plumeis shown for each of the individual instruments at Tungurahua and Nevado del Ruiz. To evaluate the plume spectra all reference spectra with a temporal distance of no longer than two weeks are used. An increase of the BrO error with the absolute difference in daytime is observable. This is quantified by a correlation between the BrO retrieval error and the absolute difference in daytime. The plots reveal a symmetry around axis with zero daytime difference. }
	\label{fig:diffdaytime}
\end{figure}
% individual text
During the day a lot of external parameters like temperature, solar altitude etc. change. In particular, the solar altitude could have an impact on the fit quality since the light path of the sun is much longer in the morning or evening compared to the noon. Therefore, the scattering effects and the Fraunehofer structures are different for both spectra.\\

% fig_curves
In \Cref{fig:diffdaytime} the \ce{BrO} error is plotted against the daytime difference between the plume and the reference spectrum. The plots are similar to the plots for the temperature: The blue dots show the mean \ce{BrO} error at the specific daytime difference, the standard deviation is illustrated with gray error bars. The mean \ce{BrO} deviation for the sametime evaluation is additionally marked with a red point. \\
As for the temperature the plots reveal a symmetry around the axis of zero daytime difference.\\
%
% tab_fit
To quantify the dependency between the BrO error and the difference in daytime the data are fitted with a polynom of the first order. Because of the observed symmetry around zero the absolute daytime difference is used for the fit. The computed fitting parameters slope and zero point for each instrument are shown in \Cref{tab:dtcalc}. \\
%
As it can be seen in \Cref{tab:dtcalc}, the zero points at Tungurahua vary from 3.28$\cdot10^{13}$ to 3.43$\cdot10^{13}$. The variation at Nevado del Ruiz ranges from  2.24$\cdot10^{13}$ to 4.01$\cdot10^{13}$. \\
The correlations ranges from 0.156 for the instrument I2J8546\_0 to  0.631 for D2J2201\_0 and exhibits a large variation between the instruments.\\
The $\Delta T_{2}$ row in the table shows the daytime difference for which the error doubles compared to a daytime difference of zero.
% tab_fit CAPTION
	\begin{table}[h]
	\begin{tabular}{|p{2cm}|p{2cm}|p{2cm}|p{2cm}|p{2cm}|p{2cm}|}
		%	\toprule
		Instrument	&D2J2140\_0&I2J8546\_0& I2J8548\_0&D2J2200\_0&D2J2201\_0\\
		\toprule
		Slope&5.07e+12&1.40e+12 &3.77e+12 &2.04e+13& 1.38e+13\\
		\midrule
		Correlation&
		0.340&
		0.156&
		0.310&
		0.538&
		0.631\\
		\midrule
		Zero point& 3.43e+13&3.39e+13&3.28e+13&  4.01e+13&  2.24e+13\\
		\midrule
		$\Delta DT_{2}$&6.8&24.2&8.7&1.9&1.62\\
		\bottomrule
	\end{tabular}
	\label{tab:dtcalc}
	\caption{The BrO measurement error as a function of the difference of daytime between the reference and the plumeis fitted with a first order polynomial for each of the individual instruments at Tungurahua and Nevado del Ruiz. This table shows the fitting parameters slope and zero point. Moreover, the correlation between the BrO error and the absolute daytime difference is shown. In the $\Delta T_{2}$ row the daytime difference for which the error doubles compared to a daytime difference of zero is shown.}
\end{table}

% tab_ratio CAPTION
	\begin{table}
	\begin{tabular}{|p{2cm}|p{2cm}|p{2cm}|p{2cm}|p{2cm}|p{2cm}|}
		%	\toprule
		Instrument	&D2J2140\_0&I2J8546\_0& I2J8548\_0&D2J2200\_0&D2J2201\_0\\
		\toprule
		Mean&
		71.97$\equiv$85,1\% &		147.35$\equiv$90,0\%&
		198.4$\equiv$91,4\%&		274.96$\equiv$96,8\%&
		205.8$\equiv$91,2\%\\
		\midrule
		Std&		31.87$\equiv$	89,0\%&31.98 $\equiv$	107,0\%&
		70.96 $\equiv$	109,5\%&		70.78 $\equiv$	101,8\%&
		50.08 $\equiv$	121,6\% \\
		\midrule
		Min&
		6 $\equiv$	75,0\%&		58 $\equiv$	51,3\%
		&91 $\equiv$	93,8\%		&54  $\equiv$	84,4\%
		&45$\equiv$	71,4\%\\
		\midrule
		Max&
		160$\equiv$	94,7\% &
		214 $\equiv$	100,0\% &
		399 $\equiv$	100,0\% &
		433  $\equiv$	100,0\% &
		297 $\equiv$	100,0\% \\
		\bottomrule
	\end{tabular}
	\caption{This table shows the absolute amount and the ratio of remaining references if restricting the daytime difference to the mean $\Delta DT_{2}$ over all instruments except I2J8546\_0 due to the large uncertainty at I2J8546\_0 ($Mean(\Delta DT_{2}) = 4.75h$). Here in the ”Mean” and “Std” row for each  instrument the average restriction is shown with the corresponding standard deviation. The “Min” and “Max” rows show the extend of restriction in the extreme cases (minimum and maximum amount of available references / restriction ratio).}
	\label{tab:daytimerest}
\end{table}	
% tab_ratio
If restricting the daytime difference to the mean $\Delta T_{2}$ over all instruments with significant impact of the daytime, ($Mean(\Delta DT_{2}) = 4.75h$) the amount of possible references decrease as shown in \Cref{tab:daytimerest}. $Mean(\Delta DT_{2}) = 4.75h$ was calculated without taking the I2J8546\_0 into account due to the low correlation of 0.156. Using the I2J8546\_0 instrument as well would lead to an  $Mean(\Delta DT_{2})$ of $8.6h$, thus, the restriction would not have any influence, since the maximal time difference is limited to the time where the sun is shining.\\ 
Excluding references with daytime differences above $4.75h$ restricts the amount of potential references to $85.1\%$ for D2J2140\_0 to $96.8\%$ for $ D2J2200\_0$. In extreme cases a restriction down to $51.3\%$ of the entire set of references can occur.
	
\subsection{ Colorindex}
	\begin{figure}
	\centering
	\includegraphics[width=0.7\linewidth]{Bilder/DiffColidxallInstruments}
	\caption{The BrO measurement error as a function of the difference of colorindex between the reference and the plume is shown for each of the individual instruments at Tungurahua and Nevado del Ruiz. To evaluate the plume spectra all reference spectra with a temporal distance of no longer than two weeks are used. An increase of the BrO error with the absolute difference in colorindex is observable. This is quantified by a correlation between the BrO retrieval error and the absolute difference in colorindex. The plots reveal a symmetry around axis with zero colorindex difference. }
	\label{fig:diffcolidx}
\end{figure}
% individual text
Clouds  have  a  strong  influence  on  the  atmospheric  radiative  transfer  and  thus  affect  the  interpretation  and  analysis of DOAS \citep{wagner2014cloud}.
Clouds can be identified by several measurement quantities that they influence.
As Mie scattering is dominant in clouds the wavelength of the light that is scattered is different than the Rayleigh sky. Thus, clouds can be easily identified by their white color.
Therefore, the cloudiness of the sky can be quantified in a scalar measure defined by the ratio of the measured intensity at two wavelengths, the so-called colour index.
\cite{wagner2014cloud} showed that for a zenith-looking instrument the measured radiation intensity is enhanced by clouds. Thus, clouds can cause large errors for the retrieved gas column density and the corresponding uncertainties. 
Cloud effects are especially severe if the cloudiness for the recorded plume and reference spectra strongly defer. Also for broken clouds the described effect can be observed as measurements at some elevation angles might be influenced by clouds while others are not.
In this work the Colour Index (CI) is the ratio between the intensities at 320nm and 360 nm.
These two wavelengths are as far apart as the filter used for stray-light prevention in the spectrometers allows.
%% I don’t understand 	
On the other hand, the lower wavelength avoids the deep UV range where \ce{SO2} and  \ce{O3} absorption plays a dominant role.
%% I don’t understand 	
The Mie scattering in the clouds is responsible for the higher amount of radiation from larger wavelengths. This results in a decrease of the CI \citep{lubcke2014optical}.\\
We evaluated the CI at the zenith, to increase the stability of the fit we added in each cases 10 intensity's. Using always the zenith to evaluate the colour index makes the colour index more comparable, but if broken clouds occur, the CI of the reference and the plume could differ from the calculated CI of the zenith. This could be a reason for the large deviations of the mean \ce{BrO} error as function of the colour index (see \Cref{fig:diffcolidx})\\
% fig_curves
In \Cref{fig:diffcolidx} the \ce{BrO} error is plotted against the colorindex difference between the plume and the reference spectrum. The plot was done similar to the plots for the temperature.
The plots reveal mostly a symmetry around the zero colorindex difference-axis. Thus the absolute colorindex can be used for the fitting which is done equivalently to the analysis of the temperature and the daytime. The computed fitting parameters slope and zero point for each instrument are shown in \cref{tab:colidxcalc}. \\
The zero points at Tungurahua vary from 3.36$\cdot10^{13}$ to 4.01$\cdot10^{13}$. The variation at Nevado del Ruiz ranges from  4.74$\cdot10^{13}$ to 7.21$\cdot10^{13}$.\\
The correlation is as well calculated and ranges from 0.2 for the instrument I2J8546\_0 to  0.409 for D2J2200\_0.\\
The $\Delta T_{2}$ row in the table shows the colorindex difference for which the error doubles compared to a colorindex difference of zero.
	\begin{table}[h]
	\begin{tabular}{|p{2cm}|p{2cm}|p{2cm}|p{2cm}|p{2cm}|p{2cm}|}
		%	\toprule
		Instrument	&D2J2140\_0&I2J8546\_0& I2J8548\_0&D2J2200\_0&D2J2201\_0\\
		\toprule
		Slope&2.30e+14 &7.92e+13 &1.17e+14 &5.42e+14&1.91e+14\\
		\midrule
		Correlation&
		0.267&
		0.200&
		0.319&
		0.409&
		0.359\\
		\midrule
		Zero point&4.01e+13&3.36e+13&3.47e+13& 7.21e+13& 4.74e+13\\
		\midrule
		$\Delta CI_{2}$&0.174&0.424&0.297&0.133&0.248\\
		\bottomrule
	\end{tabular}
	\caption{The BrO measurement error as a function of the difference of colorindex between the reference and the plume is fitted with a first order polynomial for each of the individual instruments at Tungurahua and Nevado del Ruiz. This table shows the fitting parameters slope and zero point. Moreover, the correlation between the BrO error and the absolute colorindex difference is shown. In the $\Delta T_{2}$ row the colorindex difference for which the error doubles compared to a colorindex difference of zero is shown.}
	\label{tab:colidxcalc}
\end{table}

% tab_fit CAPTION
	\begin{table}[h]
	\begin{tabular}{|p{2cm}|p{2cm}|p{2cm}|p{2cm}|p{2cm}|p{2cm}|}
		% 	\toprule
		Instrument	&D2J2140\_0&I2J8546\_0& I2J8548\_0&D2J2200\_0&D2J2201\_0\\
		\toprule
		Mean&
		84.6$\equiv$ 100,0\% &	163.7$\equiv$ 100,0\%&	215.6$\equiv$99,3\%&
		275.4$\equiv$97,0\% &219.4$\equiv$97,3\% \\
		\midrule
		Std&
		35.8$\equiv$100,0\% &	29.9$\equiv$	100,0\% &
		65.4$\equiv$	100,9\%&
		67.8$\equiv$	97,6\% &
		49.86$\equiv$	121,0\% \\
		\midrule
		Min&
		8$\equiv$	100,0\% &
		113$\equiv$	100,0\% 
		&97$\equiv$	100,0\% 
		&61 $\equiv$	95,3\% 
		&28	$\equiv$44,4\% \\
		\midrule
		Max
		&169$\equiv$	100,0\% 
		&214$\equiv$	100,0\% 
		&399$\equiv$	100,0\% 
		&421 $\equiv$	97,2\% 
		&297 $\equiv$	100,0\%  \\
		\bottomrule
	\end{tabular}
	\caption{This table shows the absolute amount and the ratio of remaining references if restricting the colorindex difference to the mean $\Delta CI_{2}$ over all instruments ($Mean(\Delta CI_{2}) = 0.2553.$). Here in the ”Mean” and “Std” row for each  instrument the average restriction is shown with the corresponding standard deviation. The “Min” and “Max” rows show the extend of restriction in the extreme cases (minimum and maximum amount of available references / restriction ratio).}
	\label{tab:colidxres}
\end{table}	
% tab_ratio CAPTION
% tab_ratio
If restricting the colorindex difference to the mean $Mean(\Delta CI_{2}) = 0.2553.$ over all instruments the amount of possible references decrease very little as can be seen in \Cref{tab:colidxres}.\\

\begin{table}
	\begin{tabular}{|p{2cm}|p{2cm}|p{2cm}|p{2cm}|p{2cm}|p{2cm}|}
		%	\toprule
		Instrument	&D2J2140\_0&I2J8546\_0& I2J8548\_0&D2J2200\_0&D2J2201\_0\\
		\toprule
		Mean&38.98&113.8&153.2&223.8&149.37\\
		\midrule
		Std&
		24.47&
		50.6&
		76.78&
		82.5&
		72.41\\
		\midrule
		Min&1&8&12&3 &6\\
		\midrule
		Max&130&213&386&399 &296\\
		\bottomrule
	\end{tabular}
	\caption{Amount of Possible references while restricting the difference in colorindex  between plume and reference to differences above 0.2553. maximal Time difference is 3.358$^{\circ}C$,maximal daytime diff is 4.75h without}
\end{table}	

\subsection{Elevation Angle}

% fig_curves CAPTION
\begin{figure}
	\centering
	\includegraphics[width=0.7\linewidth]{Bilder/DiffElevAngleallInstruments}
	\caption{The BrO measurement error as a function of the difference of elevation angle between the reference and the plume is shown for each of the individual instruments at Tungurahua and Nevado del Ruiz. To evaluate the plume spectra all reference spectra with a temporal distance of no longer than two weeks are used. The plots does not reveal a symmetry around axis with zero elevation angle difference for all instruments. The D2J2200\_0 instrument at Nevado del Ruiz is not symmetric around zero.}
	\label{fig:diffeleangle}
\end{figure}
% tab_fit CAPTION
\begin{table}[h]
	\begin{tabular}{|p{2cm}|p{2cm}|p{2cm}|p{2cm}|p{2cm}|p{2cm}|}
		%	\toprule
		Instrument	&D2J2140\_0&I2J8546\_0& I2J8548\_0&D2J2200\_0&D2J2201\_0\\
		\toprule
		Slope& 1.73e+8& 1.55e+10  &-9.00e+9 &2.92e+11&-3.96e+10\\
		\midrule
		Correlation&
		0.000&
		-0.010&
		0.012&
		0.065&
		-0.034\\
		\midrule
		Zero point&4.77e+13&4.23e+13&3.78e+13&8.37e+13 &6.44e+13 \\
		\bottomrule
	\end{tabular}
	\caption{The BrO measurement error as a function of the difference of elevation angle between the reference and the plumeis fitted with a first order polynomial for each of the individual instruments at Tungurahua and Nevado del Ruiz. This table shows the fitting parameters slope and zero point. Moreover, the correlation between the BrO error and the absolute elevation angle difference is shown. }
\end{table}
% same daytime / temperature plot 
\begin{figure}
	\subfigure{
		\includegraphics[width=0.49\linewidth]{Bilder/D2J2200_0_DiffElevAngle_onedaytime_Nevad}}
	\subfigure{
		\includegraphics[width=0.49\linewidth]{Bilder/D2J2200_0_DiffElevAngle_onetemp_Nevad}}
	\caption{The BrO measurement error as a function of the difference of elevation angle between the reference and the plumefor the D2J2200\_0 instrument. Left: restricted to a constant difference in daytime (difference of $\pm 1h$), right: restricted to a constant difference in temperature (difference of $\pm 1^\circ$C).}
	\label{fig:d2j22000diffelevangleonetempnevad}
\end{figure}


The elevation angle describes the angle between the horizon and the zenith. When using the plume spectrum and the reference spectrum of the same time, the difference in elevation angle cannot be zero, since the location of the plume does not coincidence with the location of the reference.\\
In \cref{fig:diffeleangle} the \ce{BrO} error is plotted as a function of the elevation angle. Obviously no significant correlation between the two parameters can be identified. This is not surprising because other than the previously discussed parameters we do not expect a more precise measurements for a difference of zero. 
Only the data of the D2J2200\_0 instrument significantly varys with the elevation angle. The observable variation of the BrO error with the elevation angle differs from the symmetric dependence of all other external parameter, the minimum BrO error can be found at a difference in elevation angle of -20$^{\circ}$. This curve is a result of the solar altitude over the day which can be obtained if only using data of the same day time. Such a plot can be seen in \cite{fig:d2j22000diffelevangleonetempnevad}.
	
Since the BrO error does not depend noticeable on the Elevation Angle no restriction on differences of the elevation angle are needed.
\subsection{Exposure Time}
\begin{figure}
	\centering
	\includegraphics[width=0.7\linewidth]{Bilder/DiffExpTimeallInstruments}
	\caption{The \ce{BrO}  measurement error as a function of the difference of exposure time between measuring the reference and the plume are shown. To evaluate the plume spectra all reference spectra with a temporal distance of no longer than two weeks are used. A increase of the \ce{BrO} error with the distance in exposure time is observably.}
	\label{fig:diffexptime}
\end{figure}
The Exposure Time is a degree of sky lightness. The  exposure time is the length of time the sensor of the NOVAC instrument is exposed to light. In one scan the exposure time is set constant to the exposure time of the first scan, the pre reference. The amount of light that reaches the film or image sensor is proportional to the exposure time. The exposure time is adjusted in the way that the maximum intensity does not overly the capacity of the sensor.\\
We can observe an small dependency of the \ce{BrO} error on the Exposure time at Tungurahua and Nevado del Ruiz as it is shown in \Cref{fig:diffexptime}
\begin{itemize}
	\item The \ce{BrO}  error as a function of the difference in Exposure Time is also symmetric around zero for all observed instruments, thus the absolute difference in the Exposure Time is sufficient for the evaluating.
	\item The instruments at Tungurahua does not show significantly dependence on the Exposure Time, even though there is always a minimum of the \ce{BrO} error at a difference of the Exposure Time of 0ms.
	\item Nevado del Ruiz shows a stronger correlation between the \ce{BrO}  error and the Exposure Time.
\end{itemize}
\begin{table}[h]
	\begin{tabular}{|p{2cm}|p{2cm}|p{2cm}|p{2cm}|p{2cm}|p{2cm}|}
		%	\toprule
		Instrument	&D2J2140\_0&I2J8546\_0& I2J8548\_0&D2J2200\_0&D2J2201\_0\\
		\toprule
		Slope& 5.54e+9&1.54e+10 &3.04e+10&1.72e+11&9.37e+10\\
		\midrule
		Correlation&0.067
		&0.121&
		0.251&
		0.452&
		0.434\\
		\midrule
		Zero point&4.63e+13&3.58e+13& 3.87e+13& 6.88e+13& 4.68e+13\\
		\midrule
		$\Delta T_{2}$&8357&662&1273&95&499\\
		\bottomrule
	\end{tabular}
	\label{tab:exptimecalc}
\end{table}

\begin{table}
	\begin{tabular}{|p{2cm}|p{2cm}|p{2cm}|p{2cm}|p{2cm}|p{2cm}|}
		%	\toprule
		Instrument	&D2J2140\_0&I2J8546\_0& I2J8548\_0&D2J2200\_0&D2J2201\_0\\
		\toprule
		Mean&
		81.67$\equiv$ 96,5\%		&162.79$\equiv$ 99,4\%		&212.78$\equiv$ 98,0\%		&283.99$\equiv$ 100,0\%		&225.59$\equiv$ 100,0\% \\
		\midrule
		Std&
		35.30 $\equiv$	98,6\%&		30.07 $\equiv$	100,6\%&
		64.47 $\equiv$	99,5\% &		69.5 $\equiv$	100,0\% &
		41.2$\equiv$	100,0\% \\
		\midrule
		Min  &
		8 $\equiv$	100,0\%\%&113$\equiv$	100,0
		&95 $\equiv$	97,9\%
		&64$\equiv$	100,0\%
		&63$\equiv$	100,0\%\\
		\midrule
		Max&
		167$\equiv$	98,8\% &
		214 $\equiv$	100,0\% &
		395 $\equiv$	99,0\% &
		433 $\equiv$	100,0\%  &
		297 $\equiv$	100,0\% \\
		\bottomrule
	\end{tabular}
	\caption{Amount of Possible references while restricting the difference in Exposure Time  between plume and reference to differences below 632.25ms}
	\label{tab:etrest}
\end{table}	







\begin{table}
	\begin{tabular}{|p{2cm}|p{2cm}|p{2cm}|p{2cm}|p{2cm}|p{2cm}|}
		%	\toprule
		Instrument	&D2J2140\_0&I2J8546\_0& I2J8548\_0&D2J2200\_0&D2J2201\_0\\
		\toprule
		Mean&
		36.0$\equiv$ 42,6\%&	112.9$\equiv$ 69,0\%&
		148.88$\equiv$ 68,6\%&	217.0$\equiv$ 76,4\%&	140.38$\equiv$ 62,2\%\\
		\midrule
		Std&
		22.35 $\equiv$	62,4\%&
		50.6 $\equiv$	169,2\% &
		75.9 $\equiv$	117,1\%&
		82.07 $\equiv$	118,1\% &
		71.0 $\equiv$	172,3\% \\
		\midrule
		Min&
		1 $\equiv$	    12,5\%  &
		8$\equiv$	7,1\%  &
		12 $\equiv$	12,4\%  &
		3$\equiv$	4,7\%   &
		6$\equiv$	9,5\%  \\
		\midrule
		Max
		&127$\equiv$	75,1\%
		&212$\equiv$	99,1\%
		&382$\equiv$	95,7\%
		&398 $\equiv$	91,9\%
		&283$\equiv$	95,3\%\\
		\bottomrule
	\end{tabular}
	\caption{Amount of Possible references while restricting the difference in colorindex  between plume and reference to differences above 0.2553. maximal Time difference is 3.358$^{\circ}C$,maximal daytime diff is 4.75h without Exposure Time  between plume and reference to differences below 632.25ms}
\end{table}	

\Cref{tab:exptimecalc} shows the slope, correlation, zero point and the $\Delta ET_{2}$s: th differences in Exposure time where the BrO error increases by a factor of two compared to the difference of exposure time of zero.
Restrictions of the exposure Time to the mean of the $\Delta ET_{2}$s of all instruments which is 632.25 ms leads to an average decrease compared to \cref{Tab:refstime} of data of \textcolor{red}{ hier den mittelwert}. The results for each instrument can be found in \cref{tab:etrest}.



	

	\subsection*{Dependency of external parameters on each other}
	In all discussions on the impact of the external parameter on the retrieved \ce{BrO}  error the  dependency of the external parameter on each other are neglected. It is plausible that the temperature correlates with the cloudiness or the lightness due to sunlight. Therefore the correlation of the exposure time with the \ce{BrO}  error could be a result of the correlation of the temperature with the \ce{BrO}  error. \Cref{fig:difference-in-exposure-time-msdifference-in-temperature-ctungu} shows an example of the dependency of external parameters on each other. The difference in temperature as a function of the difference in exposure time. The \ce{BrO}  error is color-coded. \\
	All correlations between the external parameters are shown in \Cref{fig:varcorrelationmatrix}. \Cref{fig:varcorrelationmatrix} shows discrete correlation values from 0.3 to 1. Correlations below 0.3 are ignored. Small plus and minus signs indicate whether the correlation is negative or positive. 
	The temperature depends on the daytime, due to the dependence of the temperature on the sun, thus a correlation between the  difference in temperature and the difference in daytime (Correlation of $\approx$ 0.5) can be observed. Since the temperature depends on the intensity of the sun, it also correlates with  the difference in exposure time (Correlation of $\approx$ 0.4). The difference in temperature also slightly correlates with the difference in colorindex, due to the dependency of temperature on the cloudiness (Correlation of $\approx$ 0.3). The low correlations could appear due to the uniform cloudiness near to the equator. The correlation between the temperature difference to the temporal difference (Correlation of $\approx$ 0.3) probably occurs due to long term changes in temperature. Furthermore the difference in exposure time correlates with the daytime and the colorindex (Correlation of $\approx$ 0.4) as a result of the dependency on the sun intensity.\\
	\begin{figure}[h]
		\centering
		\includegraphics[width=0.7\linewidth]{"../Ausarbeitung/Uebersicht/Bilder/Difference in Exposure Time [ms]_Difference in Temperature [C]_Tungu"}
		\caption{An example of the dependency of external parameters on each other. The difference in temperature as a function of the exposure time. data from Tungurahua.}
		\label{fig:difference-in-exposure-time-msdifference-in-temperature-ctungu}
	\end{figure}
	\begin{figure}[h]
		\centering
		\includegraphics[width=1\linewidth]{Bilder/varCorrelation_matrix}
		\caption{Correlation matrix of the external parameters. The correlation is discrete colour coded. Positive correlation is labeled with a plus whereas negative correlation is labeled with a minus. The correlation matrix is calculated using the data from D2J2140\_0.}
		\label{fig:varcorrelationmatrix}
	\end{figure}
%
	To eliminate the correlation between the external parameters the \ce{BrO}  error dependency on one external parameter where calculated by keeping the differences in the other external parameters constant. Hereby only parameters were kept constant, where the correlation is above 0.3. Thus, when looking at the temperature, only the difference in temperature need to be constant, since the temporal distance does not correlate wit the other considered external parameter. The results can be seen in \Cref{fig:datwithoutotherparamallinstruments} to \Cref{fig:diffdaytimewithoutotherparamallinstruments}.\\
	\\
	\Cref{fig:datwithoutotherparamallinstruments} shows the BrO retrieval error as a function of the temporal difference between the reference and the plume spectrum. All differences in temperature are below one degree. Compared to the correlations, calculated without eliminating the dependence on the temperature, the correlations increase. The dependence of the instruments installed at Tungurahua are still significant higher. From the results can be interpreted that the temporal difference between the time when measuring the plume and the reference has a impact on the fit quality, but this impact is smaller as the impact of the instrument temperature.\\
	\Cref{fig:diffexptimewithoutotherparamallinstruments} shows the dependency of the BrO retrieval error on the difference in exposure time for all considered instruments. Hereby, only data are used, where the difference in temperature is below 1 degree, the difference in colorindex is below 0.05 and the difference in daytime is below one hour. The temporal difference and the difference in elevation angle are not kept constant, since we could not observe a relation between the exposure time and the temporal difference or the elevation angle. The correlations between the BrO error and the difference in exposure time decrease for each instrument if the temperature, daytime and colorindex are kept constant. Even though, the correlations at Nevado Del Ruiz are still higher as the correlations at Tungurahua.\\
	\Cref{fig:diffcolidxwithoutotherparamallinstruments} shows the BrO retrieval error as a function of the difference in colorindex for all instruments. The temperature and the exposure time shows a dependency on the colorindex as it can be seen in \Cref{fig:varcorrelationmatrix}. Both are kept constant, the difference in temperature is below 1 degree and the difference in exposure time is below 100 ms. The correlations decrease compared to \Cref{fig:diffcolidx}. Especially the correlation of the D2J2201\_0 instrument decreases.\\
	\Cref{fig:diffdaytimewithoutotherparamallinstruments} shows the BrO retrieval error as a function of the difference in daytime for all considered instruments. As it can be seen in \cref{fig:varcorrelationmatrix} the exposure time and the temperature need to be kept constant. The difference in temperature is below 1 degree, the difference in exposure time belwo 100 ms. A general decrease of the correlations compared to \Cref{fig:diffdaytime} is observable.\\
	Restricting the data, to where the temperature difference is kept below one degree, leads to an distortion of the data. Thus the results plotted in \Cref{fig:datwithoutotherparamallinstruments} to \Cref{fig:diffdaytimewithoutotherparamallinstruments} could also have systematic errors.\\
	When comparing the correlations of the data from \Cref{fig:datwithoutotherparamallinstruments} and \Cref{fig:diffexptimewithoutotherparamallinstruments}  to the correlations of 
	\Cref{fig:datallinstruments} to \Cref{fig:diffexptime} a large reduction of the correlation is obvious. Only the difference in temperature still shows a significant correlation to the \ce{BrO}  error. However, the minimal \ce{BrO}  error coincidence in almost all cases with a difference in external parameters of zero. A dependency of the \ce{BrO}  error on the external parameter can still be seen even tough the correlation is very small. \\
	Excluding of the external parameters due to the rather low correlation leads to a worse quality of the results, since the effects of the single parameter add up to a not negligible amount. However, note that  the added impact of all external parameter except for the temperature are less important than the temperature.
	The BrO error of contaminated spectra evaluated with the contamination based method (See \Cref{chapt:contbased}) increases by 15\% if the other external parameters are excluded. If all external paramter except for the temperature are used the calculated BrO error increases by 37\%.\\
	\\
	For the final evaluation of contaminated data we use the results of \Cref{fig:difftemp} to \Cref{fig:diffcolidx}. Since the correlations between the external parameters are considered in the final 4 dimensional fit.\\
	\begin{figure}[h]
		\centering
		\includegraphics[width=0.7\linewidth]{Bilder/BrOErr_OhnEVar/DatwithoutOtherparamallInstruments}
		\caption{The BrO measurement error as a function of the temporal difference between measuring the reference and the plume are shown. Therefor, the reference spectra are restricted such that the maximal temperature difference between reference and plume is $1^\circ C$.}
		\label{fig:datwithoutotherparamallinstruments}
	\end{figure}
\begin{figure}[h]
	\centering
	\includegraphics[width=0.7\linewidth]{Bilder/BrOErr_OhnEVar/DiffExpTimewithoutOtherparamallInstruments}
	\caption{The BrO measurement error as a function of the exposure time difference between measuring the reference and the plume are shown. Therefor, the reference spectra are restricted such that the maximal color index difference between reference and plume is $0.05$. The maximal daytime difference is $1h$.}
	\label{fig:diffexptimewithoutotherparamallinstruments}
\end{figure}
\begin{figure}[h]
	\centering
	\includegraphics[width=0.7\linewidth]{Bilder/BrOErr_OhnEVar/DiffColidxwithoutOtherparamallInstruments}
	\caption{The BrO measurement error as a function of the color index difference between measuring the reference and the plume are shown. Therefor, the reference spectra are restricted such that the maximal temperature difference between reference and plume is $1^\circ C$. The maximal exposure time difference is $100 ms$.}
	\label{fig:diffcolidxwithoutotherparamallinstruments}
\end{figure}
\begin{figure}[h]
	\centering
	\includegraphics[width=0.7\linewidth]{Bilder/BrOErr_OhnEVar/DiffDaytimewithoutOtherparamallInstruments}
	\caption{The BrO measurement error as a function of the daytime difference between measuring the reference and the plume are shown. Therefor, the reference spectra are restricted such that the maximal temperature difference between reference and plume is $1^\circ C$. The maximal exposure time difference is $100 ms$.
	}
	\label{fig:diffdaytimewithoutotherparamallinstruments}
\end{figure}


	
%	\begin{figure}
%			\subfigure{\includegraphics[width=0.3\linewidth]{Bilder/BrOError_onVariables_050218/I2J8546_0DiffTemp_Tungu}}
%			\subfigure{\includegraphics[width=0.3\linewidth]{Bilder/BrOError_onVariables_050218/I2J8548_0DiffTemp_Tungu}}
%			\subfigure{\includegraphics[width=0.3\linewidth]{Bilder/BrOError_onVariables_050218/D2J2140_0DiffTemp_Tungu}}
%		\subfigure{\includegraphics[width=0.3\linewidth]{Bilder/BrOError_onVariables_050218/I2J8546_0DiffColidx_Tungu}}
%		\subfigure{\includegraphics[width=0.3\linewidth]{Bilder/BrOError_onVariables_050218/I2J8548_0DiffColidx_Tungu}}
%		\subfigure{\includegraphics[width=0.3\linewidth]{Bilder/BrOError_onVariables_050218/D2J2140_0DiffColidx_Tungu}}
%
%		\subfigure{\includegraphics[width=0.3\linewidth]{Bilder/BrOError_onVariables_050218/I2J8546_0DiffDaytime_Tungu}}
%		\subfigure{\includegraphics[width=0.3\linewidth]{Bilder/BrOError_onVariables_050218/I2J8548_0DiffDaytime_Tungu}}
%		\subfigure{\includegraphics[width=0.3\linewidth]{Bilder/BrOError_onVariables_050218/D2J2140_0DiffDaytime_Tungu}}
%
%		\subfigure[I2J8546\_0]{\includegraphics[width=0.3\linewidth]{Bilder/BrOError_onVariables_050218/I2J8546_0DiffExpTime_Tungu}}
%		\subfigure[I2J8548\_0]{\includegraphics[width=0.3\linewidth]{Bilder/BrOError_onVariables_050218/I2J8548_0DiffExpTime_Tungu}}
%		\subfigure[D2J2140\_0]{\includegraphics[width=0.3\linewidth]{Bilder/BrOError_onVariables_050218/D2J2140_0DiffExpTime_Tungu}}
%		\caption{DiffExpTime at Tungurahua}
%		\label{fig:DiffExpTime}
%	\end{figure}
%
%
%		\begin{figure}
%			\centering
%			\caption{\ce{BrO} error as a function of the external while measering the Plume and the Reference at Nevado del ruiz}
%			\subfigure{\includegraphics[width=0.3\linewidth]{Bilder/BrOError_onVariables_050218/D2J2201_0DiffTemp_Nevad}}
%			\subfigure{\includegraphics[width=0.3\linewidth]{Bilder/BrOError_onVariables_050218/D2J2200_0_DiffTemp_Nevad}}\\
%
%			\centering
%			\subfigure{\includegraphics[width=0.3\linewidth]{Bilder/BrOError_onVariables_050218/D2J2201_0DiffColidx_Nevad.pdf}}
%			\subfigure{\includegraphics[width=0.3\linewidth]{Bilder/BrOError_onVariables_050218/D2J2200_0_DiffColidx_Nevad.pdf}}
%
%			\subfigure{\includegraphics[width=0.3\linewidth]{Bilder/BrOError_onVariables_050218/D2J2201_0DiffDaytime_Nevad}}
%			\subfigure{\includegraphics[width=0.3\linewidth]{Bilder/BrOError_onVariables_050218/D2J2200_0_DiffDaytime_Nevad}}\\
%			\subfigure[D2J2201\_0]{\includegraphics[width=0.3\linewidth]{Bilder/BrOError_onVariables_050218/D2J2201_0DiffExpTime_Nevad}}
%			\subfigure[D2J2200\_0]{\includegraphics[width=0.3\linewidth]{Bilder/BrOError_onVariables_050218/D2J2200_0_DiffExpTime_Nevad}}
%			\label{fig:DiffExpTimeNevad}
%		\end{figure}


	\section{\ce{BrO}  dependence on external parameters\label{Chap:BrOdep}}
	The external parameters not only influence the fit quality but also the evaluation of the gas amount. A high different in certain external parameter could distort the calculated \ce{BrO}  column density. \Cref{fig:d2j2140060218difftemperature-cbro} shows the evaluation of one plume with respect to different references. The temporal difference between the references and the plume do not exceed two weeks. In theory we expect that the choice of  the reference should not make a difference, therefore all BrO column densities resulting from the evaluation should be equivalent. But we can see a high variation if choosing different references. The variability of the BrO column density depends as well on the external parameters, when looking at the temperature dependency a mean decrease of th BrO column density with an increasing temperature can be observed.
		\begin{figure}
		\subfigure[]{
			\includegraphics[width=0.5\linewidth]{"Bilder/OnePlumeMoreRefs_plumedate_081203_1646/D2J2140_0_6_02_18_DiffTemperature [C]_BrO"}}
		\subfigure[]{\includegraphics[width=0.5\linewidth]{"Bilder/OnePlumeMoreRefs_plumedate_081203_1646/D2J2140_0_plumeat12Temperature [C]_BrO_TemporalDiff"}}	
		\caption{\textcolor{red}{sehe ich das richtig, dass hier einer der Werte +-2 C hat und -4 Tage entfernt ist? Könnte dieser Wert genommen werden? Ich nehme an, später zeigst du noch Werte mit allen Parametern, wo das dann nicht mehr so ist}One plume is evaluated by using different references. The plume is recorded at Tungurahua volcano with the D2J2140\_0 instrument. The recording time was the  03.12.2008  at 16:46 o clock. The y axis shows the \ce{BrO}  column density difference between the NOVAC method and contamination based method. (a) The difference in \ce{BrO}  is plotted as a function of the temperature difference between the plume and the references. Every data point indicates one reference. (b) The difference in \ce{BrO}  is plotted as a function of the temporal difference between the one plume and the different references.}
		\label{fig:d2j2140060218difftemperature-cbro}		
	\end{figure}
	\Cref{fig:d2j2140060218difftemperature-cbro} is a result of an exemplary evaluation of one plume. An examination of all several plumes evaluated by different references is shown in \Cref{fig:difftempbroallinstruments}. The plots are equivalent to the plots for BrO error (for an example see \cref{fig:difftemp}). The BrO SCDs vary strongly with the differences in external parameters. It is observable, that the BrO SCDs recorded by the instruments at Nevado Del Ruiz vary in a larger range than for the BrO SCDs recorded by instruments at Tungurahua. This can also be observed for the BrO retrieval errors. The absolute BrO SCDs increase with the difference in external parameters.\\
	 (a) Temporal difference: For the temporal difference no significant correlation can be observed. (b) Difference in color index: The correlations differ for every instrument even for instruments of the same volcano. The correlations vary from -0.368 to 0.46. (c) Difference in daytime: For the instruments at Nevado Del Ruiz the BrO SCD correlates strongly with the difference in daytime. This can be a result of the dependence on the temperature, since the correlations are very similiar, only the correlations are a little bit stronger for the difference in temperature. (d) The strange dependence which can be seen between the BrO error and the difference in elevation angle for the D2J2200 instrumnt can be seen in this plot as well. The other instruments do not show any correlation between the BrO SCD and the difference in elevation angle.(e) The correlations for the difference in exposure time are opposite to the correlations in daytime. 
%	\Cref{fig:onerefd2j2140060218difftemperature-cbro} shows the exemplary evaluation of different plumes by on reference. Here we assume different column densities, but this variability should be independent of the difference in the external parameters. But \Cref{fig:onerefd2j2140060218difftemperature-cbro} shows a clear dependence on the Temperature. The correlation for this example is -0.76, thus it can be concluded, that here is also an significant dependence. 

%	\begin{figure}
%		\subfigure[]{
%		\includegraphics[width=0.5\linewidth]{"Bilder/OneRefMorePlumes_refdate_081115_1337/D2J2140_0_onerefTemperature [C]_BrO_TemporalDiff"}}
%		\subfigure[]{
%		\includegraphics[width=0.5\linewidth]{"Bilder/OneRefMorePlumes_refdate_081115_1337/onerefD2J2140_0_6_02_18_DiffTemperature [C]_BrO"}}
%		\caption{One Reference is used to evaluated different Plume spectra. The reference was recorded at Tungurahua volcano with the D2J2140\_0 instrument. recording time was the 081115 at 1337 o clock. The y axis show the \ce{BrO}  column density. (a) The \ce{BrO}  is plotted as a function of the temporal difference between the one reference and the plumes  (b) The \ce{BrO}  is plotted as a function of the temperature difference between the plume and the references. Every data point stands for another plume evaluated with the same reference.}
%		\label{fig:onerefd2j2140060218difftemperature-cbro}
%	\end{figure}
%	\begin{figure}
%		\includegraphics[width=0.7\linewidth]{"Bilder/OneRefMorePlumes_refdate_081115_1337/onerefD2J2140_0_6_02_18_DiffDaytime (Plume)_BrO"}
%		\caption{The difference of BrO when performing the evaluation with The NOVAC Method minus the contamination based method as a function of the Daytime when measuring the plume. The recording time of the reference was 081115 at 1337 o clock}
%		\end{figure}
%	\begin{figure}
%		\subfigure[]{
%		\includegraphics[width=0.5\linewidth]{"Bilder/BrODependencies/AllDataCorrWithBrOElevation Angle [deg]_BrO"}
%		}
%		\subfigure[]{
%		\includegraphics[width=0.5\linewidth]{"Bilder/BrODependencies/AllDataCorrWithBrOExposure Time [ms]_BrO"}
%		}
%		\subfigure[]{
%		\includegraphics[width=0.5\linewidth]{"Bilder/BrODependencies/AllDataCorrWithBrOTemperature [C]_BrO"}
%		}
%		\subfigure[]{
%		\includegraphics[width=0.5\linewidth]{Bilder/BrODependencies/AllDataCorrWithBrOColorindex_BrO}
%		}
%		\subfigure[]{
%		\includegraphics[width=0.5\linewidth]{"Bilder/BrODependencies/AllDataCorrWithBrODaytime (Plume)_BrO"}}
%		\caption{The Dependence of the BrO evaluation on external parameter is shown. }
%		\label{fig:alldatacorrwithbrodaytime-plumebro}
%	\end{figure}
\begin{figure}
	
	\subfigure{(a)
	\includegraphics[width=0.45\linewidth]{Bilder/BrOVariables/DatBrOallInstruments}
}
\subfigure{
	\includegraphics[width=0.45\linewidth]{Bilder/BrOVariables/DiffColidxBrOallInstruments}
}
\subfigure{
	\includegraphics[width=0.45\linewidth]{Bilder/BrOVariables/DiffDaytimeBrOallInstruments}
}
\subfigure{
	\includegraphics[width=0.45\linewidth]{Bilder/BrOVariables/DiffElevAngleBrOallInstruments}
}
\subfigure{
	\includegraphics[width=0.45\linewidth]{Bilder/BrOVariables/DiffExpTimeBrOallInstruments}
}
\subfigure{
	\includegraphics[width=0.45\linewidth]{Bilder/BrOVariables/DiffTempBrOallInstruments}}
\caption{\textcolor{red}{für welche Daten?}The BrO measurement value as a function of the difference of the external parameters between measuring the reference and the plume are shown.}
	\label{fig:difftempbroallinstruments}
\end{figure}
\FloatBarrier
	\subsection*{Summary}
This chapter examines the influence of external on the precision of the BrO evaluation. Herby the following parameter are considered: 
temporal difference, temperature, daytime, colorindex, elevation angle, exposure time.  The findings are based on the data of three instruments installed at the Tungurahua volcano and two instruments at the Nevado Del Ruiz volcano. The maximal temporal difference between measuring the plume and the reference is set to 14 days to prevent large uncertainties in the BrO evaluation. Due to the mechanical influence on the wavelength to pixel mapping the temperature for all analysed instruments has the most significant impact on the BrO evaluation for all considered external parameters. The elevation angle does not seem to influence the evaluation for all examined instruments thus the elevation is excluded from the evaluation. The influence of the other external parameter change at every instrument. So a relatively strong impact of the exposure time can be seen at the D2J2201\_0 instrument at Nevado Del Ruiz, while the exposure time does not seem to significantly influence the evaluation of the data from the  I2J8548\_0 at Tungurahua. 
	
	%--------------------------------------------------------------------------------------------------------------
\chapter{Contamination based method \label{chapt:contbased}}
\textcolor{red}{Soweit ich verstanden habe, haben die "Other Approaches" keine Relevanz für dein Ergebnis. Aber sie stören den Lesefluss. Ich würde sie in den Appendix schieben.}\\
\textcolor{red}{Wir brauchen einen anderen Namen dafür. Und man sollte hervorheben, dass das ein optionales Modul ist. Soll heißen, man kann es direkt in den Algorithmus einbauen und somit alles andere unverändert weiterverwenden. Erster Vorschlag für Namen: "Decontamination module"}
	Based on the findings about the influence of external parameters on the \ce{BrO} error we propose an algorithm which is able to pick an appropriate volcanic-trace-gas free reference. The algorithm uses the dependencies found in \Cref{Chap:BROErr} to find a sufficiently good matching reference. The input is a list of ambient conditions.\\ 
	The first step is, to evaluate every reference with solar atlas spectrum, to check for contamination. A reference is classified as contaminated if the fit against the solar atlas spectrum yields an SO2 SCD above the plume limit ($2\cdot 10^{17}$).\\
	\\
	In the second step we determine for each contaminated reference:
	\begin{figure}
		\centering
		\includegraphics[width=0.7\linewidth]{Bilder/Cont}
		\caption{Visualization of the contamination based Method. A list of possible references is available, where the temporal difference between the plume and the reference is not longer than two weeks. For every possible reference, the estimated BrO error is calculated by considering the corresponding difference in important external parameters. The reference with the so calculated minima BrO error is used for the evaluation.}
		\label{fig:Cont}
	\end{figure}

	\begin{itemize}
		\item We create a list of possible references where all references are not contaminated and the temporal distance to the plume date is no longer than 14 days.
		\item For all possible references we calculate the differences in the external parameters with respect to the corresponding plume spectrum.
		\item We use the analysis of external parameters as described in \Cref{Chap:BROErr} to estimate the \ce{BrO} error of all references
		\item We choose the reference with the smallest estimated \ce{BrO} error as new reference
		\item We evaluate the plume spectra of the corresponding scan with the new reference.
	\end{itemize}

	%
	$\epsilon_{0}$ is the \ce{BrO} error when evaluate the plume spectrum with the "same-time-reference".
	The assumption is, that the \ce{BrO} error $\epsilon_{BrO}$ can be described as the sum of $\epsilon_{0}$ and the deviation of $\epsilon_{BrO}$ with respect to all external parameters. It is limited by the accurateness of the NOVAC-instruments.
	\begin{align}
		\epsilon_{BrO} &=  \epsilon_{0}+\frac{d\epsilon}{dt}+\frac{d\epsilon}{d ^{\circ}}+\frac{d\epsilon}{dT}+\frac{d\epsilon}{dDt} +\frac{d\epsilon}{dc} + \mathcal{O}\left(OP\right) \\
		\rightarrow \Delta \epsilon_{BrO} &= \epsilon_{BrO} - \epsilon_{0} =\frac{d\epsilon}{dt}+\underbrace{\frac{d\epsilon}{d ^{\circ}}}_{=0}+\frac{d\epsilon}{dT}+\frac{d\epsilon}{ddt} +\frac{d\epsilon}{dc} + \mathcal{O}\left(OP\right) 
		\label{calc:err}
	\end{align}
	Here the parameter t stands for the time between plume-time and reference-time. The parameter T is the difference in temperature. The parameters Dt and c are the differences in the daytime and the colorindex. The term $\mathcal(OP)$ accounts for other excluded external parameters.
	The task occurring at this stage is to find the best representation for the deviations. An then find the reference which minimize $\Delta \epsilon_{BrO} $\\
	%
	\\
	%
	The easiest way is to just calculate the \ce{BrO} error of all possible references for every plume by using the DOASIS routine. If this method is used it is possible to choose the reference for which the \ce{BrO} error is minimal. However this takes to much computation time since the evaluation time would be proportional to the number of possible references because the evaluation needs to be done for $\mathcal{O}(n)$ with $n$ as the number of potential reference spectra. Doing this evaluation for every plume-reference pair makes it impossible to do the evaluation in real, or near real time.
	Furthermore taking the minimum in all cases is statistically risky since the good results can occur accidental. The interpretation is thus more easy if the algorithm searches for an appropriate reference in a defined parameter range.\\
	%
	In this thesis a novel approach of identifying an ideal reference spectrum, by considering externel parameters, is introduced. This way a much faster estimation with constant complexity $\mathcal{O}(1)$ is reached.
	But the above described optimal evaluation is used to rate new approach and compare them among each other. The optimal evaluation always choose the reference with the smallest absolute error. We don't use the relative error due to its vulnerability. Using the relative error leads to less precision since the references with the highest BrO column density is preferred.\\
	%
	The results of the algorithm which chooses the reference automatically are described relative to an optimal evaluation. If the relative error is larger than one \textcolor{red}{warum 1??} the data are excluded from the evaluation.\\
	\\
	In the following we examine several methods for choosing the best reference based on the analysis of external parameters. 
	
	\section{Fit data}
	\textcolor{red}{titel zu unspezifisch} \\
	The following chapter analyses fitting the data with a first order polynomial. \Cref{fig:difftemp} to \Cref{fig:diffexptime} show the \ce{BrO}  error as a function of external parameters. Hereby, the curves are symmetric around zero difference in the respective external parameter. Therefore, it is not necessary to distinguish between positive or negative deviations from the equal surrounding conditions. Thus, the absolute differences can be utilized.\\
	\\
	%
	A linear approximation of the \ce{BrO}  error as function of the considered external parameters leads to a variation of  \Cref{calc:err} :
	With linear differentiations of the \ce{BrO}  error with respect to the respective external parameters \Cref{calc:err} can be written as:	
	\begin{equation}
		\Delta \epsilon_{BrO} = a_{t}\cdot\Delta t+a_{ET}\cdot\Delta ET+a_{T}\cdot\Delta T+a_{dt}\cdot\Delta dt +a_{c}\cdot\Delta c + \mathcal{O}\left(OP\right)
		\label{calc:delterr}
	\end{equation}
	%
	To determine the coefficients $a_{x}$ (\Cref{calc:delterr}) the data visualized in \Cref{fig:difftemp}-\ref{fig:diffexptime} where used.  The fitting is done with an ordinary least square linear regression. In particular we used the python function Linear Regression from the library sklearn \citep{SKlearn}. \\
	
	As it can be seen in \Cref{Chap:BROErr} the impact of the different external parameters change for every instrument depending on the location and the instrument themselves. 
	While the \ce{BrO}  error does not show any dependence on some external parameters for some instruments, the error has very strong dependence on the same external parameter at an other instrument. An example is the correlation between \ce{BrO}  error and the difference in daytime of 0.6 for D2J2201\_0 (Nevado Del Ruiz) and a correlation of 0.16 for I2J8546\_0 at the Tungurahua volcano.
	To get a more stable algorithm less external parameter are preferable. Thus, we need to distinguish between the stability of the fit, which improves with less external parameters and quality of the fit, which improves with more external parameters. 
	A preferable strategy is, to find a solution which is valid for all instruments. Moreover, utilizing less external parameters saves computation time.
	One possibility is to use all external parameters where the correlation is above a certain value. Since we want to get a selection valid for all instruments there are two possibilities: The first one is that we decide by using the mean correlation of all instruments. The second option is to use the highest correlation.
	
	\begin{table}[h]
		\centering
		\begin{tabular}{|p{4cm}|p{3cm}|p{3cm}|}
			&  Mean -Correlation&  Highest   -Correlation\\
			\toprule
	%		Temporal Difference&0.798&	0.92\\
	%		\midrule
			Temperature &0.798&	0.92\\
			\midrule
			Colorindex &0.3108&	0.409\\
			\midrule
			Exposure time &0.265&	0.452\\
			\midrule
			Elevation Angle &0.02&	0.067\\
			\midrule
			Daytime &0.395&	0.631\\
			\bottomrule		
		\end{tabular}
	\label{tab:CorrEP}
	\caption{The correlation coefficients between the BrO measurement error and the different external parameters. As a correlation value both the average and the maximum correlation is given.}
	\end{table}

	\begin{figure}
		\subfigure[Nevado Del Ruiz]{
			\includegraphics[width=0.5\linewidth]{"Bilder/WelcheEP/Nevado Del Ruiz"}}
		\subfigure[Tungurahua]{
			\includegraphics[width=0.5\linewidth]{Bilder/WelcheEP/Tungurahua}}
		\caption{Deviation from the NOVAC-evaluation as a function of the selection of differences in external parameters which are used for the evaluation. Here the utilized combination of external parameters are plotted on the x-axis and the deviation on the y-axis. The ideal combination (lowest deviation) is marked with a red cross.}
		\label{fig:WelcheEP}
	\end{figure}
	To answer this question quantitatively for the fitting routine we evaluated data of Tungurahua and Nevado Del Ruiz with different combinations of the external parameter described in \Cref{Chap:BROErr}. Since we could not observe any correlation between the \ce{BrO} error and the elevation angle the external parameter elevation angle was neglected in this analysis. To rate the results for the single instruments (three at Tungurahua and two at Nevado Del Ruiz) the difference to the "NOVAC-evaluation" is used. Hereby the factor $X$, a quantity which describes the distinction between the NOVAC-Method and the contamination based method, serves as a indicator:
	\begin{equation}
	X = \frac{1}{n}\sum_{k}^{n} \frac{E_{ContBased, k}}{E_{novac,k}}
	\label{eq:mean}
	\end{equation}
	$n$ is the total amount of contaminated spectra, $Enovac$ is the relative \ce{BrO}  error, in the NOVAC-evaluation, $EContBased$ is the relative \ce{BrO}  error, in the contamination based-evaluation. 
	\Cref{fig:WelcheEP} shows the $X$ factor for the Tungurahua and the Nevado Del Ruiz volcano. The y-axis shows the factors $X$ averaged over all instrument at the volcanoes.\\
	The factors X change from instrument to instrument. The results for every instrument can be seen in the appendix (\Cref{fig:I2J8548}). The factors X range from 1.18 to 1.42.\\

%	\textcolor{red}{If using the median and not the mean as described in \Cref{eq:mean} our method is mostly better than the "optimal method".--> wenn du schon sagst, dass es mit dem median besser wird, dann schreib auch den Median Wert hin und better than optimal??? besser machen.}
	As it can be seen in \Cref{fig:WelcheEP}  for both volcanoes the $X$ factor is minimal for the combination of the following external parameters:\\
	
	\begin{table}[h!]
			\begin{tabular}{cccc}
		$\bullet$ Temperature & $\bullet$ Daytime&  $\bullet$ Colorindex & $\bullet$ Temporal Difference\\
		\label{tab:importantexternalParam}
		\end{tabular}
	\end{table}
%	

	For the final algorithm this combination of external parameters is used.\\	
	The coefficients $a_{x}$ are calculated for each instrument at Nevado Del Ruiz and Tungurahua. Furthermore, the coefficients $a_{x}$ are calculated with the combined data from all instruments installed at one volcano. The results for the Nevado Del Ruiz volcano can be found in \Cref{tab:coefNevad} and for the Tungurahua volcano in \Cref{tab:coefTung}.\\
	\\
%	
	\begin{table}
		\subfigure{
			\begin{tabular}{c|c|c}
				\toprule
				Constant\\
				\toprule
				$a_{T}$\\
				\midrule
				$a_{CI}$\\
				\midrule
				$a_{t}$\\
				\midrule
				$a_{dt}$\\
				\bottomrule
		\end{tabular}}
		\subfigure[Data of Nevado Del Riz D2J2201\_0]{
			\begin{tabular}{c|c}
				\toprule
				value & import\\
				\toprule
				 7.505$\cdot$ e+12&0.866\\
				\midrule
				3.482$\cdot$ e+13&0.048\\
				\midrule
				-3.2$\cdot$ e+09& 0.0\\
				\midrule
				1.869$\cdot$ e+12&0.095\\
				\bottomrule
		\end{tabular}}		
	\subfigure[Data of Nevado Del Riz D2J2200\_0]{
		\begin{tabular}{c|c|c}
			\toprule
			value & import \\
			\toprule
		 	 1.179$\cdot$ e+13&0.925 \\
			\midrule
			1.024$\cdot$ e+14& 0.046\\
			\midrule
			-1.6$\cdot$ e+09& 0.0\\
			\midrule
			1.054$\cdot$ e+12&0.033\\
			\bottomrule
	\end{tabular}}
	\subfigure[Data of Nevado Del Both Instruments]{
		\begin{tabular}{c|c|c}
			\toprule
			value & import \\
			\toprule
			1.07$\cdot$ e+13&0.973 \\
			\midrule
			3.48$\cdot$ e+10&  0.070\\
			\midrule
			-9.1$\cdot$ e+08& 0.0\\
			\midrule
			 1.52$\cdot$ e+11&0.006\\
			\midrule
			 -6.81$\cdot$ e+13& -0.047\\
			\bottomrule
	\end{tabular}}
	\label{tab:coefNevad}
	\caption{\textcolor{red}{import: du musst erklaeren was das bedeutet, insbesondere mathematisch/statistisch}
		\textcolor{red}{Delta a\_T wird in 1C-Schritten gemessen (oder?!) aber in was werden die anderen Deltas gemessen, das hat doch sicher einen Einfluss auf diese Werte hier. Gibt es eine Moeglichkeit das zu normieren?}
		\textcolor{red}{vielleicht kann man die "importance besser durch ein Saeulendiagramm veranschaulichen, mit allen 7 Instrumenten in einem Bild} The results of the fitting with a first order polynomial. 
		The constants of \Cref{calc:delterr} are calculated.
		The value shows the actual number. The importance,  referred to as
		import, indicates the relative impact on the evaluation.
		(a)Data from Nevado Del Ruiz from the D2J2201\_0 instrument. 
		%$\epsilon_{0} = =  5.404e+12$
			(b)Data from Nevado Del Ruiz from the D2J2200\_0 instrument.  %$\epsilon_{0} = =  1.105e+13$ (c) 
			Data from Nevado Del Ruiz from both instrument. 
			% $\epsilon_{0} = 1.260e+13$}
		}	
	\end{table}	
	\begin{table}[h!]
	\subfigure{
		\begin{tabular}{c|c|c}
			\toprule
			Constant\\
			\toprule
			$a_{T}$\\
			\midrule
			$a_{CI}$\\
			\midrule
			$a_{t}$\\
			\midrule
			$a_{dt}$\\
			\midrule
			\bottomrule
	\end{tabular}}
	\subfigure[Data of Tungurahua D2J2140\_0]{
		\begin{tabular}{c|c}
			\toprule
			value & import\\
			\toprule
			 3.732$\cdot$ e+12&0.664\\
			 \midrule
			 6.982$\cdot$ e+13&0.078\\
		     \midrule
		     3.5$\cdot$ e+10&0.2 \\
			 \midrule
			 7.797$\cdot$ e+11&0.069\\
			 \midrule
			 \bottomrule
	\end{tabular}}
	\subfigure[Data of Tungurahua I2J8548\_0]{
		\begin{tabular}{c|c|c}
			\toprule
				value & import \\
				\toprule
				6.459$\cdot$ e+12&0.920\\
				\midrule
				2.760$\cdot$ e+13&0.066\\
				\midrule
				1.2$\cdot$ e+10 &0.13 \\
				\midrule
				-6.049$\cdot$ e+11&-0.055\\
				\midrule
				\bottomrule
	\end{tabular}}
	\subfigure[Data of Tungurahua I2J8546\_0]{
		\begin{tabular}{c|c|c}
			\toprule
			value & import \\
			\toprule
			3.939$\cdot$ e+12&0.850\\
			\midrule
			-1.521$\cdot$ e+13&-0.042\\
			\midrule
			2.0$\cdot$ e+10&0.2 \\
			\midrule
			1.158$\cdot$ e+11&0.017\\
			\midrule
			\bottomrule
	\end{tabular}}
	\subfigure[Data of Tungurahua I2J8546\_0]{
	\begin{tabular}{c|c|c}
			\toprule
			value & import \\
			\toprule
			5.055$\cdot$ e+12&0.838\\	
			\midrule		
			2.534$\cdot$ e+13&0.057\\
			\midrule
			1.8$\cdot$ e+10& 0.117\\
			\midrule
			-1.109$\cdot$ e+11&-0.012\\
			\midrule
			\bottomrule
	\end{tabular}}
		\label{tab:coefTung}
		\caption{The results of the fitting with a first order polynomial. The value shows the actual number. The importance,  referred to as
			import,  indicates the relative impact on the evaluation.
			The constants of \Cref{calc:delterr} are calculated.
			(a)Data from Tungurahua  from the instrument. 
			(b)Data from Tungurahua from the  instrument. %$\epsilon_{0} = 1.105e+13$ 
			(c) Data from Tungurahua averaged over all instrument.  %$\epsilon_{0} = 1.260e+13$,
			 both instruments Tungurahua% zero point  =  1.610e+13 \textcolor{red}{besser}
		 }
	\end{table}			
%	As can be concluded from \Cref{fig:WelcheEP} the best results are found if all parameters are used with a mean correlation above 0.3 (see \Cref{tab:CorrEP}).\\
	\\
	As it is shown in \Cref{Chap:BrOdep}, not only the BrO retrieval error depends on the difference in external parameter but also the BrO SCD (see \Cref{fig:difftempbroallinstruments}). 
	The question is, if the correlation between the BrO error and an considered external parameter as can be seen in \Cref{fig:difftempbroallinstruments}, can also be found in the BrO SCD and the corresponding differences in external paramter of the final choice of the reference, resulting from the fit routine. The answer to this question is No, since the maximal correlation is below 0.01. On Comparison, the maximal correlation if all references are used is above 0.8. Thus we can conclude, that the BrO SCDs resulting from the fit routine do not depend largely on the differences in external parameter.\\
	
	The fitting routine provides a list, where the possible references are sorted by there probable compatibility with the plume spectra. To quantize the influence of the choice of the reference on the BrO SCD, the BrO SCDs, retrieved by using the first 10 spectra of the list provided by the fitting routine are compared.
	For Tungurahua the mean standard deviation between the different BrO SCDs calculated with the different references is 28.7 \% of the mean BrO SCD. The mean BrO retrieval error is 43.8\% of the BrO SCD.
	At Nevado del Ruiz the standard deviation of the BrO SCD is 25.4\% of the mean BrO SCD. The mean BrO retrieval error is 36.9\% of the BrO SCD.
	
		
%	\\
%	\textcolor{red}{The deviations between the single evalutions at Tungurahua are:\\
%		std/mean (einzeln) 0.896; (median): 0.287\\
%		mean:  4.10e+13; (median): 4.14e+13\\
%		Std: 1.64e+13 ; (median):1.41e+13\\
%		abs(mean) 4.74e+13 ; (median):4.17e+13\\
%		std/mean (alles) 0.347 ; (median):0.298\\
%		\\
%		At Nevado Del ruiz we get:\\
%		std/mean (einzeln) 1.122; (median): 0.254\\
%		mean:  6.36e+13 ; (median):5.87e+13\\
%		Std: 2.14e+13 ; (median):1.62e+13\\
%		abs(mean) 6.87e+13; (median): 5.99e+13\\
%		std/mean (alles) 0.312; (median): 0.236\\
%		\\}
%	
%mit allen
%Tungu:
%
%std/mean (einzeln) 0.605378273729 0.298527797774
%mean:  3.98967133878e+13 4.05154e+13
%Std: 1.8013442975e+13 1.53889261173e+13
%abs(mean) 4.72802139124e+13 4.09881e+13
%std/mean (alles) 0.380993263024 0.325483428349
%
%
%NDR
%std/mean (einzeln) 8.88210192704 1.88372452094
%mean:  -1.1945882144e+14 -1.11303390977e+14
%Std: 2.18819723274e+14 2.27473806618e+14
%abs(mean) 1.34119574466e+14 1.146775e+14
%std/mean (alles) 1.63152712157 1.6960522543	
	\section{Other approaches}

	Fitting is not the only possibility of finding the optimal reference out of the list of all possible references.\\
	In the following two additionally possibilities are presented. Both are based on the findings in \cref{Chap:BROErr}. 

\subsection{Nearest neighbor approach}


Beside linear regression also the nearest neighbor approach can be utilized to estimate the BrO error for the evaluation with a potential reference spectrum.
%
The nearest neighbor search \textcolor{red}{referenzieren} describes an optimization problem for a given point $m \in \mathbb{R}^n$ and a set $S \subset \mathbb{R}^n$:

%

\begin{align}
\bar{s}(m) = \min_{s \in S} d(m, s) \label{eq:1nn}
\end{align}

%
Here $d(\cdot, \cdot)$ is a distance function that computes the dissimilarity between the two input arguments. Typical distance metrics are the L1 distance $d_{\text{L1} 	
}(m, s) = ||m - s||^2$ or the L2 (also euclidean) distance function $d_{\text{L2} 
}(m, s) = ||m - s||$. 


In many cases not only one nearest neighbor but a set $M$ of $k$ nearest neighbors is of interest. In this case the optimization problem of Eq. \ref{eq:1nn} must be modified to

%

\begin{align}
\bar{S_k}(m) = \min_{S_k \subset S} \sum_{s \in S_k, m \in M} d(s, m) \label{eq:knn}
\end{align}

%

In many cases the nearest neighbor search is used to estimate a target variable $y_m$ for a feature vector $m \in \mathbb{R}^n$. This method assumes a given set feature vectors $S$ for which the target variables $y_S$ are known. Then the target variable $y_m$ for a given $m$ can be estimated by:

%

\begin{align}
y_m = \frac{1}{k} \sum y_S \label{eq:knn_regression}
\end{align}

%

\subsection*{Advantages of the nearest neighbour approach}
	The main advantage of the nearest neighbour method is that there is no need for a pre-assumption of a fitting method. The fitting method assumes that the BrO error depends linearly  on the external parameter. As it can be seen in \Cref{fig:difftemp}-\Cref{fig:diffexptime} this does not need to be the case. Thus, the nearest neighbour method is able to approximate the BrO error as a function of the external data in a more variable way\textcolor{red}{more general}.
%
\subsection*{Disadvantages of the nearest neighbour approach}
	Compared to fitting the data, the nearest neighbour is much slower. The reason for this is, the vast amount of compassion operations that is necessary for each computation. Thus, the calculation time increases with the amount of training data.\\
	To normalize the distances d, the normalization function needs to be chosen and therefore pre-assumptions about feature importance are required.
	The X factor increases by 80\%, from 1.24 for the fitting method 
	to 2.24 for the nearest neighbour method.
	Including the fact, that the results of the nearest neighbor method is worse compared to the fitting method, the disadvantages of the nearest neighbour method outweigh the advantages.

	\subsection{Iterative approach}

	The idea of the iterative method is, that the importance of the individual external parameters are very different, that means if we have the list of possible references, we take all references where the temperature difference is minimal, so we get a new, much smaller list of possible references. From this list we choose all references where the next external parameter for example the daytime is minimal and again get a new list. We proceed this way with the following external parameters. We experiment with the sequence of the parameters, to increase the success of the method. The final sequence is:
	\begin{equation*}
	Temperature \bullet  Daytime  \bullet Colorindex \bullet Temporal\,\, Difference \bullet Exposure \,\, time
	\end{equation*} 
	\subsection*{Advantages of the iterative approach}
	The advantages of the iterative approach is the simple calculation and thus the short calculation time. Since the temperature is by far the external impact with the largest impact on the evaluation it is reasonable to look at first at the temperature, and after that on the other parameters. \\
	Another advantage is, that the external conditions of the resulting references, are in an well-defined framework, similar to the  conditions of the plume. Therefore the systematic errors resulting from the effects on the BrO SCDs (see \Cref{fig:difftempbroallinstruments}) can be kept small.
	\subsection*{Disadvantages of the iterative approach}
	The iterative approach leads to an rigid evaluation of the data. References, which have very similar conditions as the plume could be excluded of the evaluation due to an large difference in one external parameter. The decision for the best suited reference is based on one by one parameter. Thus it is impossible to look at all parameters by the same time.   This could lead to a less optimal evaluation when looking at the BrO error.
	As a result the iterative method has a worse X factor (X=1.73) compared to the nearest neighbour (X=2.24) and the fitting method (X=1.24).
	
	
	
	\section*{\textcolor{red}{summary}}
	
	
	
	

	
	\chapter{Results}
	\textcolor{red}{1) Verbesserung der Lesbarkeit: "Strategisches" von "Technischem"
		teilen. Alles was für beiden Vulkane/allgemein erklärt werden kann vorne
		oder hinten gesammelt besprechen. Die ganzen Prozentzahlen etc. in einer
		Tabelle sammeln und ansonsten nur mit minimaler Wortzahl im Text
		besprechen. Alle technischen Details der Bildbesprechung in die
		Bildunterschriften. Dann kann man je nach Ziel entweder strategische
		oder technische Abschnitte überspringen. Gerade ist es eher gemixt und
		zwingt einen alles zu lesen.}\\
		%
		\textcolor{red}{Bisher besprichst du v.a. die quantitativen Ergebnisse, d.h. wie
		ändern sich die Zeitreihen und Mittelwerte. Die qualitativen Ergebnisse
		sollte am Ende nochmal klar benannt und interpretiert werden. Z.B.
		siehst du eine klare BrO Kontaminierung. Angenommen dass die
		kontaminierende Fahne vom Vortag stammt, bedeutet das, dass die BrO
		Lebensdauer auch mindestens einen (halben) Tag lang sein müsste.}
	This chapter shows and discusses the difference of the \ce{SO2}, \ce{BrO} and \ce{BrO}/\ce{SO2}  ratio data when evaluating with the NOVAC-Method, or with the contamination based method.
	The aim is to discover the systematic differences between the different retrievals and to discuss the reliability of the data.\\
	\\
	To obtain the reference leading to an evaluation with minimal expected \ce{BrO}  error, the calculation of \Cref{calc:delterr} and the corresponding coefficients from \Cref{tab:coefTung} (Tungurahua) and \Cref{tab:coefNevad} (Nevado Del Ruiz) are used. 
	For the retrieval only "Multi Add" data are used. The maximal temporal difference between measuring the reference and the plume, accepted for the retrieval, is two weeks. The maximal accepted temperature difference is 3.3 $^{\circ}$C.\\

	\begin{figure}[h!]	
		\subfigure[Data of Tungurahua]{\includegraphics[width=0.5\textwidth]{Bilder/Tungurahua_Pic/tung_so2_novac_conbased}
		\label{fig:diffNovaca}}
		\subfigure[Data of Nevado Del Riz]{\includegraphics[width=0.5\textwidth]{Bilder/NevadoDelRuiz_Pic/so2_novac_conbased}
			\label{fig:diffNovacb}}
		\subfigure[Data of Tungurahua]{\includegraphics[width=0.5\textwidth]{Bilder/Tungurahua_Pic/tung_bro_novac_conbased}
		\label{fig:diffNovacc}}
		\subfigure[Data of Nevado Del Riz]{\includegraphics[width=0.5\textwidth]{Bilder/NevadoDelRuiz_Pic/bro_novac_conbased}
		\label{fig:diffNovacd}}

		\caption{Comparison of the results of contaminated data for performing the evaluation with the NOVAC-method and the contamination based method. The black solid line shows a linear fit of the data (m:slope, b:intercept). For the fit only data are used where the corresponding \ce{SO2}  column density (retrieved from the contamination based method) lies above the plume limit of $SO2\_SCD>7\cdot 10^{17}$. The dotted line indicates the unity line. (a) Results for the \ce{SO2}  column densities from Tungurahua; (b) Results for the \ce{SO2}  column densities from Nevado Del Ruiz; (c) Results for the \ce{BrO}  column densities from Tungurahua; (d) Results for the  \ce{BrO} column densities from Nevado Del Ruiz}
		\label{fig:diffNovac}
	\end{figure}
\begin{figure}[h!]		
	\subfigure[]{\includegraphics[width=0.5\textwidth]{Bilder/Tungurahua_Pic/tung_ratio_novac_conbased}
		\label{fig:diffratioa}}
	\subfigure[]{\includegraphics[width=0.5\textwidth]{Bilder/NevadoDelRuiz_Pic/ratio_novac_conbased}
		\label{fig:diffratiob}}
	\subfigure[]{\includegraphics[width=0.5\textwidth]{Bilder/Tungurahua_Pic/tung_ratio_diff_novac_conbased}
		\label{fig:diffratioc}}
	\subfigure[]{\includegraphics[width=0.5\textwidth]{Bilder/NevadoDelRuiz_Pic/ratio_diff_novac_conbased}
		\label{fig:diffratiod}}
	\caption{\textcolor{red}{simon hat hier sehr viel geschrieben}Comparison of the results of contaminated data for performing the evaluation with the NOVAC-method and the contamination based method.  (a)+(b): The black solid line shows a linear fit of the data (m:slope, b:intercept). For the fit only data are used where the corresponding \ce{SO2}  column density (retrieved from the contamination based method) lies above the plume limit of $SO2\_SCD>7\cdot 10^{17}$. The dotted line indicates the unity line. (a) Results for the \ce{BrO}  /\ce{SO2}  column densities from Tungurahua; (b) Results for the \ce{BrO}  /\ce{SO2}  column densities from Nevado Del Ruiz. 
		(c) +(d) The column density calculated with NOVAC is subtracted from the corresponding column density retrieved with the contamination based method. The black valid line indicates a linear fit of the data, the dotted grey line shows where the difference is zero. That means both evaluations lead to the same ratio. (c) Tungurahua (d) Nevado Del Ruiz \textcolor{red}{FL:Die Achsen sollten soweit wie möglich gleiche Bereiche zeigen. Wenn eine Plot quantitative ganz aus der Reihe fällt, dann kann man die Bereiche ausnahmsweise ändern (und ggf. in der Bildunterschrift darauf hinweisen).}}
	\label{fig:diffratio}
\end{figure}
	We are interested in the systematic differences if using the contaminated same time references spectrum or a gas free alternative reference.  
	\Cref{fig:diffNovac} shows a comparison of the results of contaminated data if performing the evaluation with the NOVAC-method and the contamination based method. 
	%Only data are used where the \ce{SO2} SCDs evaluated with the contamination based method  lies above the plume limit ($SO2\_SCD>7\cdot 10^{17}\frac{molec}{cm^2}$). Thus, the corresponding \ce{SO2}  SCDs evaluated with NOVAC could be below $7\cdot 10^{17}$. 
	As expected the \ce{SO2}  column densities retrieved by the contamination based method are larger for almost every measurement compared to the NOVAC method. The difference between the \ce{SO2} SCDs retrieved by the conventional NOVAC-method increases slightly with increasing SO2 SCD. Fitting of the data results in a slope of 1.22 (Tungurahua) and 1.16 (Nevado del Ruiz). Thus, we can conclude that  a higher SO2 amount leads to a slightly higher contamination (see \Cref{fig:diffNovac}). The extent of BrO contamination does not seem to depend on the BrO amount. The slope of the linear fit of the BrO SCDs have a slope which does not differ significantly from one. The slope at Tungurahua is 1.01 and for Nevado del Ruiz  is 1.07. \\
	\Cref{fig:diffratio} shows the difference of the \ce{SO2}/\ce{BrO} ratio when performing the evaluation with the NOVAC method and the contamination based method.\\
	(\Cref{fig:diffratiob} , \ref{fig:diffratiod}). \Cref{fig:diffratioa} and \ref{fig:diffratiob} show the results of the contamination based method plotted against the results of the NOVAC method. \Cref{fig:diffratioc}  and \ref{fig:diffratiod} show the actual difference between both methods. The column density calculated with NOVAC is subtracted from the corresponding column density retrieved with the contamination based method.  For low \ce{BrO}/\ce{SO2}  ratios around below zero, the ratios calculated with the contamination based method are higher than the ratios retrieved with the NOVAC method. For higher \ce{BrO}/\ce{SO2}  ratios (above zero) the ratios calculated with the NOVAC method are larger. The absolute difference between both evaluation methods increases with the increase of the 
\textcolor{red}{postives Offset, was ist der Effekt von der höheren Anzahl Datenpunkte}
	Due to the increase of the \ce{SO2}  column density when performing the evaluation with the contamination based method more \ce{SO2}  SCD lies above the plume limit of $7\cdot10^{17}\quad \frac{molec}{cm^2}$. This leads to an increase of the amount of data which show a large \ce{SO2} signal.\\ 
	In the following we discuss the results for the Tungurahua volcano and the Nevado Del Ruiz. 




\begin{table}[h!]
	\subfigure{
		\begin{tabular}{p{5cm}p{4cm}p{3cm}p{3cm}}
			&&Tungurahua&Nevado Del Ruiz\\
			\toprule
			Total amount of scans&&6500&14005\\
			\midrule
			NOVAC: SO2 SCDs above plume limit&&0.064\%& 0.121\%\\
			\midrule
			NOVAC: SO2 SCDs above plume limit without cont.&& 0.055\%& 0.088\%\\
			\midrule
			contaminated scans&&6\%&9.9\%\\
			\midrule
			&NOVAC: above  plume limit & 0.153&0.328\\
			\midrule
			&Contbased: above plume limit& 0.299&0.598\\			
			\midrule
			Not conatminated data: above plume limit &&0.055&0.088\\
			\midrule
			In total: above plume limit &&0.073&0.14\\
			\bottomrule			
	\end{tabular}}
	\caption{Results for Tungurahua and Nevado Del Ruiz. The absolute amount of data and the amount of data above the plume limit and the amount of contaminated data are shown.  Results for Tungurahua and Nevado Del Ruiz. The temporal difference between the measuring times of plume and reference is not longer than $\pm$2week. The maximal temperature difference is smaller then 3.3$^{\circ}$C ($\Delta T < 3.3^{\circ}$C). The maximal relative BrO error for the contaminated data is not larger as the corresponding BrO SCD. }
	\label{tab:calc1}
\end{table}



\section{Tungurahua}
	In the considered time interval from June 2008 to August 2009, 6500 multi add spectra are available. \\
	If contamination is not considered
	6.4\% of the evaluated volcanic gas plumes contained \ce{SO2} above the "plume threshold of $7\cdot10^{17}\quad \frac{molec}{cm^2}$. Thus 6.4\% of the data can be used for the examination of the volcanic gas emissions in this timespan.\\
 	The analysis of the reference with an solar atlas spectrum spectra found 6.0 \% of all spectra as contaminated. To prevent a systematic error of the results this spectra need to be excluded from the conventional NOVAC-evaluation if no further calculations are made.\\
 	From the remaining data,  if the contaminated spectra are excluded, only 5.5 \% of the \ce{SO2}  column densities are above the plume limit  (see  Table \ref{tab:calc}). \\
 	A higher amount of spectra above the \ce{SO2}  plume limit can be found in the contaminated data even if the contaminated data are evaluated with the NOVAC-method. Here, the percentage of data in the plume-limit is 0.153\%.
 	This means that the contaminated data are 2.381 times more frequently above the plume limit. This leads to the presumption, that the probability of getting contaminated data increases with the gas amount leaving the volcano.\\
 	The following paragraph deals only with the contaminated data.	
 	When performing the evaluation of the contaminated data with the contamination based method 0.299\% of the resulting \ce{SO2}  column densities are in the plume limit. Thus, the reliable amount of contaminated data increase bya factor of two while the total amount of data increase by 1.8 percent points. Thus 0.073\% of all data are above plume limit when using the contamination based method.\\
 	By using trace gas free references instead of contaminated references about 25.2 \% more data are available. \\

%	\begin{figure}
%		\centering
%		\includegraphics[width=0.7\linewidth]{Bilder/tung_so2_novac_sametime}
%		\caption{Time series of \ce{SO2} column densities at Tungurahua. Blue data points are below the detection limit, green data points are not contaminated, valid \ce{SO2} SCDs. Red data points are contaminated data, evaluated with the contamination based method.}
%		\label{fig:tungso2novacsametime}
%	\end{figure}
	


\section{Nevado Del Ruiz}
At Nevado Del Ruiz a larger time span (from the end of 2009 to the end of 2011) is examined. The resulting amount of "Multi Add" data is 14005.
Neglecting contamination and thus evaluate all data using the conventional NOVAC method results in 0.121 of all data above the plume limit. 
%When ignoring contamination and evaluating all spectra with the NOVAC-method 12.8\% of the \ce{SO2} SCDs are above the plume limit.\\
The total amount of contamination data is 1392. This is equivalent to 9.9\% of all data. Evaluating the contaminated data with the NOVAC-method yields in 32.8\% of the cases in SO2 SCDs above the plume limit. As at Tungurahua the occurrence of data above the plume limit within the contaminated data is larger as for not contaminated data (by a factor of 2.7).\\
The following paragraph only deals with the contaminated data.
If using the contamination based method 0.598\% of the \ce{SO2} column densities are above the detection limit. Thus, an increase of the number of detected \ce{SO2} SCDs above the plume limit within the contaminated data of 95\% is observable. In total the number of \ce{SO2} SCDs above the plume limit increases by 2.7 percent points. As a result 14.8\% of all data are above the plume limit.\\
\FloatBarrier


%Menge an Daten insgesamt: ----------------------------------------------------- 14005 ≙ 1
%Davon: Menge an daten (auch kontaminierten)(NOVAC) über plume limit--------- 1793  ≙ 0.128
%Davon: Menge an Daten, die nicht Kontaminiert sind:------------------------- 12613 ≙ 0.901
%Davon im Plume-limit:  ------------------------------------------------ 1238  ≙ 0.088
%Davon über dem Detection Limit:---------------------------------------- 234   ≙ 0.017
%Davon sind kontaminiert: ---------------------------------------------------- 1392  ≙ 0.099
%Menge an Daten die nicht ausgewertet werden konnten:------------------- 306  ≙ 0.078
%Menge an kontamininierten Daten, mit NOVAC, über plume limit:---------- 556   ≙ 0.399 der kont daten
%Menge an contaminierten Daten (Neue Auswertung) über plume limit------- 1085  ≙ 0.779
%
%Dh in den kontaminierten daten sind mit NOVAC ausgewerteten daten 3.120 häufiger über dem plume limit
%
%	Dh in den kontaminierten daten sind mit NOVAC ausgewerteten daten \underline{3.449} häufiger über dem plume limit\\
%
\section{BrO/\ce{SO2}  time series}

This section presents the final time series of SO2, BrO and \ce{BrO}/\ce{SO2}  for Tungurahua and Nevado Del Ruiz.\\


%	\begin{figure}
%		\subfigure{
%		\includegraphics[width=0.49\linewidth]{Bilder/Results/Results_Tungurahua}}
%		\subfigure{
%		\includegraphics[width=0.49\linewidth]{"Bilder/Results/Results_NevadoDelRuiz (1)"}}
%		\caption{Time series of the BrO/\ce{SO2} ratios for Nevado del Ruiz (left) and Tungurahua (right). The contaminated data are evaluated by using the contamination based method. The \ce{SO2} SCDs used for this plot are above the plume limit.}
%		\label{fig:resultsnevadodelruiz-1}
%	\end{figure}

\begin{figure}
	\centering
	\subfigure{
		\includegraphics[width=1\linewidth]{Bilder/tung_so2_novac_sametime}}
	\subfigure{
	\includegraphics[width=1\linewidth]{Bilder/tung_bro_sametime}}
	\subfigure{
	\includegraphics[width=1\linewidth]{Bilder/tung_rat_sametime}}

	\caption{Time series of the BrO/\ce{SO2} ratios for Tungurahua. The contaminated data are evaluated by using the contamination based method.  Blue data points are below the detection limit, green data points are not contaminated, valid \ce{SO2} SCDs. Red data points are contaminated data, evaluated with the contamination based method.}
	\label{fig:tungso2novacsametime}
\end{figure}
\begin{figure}
	\centering
	\subfigure{
		\includegraphics[width=1\linewidth]{Bilder/ndr_so2_novac_sametime}}
	\subfigure{
		\includegraphics[width=1\linewidth]{Bilder/ndr_bro_sametime}}
	\subfigure{
		\includegraphics[width=1\linewidth]{Bilder/ndr_rat_sametime}}
	\caption{Time series of BrO, SO2 and the BrO/\ce{SO2} ratios for Nevado del Ruiz . The contaminated data are evaluated by using the contamination based method. Blue data points are below the detection limit, green data points are not contaminated, valid \ce{SO2} SCDs. Red data points are contaminated data, evaluated with the contamination based method. The contaminated data points are only marked in red, if the \ce{SO2} SCD is above the plume limit. Contaminated data below the detection limit are not particularly marked. }
	\label{fig:ndrso2novacsametime}
\end{figure}	

	%
	Contaminated data either need to be excluded from the evaluation or to corrected due to using a not contaminated reference. If the contaminated data are excluded from the evaluation, the amount of data where the corresponding SO2 amount is above the plume limit is significantly smaller. The additionally data retrieved by using the contamination based method can be seen in \Cref{fig:tungso2novacsametime} for Tungurahua (visualized with red circles) respectively for Nevado del Ruiz in \Cref{fig:ndrso2novacsametime}.
	The time series of Tungurahua and Nevado del Ruiz show, that contamination occurs rather frequently at high SO2 and BrO column densities. The share of contaminated data above a SO2 SCD of $2e18\frac{molec}{cm^2}$ is significantly higher as below (see \Cref{fig:ndrso2novacsametime} for Nevado del Ruiz and \Cref{fig:tungso2novacsametime} for the Tungurahua volcano). High BrO SCDs (above $1e14\frac{molec}{cm^2}$) are as well above average contaminated data.\\
	While the SO2 SCDs and BrO SCDs increase on average due to using the contaminated data the average BrO/SO2 ratio does not chance significantly compared to the average ratio when only using not contaminated data, as can be seen in \Cref{fig:tungso2novacsametime} for Tungurahua (the plot at the bottom) or in \Cref{fig:ndrso2novacsametime} for Nevado del Ruiz.\\

	
%	\Cref{fig:ndrso2novacsametime} shows the results for the Nevado del Ruiz volcano. The plot at the top of \Cref{fig:ndrso2novacsametime} shows the results for SO2. If the contaminated data would be excluded from the evaluation only the green data points can be used for the evaluation of the time series.\\ 
%	The middle plot of \Cref{fig:ndrso2novacsametime} shows the corresponding BrO time series. The green data points indicates BrO SCDs where the corresponding \ce{SO2} amount is above the detection limit, while the red BrO SCDs show contaminated data above the detection limit.\\
%	The plot at the bottom of \Cref{fig:ndrso2novacsametime} show the BrO/\ce{SO2} ratio time series. The increase of usable data can be seen.
%	\Cref{fig:tungso2novacsametime} shows the equivalent plot for the Tungurahua volcano.\\
	
	\begin{figure}
		\subfigure{
		\includegraphics[width=0.5\linewidth]{Bilder/dailymeansNevadoDelRuiz}}
		\subfigure{
			\includegraphics[width=0.5\linewidth]{Bilder/dailymeanstungurahua}}
		\caption{
			BrO/\ce{SO2} ratio daily mean time series for Nevado del Ruiz (left) and Tungurahua (right). \ce{SO2} SCDs used for this plot are above the plume limit of 7$\cdot 10^{17}$. The minimum amount of data per day are four. Days where less than four valid ratios are recorded are not considered in this plot. As a result of using the contaminated data as well more daily means are available, since more days have an amount of valid data's above four. Those dates are marked with red. The other data are colored in green.}
		\label{fig:dailymeanstungurahua}
	\end{figure}
	\subsection*{Daily Means}
	
	 Due to the very small amount of volcanic gas plumes with 
	\ce{BrO}  column densities above the detection limit often the daily mean of the \ce{BrO}/\ce{SO2}  ratios is used. Hereby at least four "Multi Adds" per day above the plume limit are considered in order to avoid outliers. Thus, performing the evaluation of contaminated data with the contamination based method leads to more data.\textcolor{red}{wie gehts eigentlich dem mitteleren BrO/SO2 fehler? (also (BrO/SO2)/sqrt(datenpunkte an dem tag) )}
	%Some days which had less than  three to four valid data points could then have four or more multi adds. 
	At Tungurahua In fact 30.8\% more daily mean data can be retrieved when using the contamination based method compared to exclude the contaminated data. The amount of daily means increases less than the total amount of data, this effect can be explained due to a higher occurrence of contamination if the \ce{SO2}  column densities are high, thus more data are retrieved for days with a high \ce{SO2}  amount.
	In the considered time period we have data on 365 days. 
	If we exclude the contaminated data from the evaluation we get at 36 days daily means (more than 4 valid data points per day). If the contamination problem is neglected at 43 days daily means are retrieved. At 52 days we can calculate daily means if we use the contamination based method this is equivalent to 14.\% of all data.\\

	At Nevado del Ruiz the amount of daily means increases by 27.9\%. if the contaminated data are used compared to the scenario where the contaminated data are excluded.
	In total we have data at 688 days.	When excluding the contaminated data it total there are 165.
	If the contamination problem is neglected the total amount of daily means in the evaluated time period is 198. And if we use the contamintion based method for the contaminated data we get an total amount of 229 data. This is  equivalent to 33\% of all data. Thus we have twice at much daily means at Nevado del Ruiz compared to Tungurahua.
	
	\Cref{fig:dailymeanstungurahua} shows a time series of Daily means of the BrO/\ce{SO2} ratio. The minimum amount of valid data point within one day is above 4. Days where 3 or less valid data where recorded are not considered in this figure.  Thus all considered ratios have an \ce{SO2} SCD above the plume limit. The red marked ratios shows daily means which includes at least one contaminated spectra.
%Menge an Daten insgesamt: ----------------------------------------------------- 14005 ≙ 1
%Davon: Menge an daten (auch kontaminierten)(NOVAC) über plume limit--------- 1793  ≙ 0.128
%Davon: Menge an Daten, die nicht Kontaminiert sind:------------------------- 12613 ≙ 0.901
%Davon im Plume-limit:  ------------------------------------------------ 1238  ≙ 0.088
%Davon über dem Detection Limit:---------------------------------------- 234   ≙ 0.017
%Davon sind kontaminiert: ---------------------------------------------------- 1392  ≙ 0.099
%Menge an Daten die nicht ausgewertet werden konnten:------------------- 306  ≙ 0.078
%Menge an kontamininierten Daten, mit NOVAC, über plume limit:---------- 556   ≙ 0.399 der kont daten
%Menge an contaminierten Daten (Neue Auswertung) über plume limit------- 1085  ≙ 0.779
%
%Dh in den kontaminierten daten sind mit NOVAC ausgewerteten daten 3.120 häufiger über dem plume limit
%
%	Dh in den kontaminierten daten sind mit NOVAC ausgewerteten daten \underline{3.449} häufiger über dem plume limit\\
\FloatBarrier
	\section{Issues of the contamination based method}
	\textcolor{red}{diese Besprechung ist vielleicht vorne im Kapitel zu Kontaminierung besser aufgehoben. Da dann 8.6. als realistischen/klassischen Fall beschreiben. Also ohne Korrektur unterschätzung, mit korrektur überschätzung. Den unkorrigeten Fall kennen wir schon, du (und vor dir auch Lübcke et al 2016) untersuchen jetzt den korrigierten (überschätzen Fall). Die Wahrheit liegt vermutlich irgendwo dazwischen aber können wir zum heutigen Stand nicht ermitteln. Dann ist klar was du vor hast und wie die folgenden ergebnisse interpretiert werden müssen/dürfen. Und es gibt nicht am ende der Arbeit ein großes Aber. Figure 8.7 kann/sollte auch schon vorne kommen. Mit dem Verweis, dass wir solche Fälle ignorieren bzw. annehmen dass nur 8.6. vorliegt.}
	The contamination method is arguable because it does not consider contamination of the plume due to a lack of information. Moreover, we do not consider the different lifetimes of \ce{SO2} and BrO. This leads to a faster depletion of BrO.
	\textcolor{red}{die besprechung der Lebensdauern muss noch ausführlicher werden. Insbesondere implizieren deine Daten, dass die BrO Lebensdauer relative lange ist. Ansonsten wäre ja zumindest eine Kontaminierung vom Tag davor nicht möglich. Das sollte auf jeden Fall einen ausführlich besprochenes ergebnis sein. Passend dazu eine Frage: Ist die Kontaminierung eigentlich vornehmlich morgens aufgetreten? Denn nachts kommt ja genauso viel \ce{Hbr} aus dem Vulkan, nur dass es eben nicht umgesetzt wird. (Memo an mich selbst!! :-D )}
	\subsection{Contamination of the plume}
	\textcolor{red}{UP:sollte dieser Para. nicht vor den Ergebnissen kommen?}
	\begin{figure}
		\centering
		\includegraphics[width=0.7\linewidth]{Bilder/Contaminationplume}
		\caption{Visualization of the contamination of the plume. Due to a lack of wind the old plume sinks down an accumulates
		above the instrument. The light path trough the old plume is longer when recording the reference spectra (orange).}
		\label{fig:contaminationplume}
	\end{figure}
		\begin{figure}
			\centering
			\subfigure{\includegraphics[width=0.49\linewidth]{Bilder/Contaminationplumeinstrinplum}}
			\subfigure{\includegraphics[width=0.49\linewidth]{Bilder/Contaminationplumewideplume}}
			\caption{Visualization of possible scenarios for contamination. Left: the plume sinks in a way that the instrument is within the plume, therefore, all elavation angles will contain volcanic trace gases, while the plume is not additionally contaminated. Right: the plume covers the hole sky, thus all elvation angles "see" volcanic trace gases. }
			\label{fig:contaminationplumewideplume}
		\end{figure}

	 As discussed above it occurs, that the plume is contaminated as well (see \Cref{fig:contaminationplume}). This might be the case if the volcanic gas of the volcano is not taken away by the wind, but accumulates in the vicinity of the volcano and thus is seen by the instrument. In such a scenario, using an contamination free reference recorded at another time would lead to an overestimation of the gas column densities in the plume.\\
	 This is one possible occurence of contamination. As it can be seen gas of the old plume affects the measurement of the reference and the plume. However, for this example the influence on the measurement of the reference is much larger since the light path trough the old plume (coloured orange in \Cref{fig:contaminationplume}) is longer for the reference than for the measurement of the volcanic plume. Thus, we underestimate the gas amount if we do not use a gas free reference, but might overestimate if we do so.
	 The real gas amount might be between the measured amount with and without using a reference measured at another time.\\
	 The contamination setup could differ from \Cref{fig:contaminationplume}. This would lead to different results. However, the reference region is in most cases at a larger elevation angle than the plume. Thus, the assumption that the light path trough the old plume is in average longer for the reference, if both, reference and plume are contaminated.
	 
	  With the data retrieved by the NOVAC instruments it is very difficult or even impossible to discover whether the plume is contaminated or not. \\
%	
%	\textcolor{red}{
%	\cite{geeignete quelle, irgendwo hab ich das schon gehoert} made investigations on the lifetime of \ce{SO2} and BrO. The lifetime of \ce{SO2} is several weeks, while the decay of BrO proceed in smaller time scales. As a result we should observe a lower contamination for BrO. This can be observed in our results. In theory it can be that due to the smaller lifetimes of BrO contamination only influences the \ce{SO2} evaluation, if that is the case, the best option would be to use a high defined solar atlas spectrum as reference for the \ce{SO2} evaluation, but an reference recorded with the same instrument at the same time for the BrO evaluation, since BrO could not be retrieved using an high defined solar atlas spectrum.}


	\chapter{Conclusion}
	
	The Network for Observation of Volcanic and Atmospheric Change provides a very large of longtime timeseries. This treasure of the data contains a huge amount of informations about volcanoes, which give an interesting insight of the physical foundations on volcanic activity.\\
	The evaluation of the volcanoes is is in particular based on the assumption that the atmospheric background of the volcanic trace gases \ce{SO2} and BrO is negligible. This is important to assure since the DOAS method only captures the difference in gas amount between the background, the reference and the signal, the plume spectra.\\
	By the usage of a solar atlas spectrum as reference, \citet{lubcke2014bro} found  that the reference provided by the NOVAC-Instrument is not always gas free. Thus is contaminated with volcanic trace gases. If that is the case the contaminated spectra either needs to be excluded from the evaluation or the contaminated reference needs to be replaced by a gas free reference. If we want to keep the contaminated data we need to find an appropriate gasfree reference. A solar atlas spectrum can not be used for the BrO evlaution due to the low amount of BrO and the increase of the BrO error if using a solar atlas spectrum. Thus, a gasfree reference recorded by the same instrument is a possible choice.\\
	As the instruments are recoring continously, there are multiple potential gasfree reference spectra available for each plume spectrum. Due to the low accuracy of the BrO retrieval it is reasonably to choose the reference with respect to the BrO error.
	An analysis of the if the dependency of the BrO error on several external parameters is presented. Hereby we focus on the temporal difference and the differences in temperature, Colorindex, exposure time, elevation angle and daytime between the plume and the reference spectra. These external parameters are found crucial for the retrieved BrO error. The dependence of the BrO column density on external parameters was discussed as well. 
	\\
	The result of this analysis, that in general the retrieved BrO error is minimal, if the considered parameter are similar for the plume and reference. Only the elevation angle does not seem to influence the retrieval. From the considered parameter the temperature has the most significant impact of 60\% up to 97\% compared to the other parameters, at all observed instruments at Tungurahua as well as at the Nevado Del Ruiz volcano. The importance of the other instruments are different for every instrument, depending on the location of the instrument and the instrument itself.\\
	If both, the the reference and the plume a recorded at the same time all those external parameter are  identical which corresponds to the minimal expected error however might include involve a systematic error due to contamination.\\
	Based on this findings an algorithm is introduced to automatically find the alternative reference if the reference, which is measured at the same time as the plum spectrum, to be replaced due to contamination.\\
	Fitting the data with a four dimensional first order polynomial turned out to give the best estimation of the expected error on the trace gas measurement.\\
	The evaluation of the contaminated data with the here proposed algorithm shows that BrO as well as \ce{SO2} was underestimated by ignoring the contamination problem.
	We shows that the contamination based method in average estimates the value of the BrO column destiny by 1.2$\cdot$ e+13 higher at Tungurahua (2.0e13 Nevado Del Ruiz). The \ce{SO2} amount increases as well by an amount of 6.5$\cdot$ e+17 at Tungurahua (7.4$\cdot$ e+17 Nevado Del Ruiz) . Thus, the results for the BrO/\ce{SO2} ratio calculated with the contamination based method changed compared to the conventional method in the way, that low ratios increases, while high ratios decrease.\\
	A further advantage of the new calculation is, that due to the higher amount of the \ce{SO2} SCDs the amount of data above the plume limit increases. Using the contamination based method leads to an total increase of data of 45 \% (Tungurahua) and 87\% (Nevado Del Ruiz).\\
	\\
	The Method proposed in this thesis improves the evaluation of the spectra provided by NOVAC and eliminates the systematic error occurring due to contamination. As a result the amount of usable data increases. This is a important step fo a better understanding of volcanoes.


	
	
	



  \part{Appendix}
  \begin{appendix}
	\begin{figure}
		\subfigure[]{\includegraphics[width=0.45\linewidth]{Bilder/WelcheEP/D2J2140}}
		\subfigure[]{\includegraphics[width=0.45\linewidth]{Bilder/WelcheEP/D2J2200_0}}
		\subfigure[]{\includegraphics[width=0.45\linewidth]{Bilder/WelcheEP/D2J2200_1}}
		\subfigure[]{\includegraphics[width=0.45\linewidth]{Bilder/WelcheEP/D2J2201_0}}
		\subfigure[]{\includegraphics[width=0.45\linewidth]{Bilder/WelcheEP/I2J8546}}
		\subfigure[]{\includegraphics[width=0.45\linewidth]{Bilder/WelcheEP/I2J8548}}
		\caption{}
		\label{fig:I2J8548}
		\end{figure}
	\subsection*{Dependency external parameters on each other}
			\textbf{Difference of Temperature on  Difference of exposure time}
	\begin{equation}
	\Delta T =  -26.32\cdot \Delta ExpTime + 2\cdot 10^{-15}
	\end{equation}
	
	\textbf{Difference of Temperature on  Difference of colorindex}
	\begin{equation}
	\Delta T = 0.0022\cdot \Delta ColIdX +2\cdot 10^{-19}
	\end{equation}
	
	\textbf{Difference of Temperature on  Difference of Day time}
	\begin{equation}
	\Delta T =0.262\cdot \Delta daytime -4\cdot 10^{-17}
	\end{equation}
	
	\textbf{Difference of Temperature on  Difference of Elevation angle}
	\begin{equation}
	\Delta T =1.08\cdot \Delta Elev Angle -1\cdot 10^{-16}
	\end{equation}
	
	\textbf{Difference of Exposure on  Difference of  Col Idx}
	\begin{equation}
	\Delta Exposure  =-6.22\cdot 10^{-5} \Delta Col Idx  +1\cdot 10^{-18}
	\end{equation}
	
	\textbf{Difference of  Exposure on  Difference of Day time}
	\begin{equation}
	\Delta Exposure  =-0.004\cdot \Delta daytime -1\cdot 10^{-17}
	\end{equation}
	
	\textbf{Difference of  Exposure  on  Difference ofElevation angle}
	\begin{equation}
	\Delta Exposure  =-0.047\cdot \Delta  ElevAngle +3\cdot 10^{-16}
	\end{equation}
	
	
	\textbf{Difference of  Colorindex  on  Difference of Day time}
	\begin{equation}
	\Delta ColIdX  =4.51\cdot \Delta daytime-1.2\cdot 10^{-15}
	\end{equation}
	
	\textbf{Difference of  Colorindex  on  Difference of Elevation angle}
	\begin{equation}
	\Delta ColIdX  =-52\cdot \Delta ElevAngle+ 1.45\cdot 10^{-14}
	\end{equation}
	
	\textbf{Difference of  Colorindex  on  Difference of Elevation angle}
	\begin{equation}
	\Delta ColIdX  =3.5\cdot \Delta ElevAngle-6\cdot 10^{-16}
	\end{equation}
	\subsection*{Figures to more plumes to one ref}
	\begin{figure}
		\subfigure[]{
		\includegraphics[width=0.5\linewidth]{Bilder/OneRefMorePlumes_refdate_081115_1337/D2J2140_0_onerefColorindex_BrO_TemporalDiff}
			}
		\subfigure[]{
		\includegraphics[width=0.5\linewidth]{"Bilder/OneRefMorePlumes_refdate_081115_1337/D2J2140_0_onerefDaytime (Plume)_BrO_TemporalDiff"}
				}
		\subfigure[]{
		\includegraphics[width=0.5\linewidth]{"Bilder/OneRefMorePlumes_refdate_081115_1337/D2J2140_0_onerefElevation Angle [deg]_BrO_TemporalDiff"}
				}
		\subfigure[]{
		\includegraphics[width=0.5\linewidth]{"Bilder/OneRefMorePlumes_refdate_081115_1337/D2J2140_0_onerefExposure Time [ms]_BrO_TemporalDiff"}
			}\\
		\subfigure[]{
		\includegraphics[width=0.5\linewidth]{Bilder/OneRefMorePlumes_refdate_081115_1337/onerefD2J2140_0_6_02_18_DiffColorindex_BrO}
			}
		\subfigure[]{
		\includegraphics[width=0.5\linewidth]{"Bilder/OneRefMorePlumes_refdate_081115_1337/onerefD2J2140_0_6_02_18_DiffElevation Angle [deg]_BrO"}
			}\\
	
		\subfigure[]{
		\includegraphics[width=0.5\linewidth]{"Bilder/OneRefMorePlumes_refdate_081115_1337/onerefD2J2140_0_6_02_18_DiffExposure Time [ms]_BrO"}
			}
		\caption{}
		\label{fig:donerefd2j2140060218difftemperature-cbro}
	\end{figure}
\subsection*{Evaluation of the dependence of the BrO SCD on external dependence}
	\begin{figure}
		\subfigure[]{
		\includegraphics[width=0.45\linewidth]{Bilder/Appendix_BrOTotalevaluation/DiffColidxBrOwithoutOtherparamallInstruments}
				}
		\subfigure[]{
		\includegraphics[width=0.45\linewidth]{Bilder/Appendix_BrOTotalevaluation/DiffDaytimeBrOwithoutOtherparamallInstruments}
				}
		\subfigure[]{
		\includegraphics[width=0.45\linewidth]{Bilder/Appendix_BrOTotalevaluation/DiffElevAngleBrOwithoutOtherparamallInstruments}
				}
		\subfigure[]{
		\includegraphics[width=0.45\linewidth]{Bilder/Appendix_BrOTotalevaluation/DiffExpTimeBrOwithoutOtherparamallInstruments}
				}
		\subfigure[]{
		\includegraphics[width=0.45\linewidth]{Bilder/Appendix_BrOTotalevaluation/DatBrOwithoutOtherparamallInstruments}}
		\caption{}
		\label{fig:datbrowithoutotherparamallinstruments}
	\end{figure}
	\begin{figure}
		\subfigure[]{
			\includegraphics[width=0.45\linewidth]{Bilder/Appendix_BrOTotalevaluation/DatTotalBrOwithoutOtherparamallInstruments}
		}
		\subfigure[]{
			\centering
			\includegraphics[width=0.45\linewidth]{Bilder/Appendix_BrOTotalevaluation/DiffColidxTotalBrOwithoutOtherparamallInstruments}
		}
		\subfigure[]{
			\includegraphics[width=0.45\linewidth]{Bilder/Appendix_BrOTotalevaluation/DiffDaytimeTotalBrOwithoutOtherparamallInstruments}
		}
		\subfigure[]{
			\includegraphics[width=0.45\linewidth]{Bilder/Appendix_BrOTotalevaluation/DiffElevAngleTotalBrOwithoutOtherparamallInstruments}
		}
		\subfigure[]{
			\includegraphics[width=0.45\linewidth]{Bilder/Appendix_BrOTotalevaluation/DiffExpTimeTotalBrOwithoutOtherparamallInstruments}
		}
		\caption{}
		\label{fig:instruments}
	\end{figure}	
    \chapter{Lists}
    \listoffigures
    \listoftables
    \bibliography{references}{}
    \citestyle{egu}
    \bibliographystyle{plainnat}
    \include{deposition}
  \end{appendix}
\end{document}




