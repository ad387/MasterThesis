\documentclass[]{article}
\usepackage{graphicx} % Required for including images in figures
\usepackage{booktabs} % Required for horizontal rules in tables
\usepackage{multicol} % Required for creating multiple columns in slides
\usepackage[english]{babel} % Document language - required for customizing section titles
\usepackage{color}
\usepackage{booktabs} % Allows the use of \toprule, \midrule and \bottomrule in tables
\usepackage{multicol}
\usepackage{textpos}
\usepackage{tikz}
\usepackage{wrapfig}
\usepackage{hyperref}

%opening


\begin{document}


\section{Calcualtions}

\begin{equation}
\Delta \epsilon = \frac{d\epsilon}{dt}+\frac{d\epsilon}{d ^{\circ}}+\frac{d\epsilon}{dT}+\frac{d\epsilon}{ddt} +\frac{d\epsilon}{dc}
\end{equation}
With $\epsilon$ describes the BrO Error, t: time between plumetime and referencetime, T, temperaure; dt: daytime, c: colorindex\\
If we assume a that all differentiations are linear, than we get a 
equation system :

\begin{equation}
\Delta \epsilon = a_{t}\cdot\Delta t+a_{^{\circ}}\cdot\Delta ^{\circ}+a_{T}\cdot\Delta T+a_{dt}\cdot\Delta dt +a_{c}\cdot\Delta c
\end{equation}
Hereby are the constants 
\begin{table}[h]
\begin{tabular}{c|c}
\toprule
$a_{t}$&\\
\midrule
$a_{^{\circ}}$&\\
\midrule
$a_{T}$&\\
\midrule
$a_{dt}$&\\
\midrule
$a_{c}$&\\
\bottomrule
\end{tabular}
\end{table}
\section{Error as a function of Plume-Time and Reference Time}
As one can assume the mean BrO error raises with the time difference between the Plume and the reference. 
\\
picture
\\
One can observe that a possible time interval where the values of the BrO errors are still acceptable is about 3 days, otherwise the values of the calculated BrO are
in the majority negative due to instrument drifts. The following figures show how the data changes with the maximum time difference of 3 days - 72 hours
\end{document}
